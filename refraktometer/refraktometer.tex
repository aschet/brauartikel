% Copyright 2021 Thomas Ascher
% SPDX-License-Identifier: CC-BY-SA-4.0

\documentclass[a4paper,parskip=half]{scrartcl}

\usepackage[T1]{fontenc}
\usepackage[ngerman]{babel}
\usepackage{csquotes}
\usepackage[regular,condensed,sfdefault]{roboto}
\usepackage{booktabs}
\usepackage{graphicx}
\usepackage{chemformula}
\usepackage{amsmath,amsfonts,amssymb}
\usepackage{icomma}
\usepackage{textcomp}
\usepackage{gensymb}
\usepackage[italic,symbolgreek]{mathastext}
\usepackage{float}
\usepackage[backend=biber]{biblatex}
\usepackage[hidelinks,pdfencoding=auto,
  pdfauthor={Thomas Ascher},
  pdfusetitle,
  pdfkeywords={Bier,Refraktometer,Alkoholmessung,Terrill,Novotný}]{hyperref}
\usepackage{microtype}

\addto\extrasngerman{
\def\figureautorefname{Abb.}
\def\tableautorefname{Tab.}
\def\equationautorefname{Gl.}
}

\addto\captionsngerman{
\renewcommand{\figurename}{Abb.}
\renewcommand{\tablename}{Tab.}
}

\title{Alkoholmessung mit dem Refraktometer}
\author{Thomas Ascher <thomas.ascher@gmx.at>}
\date{14. August 2021, \href{http://creativecommons.org/licenses/by-sa/4.0/}{CC BY-SA 4.0}}

\addbibresource{refraktometer.bib}

\newcommand{\bxi}{\mathit{R}_i}
\newcommand{\bxitext}{$\smash{\bxi}$}
\newcommand{\bxic}{\mathit{Rc}_i}
\newcommand{\bxictext}{$\smash{\bxic}$}
\newcommand{\bxf}{\mathit{R}_f}
\newcommand{\bxftext}{$\smash{\bxf}$}
\newcommand{\bxfc}{\mathit{Rc}_f}
\newcommand{\bxfctext}{$\smash{\bxfc}$}
\newcommand{\sg}{\mathit{SG}}
\newcommand{\sgtext}{$\smash{\sg}$}
\newcommand{\fg}{\mathit{SG}}
\newcommand{\abv}{\mathit{ABV}}
\newcommand{\abvtext}{$\smash{\abv}$}
\newcommand{\abw}{\mathit{ABW}}
\newcommand{\abwtext}{$\smash{\abw}$}
\newcommand{\oex}{\mathit{OE}}
\newcommand{\aex}{\mathit{AE}}
\newcommand{\rex}{\mathit{RE}}
\newcommand{\rextext}{$\smash{\rex}$}
\newcommand{\wcf}{\mathit{WCF}}
\newcommand{\wcftext}{$\smash{\wcf}$}
\newcommand{\adf}{\mathit{ADF}}
\newcommand{\rdf}{\mathit{RDF}}

\begin{document}
\maketitle

\section*{Einleitung}

Mittlerweile stehen dem Heimbrauumfeld eine breite Auswahl von Instrumenten
mit verschiedenen Messprinzipien zur Verfügung, die es ermöglichen, den
Alkoholgehalt eines selbst gebrauten Biers abzuschätzen oder den
Gärverlauf zu beobachten. Das ist neben der klassischen Bierspindel
unter anderem das Refraktometer. Die Refraktometrie ist
definitiv kein Neuankömmling im Bereich der Bieranalyse. Versuche,
auf Basis dieses Messprinzips den Alkoholgehalt zu bestimmen, reichen bis
in das Jahr 1843 zurück \autocite{Gamer1959}.

Auch wenn das Refraktometer den Ruf besitzt, weniger genaue Messergebnisse
zu liefern als andere labortechnische Instrumente, liegen dessen
Vorteile klar auf der Hand: kurze Messzeiten bei geringem Probevolumen
und überschaubarer Bedienungskomplexität. Im Gegensatz zu den beim
Bierspindel üblichen 100 Millilitern pro Probe werden nur wenig
Milliliter benötigt, die sich dementsprechend auch schneller nach dem
Hopfenkochen auf Messtempertur abkühlen lassen. \autocite{Bettner1969, Terrill2011}

Wie nun Messungen mit einem Refraktometer während des Brau- und
Gärprozesses durchzuführen sind und sich darauf basierend der
Alkoholgehalt als Näherungswert berechnen lässt, wird
Rahmen dieses Artikels erörtert.

\section*{Messprinzip und Gerätetypen}

Wenn Licht auf eine Grenzfläche zwischen unterschiedlich optisch
dichten Medien (verschiedene Brechungsindexe) trifft,
kommt es zur Lichtbrechung und Reflexion. Dieser physikalische Effekt
wird in einem Refraktometer als Brechungsindex quantifiziert.
Maßgeblich zur Messwertbestimmung ist hierfür der materialabhängige
kritische Einfallswinkel, ab dem keine Brechung mehr stattfindet,
sondern das Licht in seiner Gesamtheit reflektiert wird.
Der Brechungsindex \(\mathit{nD}\) ist als Verhältnis zwischen der
Geschwindigkeit von Licht im Vakuum und einem Medium definiert. Je höher
die optische Dichte ist, desto größer ist auch der Brechungsindex.
\autocite{AKRSSOGH2021,Bonham2001,Gamer1959}

Refraktion ist temperaturabhängig. Messungen sind daher je nach
Anwendungsgebiet bei einer festgelegten Temperatur durchzuführen.
Das sind im Normalfall 20~°C. Selbst kostengünstige Handrefraktometer
besitzen deshalb einen Kompensationsmechanismus (Automatical Temperature
Compensation, ATC) in Form eines Bimetallsteifens. Dieser 
verschiebt je nach Temperatureinwirkung zur Korrektur selbstständig
die Messskala des Geräts. Hierbei ist zu beachten, dass diese Form der
ATC nur für den kalibrierten Probentyp des Geräts ausgelegt ist.
\autocite{Depalma2017,Distillique2020,Gossett2012,Terrill2013}

Der Brechungsindex ist mitunter keine geeignete Messgröße für alle
technischen Prozesse. Deshalb wird dieser je nach Anwendungsgebiet
und Probe auf die gewünschte Messgröße kalibriert. Zur Messung
der relativen Dichte von Saccharose/Wasser-Lösungen im Bereich der
Wein- und Saftherstellung ist diese zum Beispiel Grad Brix (°Bx), wobei
ein Grad Brix einem Gramm Saccharose auf 99 Gramm Wasser entspricht. Die
Umrechnung zwischen Brechungsindex und Grad Brix ist durch die ICUMSA
Tabellen normiert.
\autocite{Bonham2001,Terrill2013}

In den Shops für Heimbraubedarf werden im wesentlichen Refraktometer in
zwei verschiedenen Bauformen angeboten: analoge Handgeräte
und kompakte digitale Tischgeräte wie das für die Weinherstellung
gedachte \href{https://milwaukeeinstruments.eu/milwaukee-ma885-digital-brix-oechsle-oe-and-kmw-babo-refractometer/}{Milwaukee MA885} (\autoref{fig:refactotype},
\autoref{table:refactospec}). Beide Bauformen setzten Prismen zur
Lichtbrechung ein. Handrefraktometer funktionieren
nach dem Durchlichtprinzip. Vereinfacht dargestellt scheint dabei Licht
von einer externen Quelle durch die Probe, wird über Prismen gebrochen
und durch weitere optische Komponenten und eine Skala bis hin zu einem
Okular geleitet. Unterschiedliche kritische Winkel erzeugen dabei eine
andere Teilung des Sichtfelds, anhand der der Messwert abgelesen werden
kann (\autoref{fig:refractoscale}). Im Unterschied dazu besitzen
digitale Refraktometer eine interne Lichtquelle und erfassen die
Messung über einen CCD-Sensor. Darüber hinaus muss das Licht für
die Brechung die Probe nicht durchqueren. \autocite{AKRSSOGH2021,Gamer1959,Terrill2013}
 
\begin{figure}[h]
\centering
\includegraphics[width=4.8cm]{images/types.jpg}
\caption{Typische Refraktometer im Heimbraubereich}
\label{fig:refactotype}
\end{figure}

\begin{table}[h]
\centering
\begin{tabular}{lrr}
\toprule
Parameter &  Handrefraktometer &  Milwaukee MA885 \\
\midrule
Preis [€] & 40 & 185 \\
Messbereich [°Bx] & 0--32 & 0--50 \\
Auflösung [°Bx] & 0,2 & 0,1 \\
Genauigkeit [°Bx] & 0,2 & 0,1 \\
ATC [°C] & 10--30 & 10--40 \\
Maßeinheiten & °Bx, SG & °Bx, °Oe, °KWM \\
Kalibrierschein & nein & ja \\
\bottomrule
\end{tabular}
\caption{Spezifikation typischer Refraktometer im Heimbraubereich}
\label{table:refactospec}
\end{table}

\begin{figure}[h]
\centering
\includegraphics[width=4.8cm]{images/scale.jpg}
\caption{Durchsicht durch ein Handrefraktometer}
\label{fig:refractoscale}
\end{figure}

\section*{Exkurs Bierspindel}

Nachdem sich die Messabläufe und Berechnungswege zur Bestimmung des
Alkoholgehalts zwischen Refraktometer und Bierspindel großteils
decken, besteht die Notwendigkeit den Umgang mit der Bierspindel
näher zu erläutern. 

Mit der Bierspindel bzw. auch Saccharometer oder Aräometer wird
die relative Dichte (Specifiv Gravity, \sgtext) einer Flüssigkeit
nach dem Prinzip des Auftriebs bestimmt. Die spezifische Dichte
bezieht sich hierbei auf Wasser. Nachdem 


Auftrieb, = gewicht der verdrängten flüssigkeit, sinkt weiter
ein bei weniger dichten Würze.
100 bis 200 ml Probe
co2 in Lösung beeinflusst den Messwert um circa 0.3 P
Refraktometer weniger einfluss durch co2
Messung fg zu niedrig wegen Alkoholfehler
\autocite{Novotny2017}

Dichtemessung, geringere Dichte sinkt weiter ein, 20°C kalibriert.
masse \%, Zucker lösung, saccarose, von oben oder unten lesen je
nach angabe -> Flüssigkeitsmeniskus.
Würze bzw. Probe gut durchmischen. kein Wasser am Hydrometer trocken
sein, nicht zu tief eintauchen um Würze Anhaftung zu vermeiden
Korrektur notwenig bei anderer Temp -> Dichteänderung
Umrechnung zwischen dichte und Extraktgehalt erfolgt über Plato Tabelle
Alkoholfehler Messabweichung.
Balling Faktor 0.81
\autocite{Kunze2004}

Alkoholmessung bei 20°C Ethylalkohol Ethanol im Bier
Volumen Abhängig von Temperatur
0.3 \% Toleranz in den USA
SG = Einheitenlos, realtiv zu std Substanz z.B. Wasser in 20 °C
AE = änderung der Dichte, verfälscht durch Alkohol
RE = maß des verbleibenden Zuckers + Protein und Sonstige
g Extrakt / 100g Bier = Plato.
RDF real degree of fermentation
Hydrometer Prinzip Auftrieb, Messfehler durch Solids + Gase
Temp Kompensation notwendig,
SG oder Plato. 
\autocite{Spedding2016}

Sie wird in der Praxis mit Saccharometern durchgeführt, die auf dem Prinzip
der Aräometer beruhen. Nachdem ursprünglich die Saccharometer nach Balling
anhand einer Rohrzuckerlösung geeicht waren, ist an die Stelle der Ballings-
chen Tabelle die der Normaleichungskommission (PLATO) getreten [1039]. Die
Brauereisaccharometer sind als Thermosaccharometer ausgebildet, die die Wer-
te bei 20 8C angeben. Anhand einer Reduktionsskala werden bei Temperaturen
über 20 8C Korrekturwerte hinzugezählt, bei Temperaturen unter 20 8C dagegen
abgezogen. Die Genauigkeit der Saccharometeranzeige ist nur bei geeichten
Spindeln, die am besten in bestimmte Messbereiche eingeteilt sind (0–5, 5–10,
10–15, 15–20\%), zufriedenstellend.
Bei genauen Messungen, wie sie z. B. bei Sudhausabnahmen erforderlich
sind, muss eine Extraktbestimmung im Laboratorium, mittels Pyknometer oder
Biegeschwinger vorgenommen werden. Die zur Aufnahme der Würzeprobe die-
nenden Gefäße müssen trocken sein.

Der Extraktgehalt ist in Gewichtsprozenten angegeben, d. h. eine 12%ige
Würze enthält in 100 g: 12 g Extrakt und 88 g Wasser.
Um genaue Ergebnisse zu erhalten und Ablesefehler zu vermeiden, sind fol-
gende Punkte zu beachten:
a) das Saccharometer muss rein und trocken sein; es soll ungefähr jene Tem-
peratur haben wie die zu messende Flüssigkeit;
b) das Messgefäß muss rein und mit der zu spindelnden Flüssigkeit vorgespült
sein;
c) das Messgefäß muss in der Waage stehen; dies wird auch durch eine Kar-
danaufhängung bewirkt;
d) das Messgefäß muss so weit sein, dass das Saccharometer, ohne zu großen
Spielraum, bequem Platz findet;
e) vor dem Ablesen wird das Saccharometer mehrmals rasch eingetaucht und
wieder herausgezogen, um anhaftende Gasblasen zu entfernen;
f) die Ablesung des Saccharometers erfolgt am oberen Ende des sich an der
Spindel ausbildenden Meniskus der Flüssigkeit, sofern dies nicht eigens an-
ders vermerkt ist.
g) Bei Abkühlen der Würze wird durch Abdecken eine Verdunstung vermieden;
vor der Spindelung ist die abgekühlte Flüssigkeit durch mehrmaliges
Stürzen zu mischen.
Es ist von entscheidender Bedeutung, dass die Entnahme der Würzeprobe zur
Extraktermittlung und die Erfassung der Würzemenge in der Pfanne zur glei-
chen Zeit vorgenommen werden. Es besteht sonst die Gefahr, dass durch ein
weiteres Verdampfen von Wasser, z. B. bei stark ziehenden Pfannen, eine Verfäl-
schung der Ergebnisse der Ausbeuteermittlung eintritt.
Bei modernen Würzekochsystemen ist die Entnahme einer repräsentativen
Würzeprobe schwierig; es können sich, z. B. bei geöffneter Pfannentür Konzent-
rationsunterschiede durch Verdunstung oder generell Probleme durch Inhomo-
genitäten ergeben.
\autocite{Narziss2009}

\autocite{Distillique2020} Meniskus lesen einsinken

\begin{equation}
\sg\:[g/ml] = \frac{\degree P}{258,6 - \mathit{P} / 258,2 \cdot 227,1} + 1
\label{eq:calcptosg}
\end{equation}

\begin{equation}
\degree P\:[g/100g] = -205,347 \cdot \sg^2 + 668,72 \cdot \sg - 463,37
\label{eq:calcsgtop}
\end{equation}

Lincoln equations
\autocite{Spedding2016}

\section*{Messung}

Vor einer Messung sollte der Probenaufnahmebereich eines Refraktometers
und zum Transfer der Probe eingesetzte Pipetten mit destilliertem Wasser
gereinigt werden um eine Messwertverfälschung durch Rückstände vorheriger
Proben zu verhindern. Neben Kontaminierungen ist die Temperatureinwirkung
einer der häufigsten Quellen für Messfehler. Das eingesetzte Refraktometer
sollte daher auf Raumtemperatur erwärmt sein. Danach ist eine 
Einpunkt-Kalibrierung durchzuführen. \autocite{Depalma2017}

Proben bedürfen einer Aufbereitung. Darin enthaltene Schwebestoffe und
Gase sind zu entfernen. Das heißt, das
während der Gärung entnommen Proben längere Zeit stehen gelassen werden
müssen bis ein Großteil der darin enthaltenen Hefe sedimentiert
ist. Zur Entfernung von \ch{CO2} kann die Probe innerhalb eines
Behälters mehrmals geschüttelt werden. Inzwischen ist der Behälter
kurz zu öffnen, damit das Gas entweichen kann. Heiße Würze
sollte niemals direkt auf ein Refraktometer aufgetragen werden,
denn hierbei kommt es zur Verdampfung und somit zu einer
Konzentrationsänderung Proben sollten daher,
bis diese die Raumtemperatur angenommen haben, in einem abgedeckten
Behälter aufbewahrt werden. \autocite{Gamer1959,MEBAK2013,Terrill2013}

Für die eigentliche Messung sind mittels Pipette einige Tropfen
der Probe auf den Probenaufnahmebereich aufzutragen
\autoref{fig:refactomeasure}. Beim Handrefraktometer
ist danach die Abdeckung zu schließen und dann das Messgerät
zum Ablesen des Messwerts gegen eine Lichtquelle zu richten.
Der gemessene Wert wird durch die Trennlinie zwischen
farblichem und farblosem Bereich signalisiert (\autoref{fig:refractoscale}).
Ist die Trennlinie verschwommen befinden sich mitunter Schwebestoffe
in der Probe. Beim MA885 wird die Messung nach
dem Einschalten durch Drücken der Taste „Read“ ausgelöst. Der
Messwert ist danach vom Display abzulesen. Bei digitalen
Refraktometern ist zu beachten, dass durch externe Lichtquellen
verursachte Interferenzen die Messung beeinflussen können.
\autocite{Gossett2012a,Terrill2013}

\begin{figure}[h]
\centering
\includegraphics[width=4.8cm]{images/measure.jpg}
\caption{Durchführung einer Messung}
\label{fig:refactomeasure}
\end{figure}

\section*{Kalibrierung und Justierung}

Um Sicherzustellen, dass ein Messgerät weiterhin genaue Messergebnisse
liefert, ist in regelmäßigen Abständen eine Kalibrierung durchzuführen.
Das heißt den mit dem Messgerät erfassten Messwert mit einer bekannten
Referenz (rückführbarer Kalibrierstandard), zu vergleichen und die
Abweichung festzustellen. Das sind bei Refraktometern zum Beispiel
wässrige Zuckerlösungen mit einer definiertem Zuckergehalt.
Wurde eine unerwünschte Abweichung festgestellt, kann diese gegebenenfalls
durch Justierung des betroffenen Messgeräts ausgeglichen werden.

Eine Kalibrierung ist je nach Anzahl der gemessenen Referenzen
eine Einpunkt-, Zweipunkt- oder Mehrpunkt-Kalibrierung.
Bei der Einpunkt-Kalibrierung wird im wesentlichen ein definierter
Nullpunkt gesetzt. Die Erfassung von mehreren Messpunkten erlaubt
die Korrektur von linearen oder nicht linearen Abweichungen
über den Messbereich. Ab zwei Messpunkten kann eine Geradengleichung
aufgestellt und ab drei Messpunkten eine Polynomfunktion eingepasst
werden. \autocite{Earl2015}

Vor jeder Messserie empfiehlt sich die Durchführung einer
Einpunkt-Kalibrierung bei Raumtemperatur mit destilliertem
Wasser. Das Refraktometer solle dabei die Raumtemperatur
angenommen haben und gereinigt sein. Zur Kalibrierung ist
eine Messung mit destilliertes Wasser als Probe durchzuführen.
Der angezeigte Messwert sollte dabei 0~°Bx betragen
(\autoref{fig:refractoscale}).
Wird ein Abweichung festgestellt kann ein Handrefraktometer durch
drehen an der Einstellschraube justiert werden (\autoref{fig:refactoadjust}),
bis die Anzeige 0~°Bx beträgt. Beim MA885 erfolgt die
Justierung durch Drücken der Taste „Zero“.
\autocite{Bonham2001,Depalma2017,Terrill2013}

\begin{figure}[h]
\centering
\includegraphics[width=4.8cm]{images/adjust.jpg}
\caption{Justierung eines Handrefraktometers}
\label{fig:refactoadjust}
\end{figure}

Messgeräte sollten nach der Herstellung einer ausreichenden
Qualitätssicherung unterlaufen. Bei günstigen Handrefraktometern
muss dies nicht zwangsweise der Fall sein. Zur Feststellung
ob die integrierte Messskala eine zufriedenstellende Abdeckung
des Messbereichs bietet kann eine Zweipunkt-Kalibrierung durchgeführt
werden. Da der Preis eines Kalibrierstandards im Normalfall
deutlich über den Anschaffungskosten des Messgeräts
liegen, kann stattdessen eine Referenzlösung mit Haushaltszucker,
destilliertem Wasser und einer Feinwaage hergestellt werden. Zuerst
ist beim betroffenen Refraktometer, wie zuvor beschrieben, der
Nullpunkt zu kalibrieren und justieren. Danach ist eine Referenzlösung
für den oberen Bereich der Messskala herzustellen. Eine Lösung 
für 32~°Bx entspricht dabei 32~g Zucker und 68~g Wasser.
Die festgestellte Abweichung lässt sich nicht justieren aber
durch Aufstellen einer Geradengleichung korrigieren.
\autocite{Earl2015,Terrill2013,Troester2012}

\section*{Würze-Korrekturfaktor und Stammwürze}

Auch wenn sich die Einheiten °Bx und °P beide auf
wässrige Saccharoselösungen beziehen, ein auf °Bx
kalibriertes Refraktometer sollte bei gleicher Stammwürze einen
höheren Messwert liefern als eine Bierspindel. Der Grund hierfür ist,
dass Würze noch weitere Bestandteile enthält, die sich auf den
Brechungsindex auswirken. Das sind unter anderem Säuren, Salze,
Proteine und Hopfenöle. Messwerte müssen dementsprechend durch einen
Korrekturfaktor (Wort Correction Factor, \wcftext) angepasst
werden. Zur Ermittlung der Stammwürze in °P ist daher 
der Messwert des Refraktometers der Stammwürze (\bxitext) gemäß
\autoref{eq:applywcf} durch den \wcftext\,zu dividieren.
\autocite{Bonham2001,BSHB2010,Roberts1950,Terrill2013}

\begin{equation}
\oex \:[\degree P] = \bxic = \frac{\bxi}{\wcf}
\label{eq:applywcf} 
\end{equation}

Der \wcftext\,ist stark von der Würzezusammensetzung abhängig und
sollte deshalb pro unterschiedlicher Schüttung mit Hilfe einer
Extraktmessung durch ein anderes Messgerät gemäß \autoref{eq:calcwcf}
bestimmt werden. Ein gängiger Wertebereich für diesen ist 1,02 bis 1,06.
Der etablierte Durchschnittswert entspricht 1,04. Dieser sollte jedoch
nicht als Konstante betrachtet werden.
\autocite{Bonham2001,Roberts1950,Terrill2013}

\begin{equation}
\wcf = \frac{\bxi}{\oex} \approx 1,04
\label{eq:calcwcf} 
\end{equation}

\section*{Korrelationsmodelle und scheinbarer Restextrakt}

Der während der Gärung gebildete Ethanol verfälscht nicht nur
die Extraktmessung mit der Bierspindel sondern beeinflusst auch
den Brechungsindex. Es besteht daher die Notwendigkeit einen
Refraktometermesswert für die Berechnung des Alkoholgehalts in eine
andere Messgröße zu transformieren. Die entsprechende Messgröße ist
im Normalfall der scheinbare Restextrakt, so wie er mit einer
Bierspindel gemessen würde. Die Umrechnung erfolgt durch eine
Korrelationsfunktion. Insgesamt wurden mindestens fünf
Korrelationsmodelle im Heimbrauumfeld für diesen Zweck veröffentlicht,
die ihre jeweiligen Stärken und Schwächen besitzen. Ein grundlegendes
Problem bei der Entwicklung eines solchen Modells ist,
dass verschiedene Zusammensetzungen einer vergorene
Probe den gleichen Brechungsindex aufweisen können. \autocite{Terrill2010a,Terrill2010}

Zur Berechnung der folgenden Korrellationsfunktionen ist eine
Probe der Stammwürze (\bxitext) und eine während oder nach der
Gärung entnommene Probe (\bxftext) mit dem Refraktometer zu
messen. Je nach Funktion sind unveränderte (\bxitext, \bxftext)
oder um den Würze-Korrekturfaktor veränderte (\bxictext, \bxfctext) 
Messwerte einzusetzen.

\subsection*{Gardner}

Das lineare Gardner Modell zu Berechnung des scheinbaren Restextrakts
(\autoref{eq:gardner}) wurde im Jahr 2000 in der Zeitschrift
„The New Brewer“ veröffentlicht und von existierenden Gleichungen
abgeleitet. Bonham beschreibt dieses Modell als ausreichend
für die private Anwendung, weißt aber darauf hin, dass bereits
genauere Näherungsverfahren zu diesem Zeitpunkt existiert
hätten. Neben der Gleichung zur Bestimmung des scheinbaren
Restextrakts hat Gardner auch eine Gleichung zur Bestimmung
des Alkoholgehalts in Gewichtsprozent (Alcohol by Weight, \abwtext, \autoref{eq:gardnerabw}).
Im Heimbraubereich scheint dieses Modell keine große Verbreitung
erfahren zu haben. \autocite{Bonham2001}

\begin{equation}
\mathit{AE}\:[g/100g]=1,53 \cdot \bxf - 0,59 \cdot \bxic
\label{eq:gardner} 
\end{equation}

\begin{align}
\begin{split}
\abw\:[\%\:w/w] &= 1,09 \cdot \bxf - 1,13 \cdot \aex
\end{split} \label{eq:gardnerabw} 
\end{align}

\subsection*{Bonham}

Das von Bonham im Jahr 2001 im Zymurgy Magazin der American Homebrewers
Association veröffentlichte Korrelationsmodell ist auch als
Standardformel bekannt. Als Grundlage dienten bereits existierende
Gleichungen, die an die üblichen Maßeinheiten der amerikanischen
Heimbrauerszene angepasst wurden. Bis zur Ablösung durch die Terrill
Formel im Jahr 2011 war dieses Modell weit verbreitet. Heute
wird es noch vereinzelt wie zum Beispiel von 
„\href{https://www.maischemalzundmehr.de/index.php?inhaltmitte=toolsrefraktorechner}{Maische, Malz und Mehr}“ für Messungen während der Gärung empfohlen. \autocite{Bonham2001,Terrill2010a}

Das Bonham Modell besteht aus einer Gleichung zur Bestimmung der
spezifischen Dichte eines vergorenen Bieres (\autoref{eq:bonham})
und einer Gleichung zur Bestimmung des Alkoholgehalts in Volumenprozent
(Alcohol by Volume, \abvtext, \autoref{eq:bonhamabv}).

\begin{align}
\begin{split}
\fg\:[g/ml] &= 1,001843 - 0,002318474 \cdot \bxic - 0,000007775 \cdot \bxic^2 -
0,000000034 \cdot \bxic^3 \\
& \quad + 0,00574 \cdot \bxf +
0,00003344 \cdot \bxf^2 + 0,000000086 \cdot \bxf^3
\end{split} \label{eq:bonham} 
\end{align}

\begin{align}
\begin{split}
\abv\:[\%\:v/v] &= (277,8851 - 277,4 \cdot \fg + 0,9956 \cdot \bxf + 0,00523 \cdot \bxf^2 + 0,000015 \cdot \bxf^3) \\
& \quad \cdot \fg / 0,79
\end{split} \label{eq:bonhamabv} 
\end{align}

\subsection*{Terrill}

Bei Vergleichsmessungen zwischen Bierspindel und einem Handrefraktometer
in Verbindung mit der Bonham Gleichung hat Terrill im Jahr 2010 eine
mittlere Abweichungen der Messwerte von 1,3 Grad Plato festgestellt und darauf hin mit der Entwicklung eines eigenen Korrelationsmodell begonnen welches
in linearer (\autoref{eq:terrilllinear}) und vereinfachter kubischer Form
(\autoref{eq:terrillcubic}) veröffentlicht wurde. \autocite{Terrill2010a}

\begin{equation}
\fg\:[g/ml] = 1 - 0,00085683 \cdot \bxic + 0,0034941 \cdot \bxfc
\label{eq:terrilllinear} 
\end{equation}

\begin{align}
\begin{split}
\fg\:[g/ml] &= 1 - 0,0044993 \cdot \bxic + 0,011774 \cdot \bxfc + 0,00027581 \cdot \bxic^2 - 0,0012717 \cdot \bxfc^2 \\
& \quad  - 0,0000072800 \cdot \bxic^3  + 0,000063293 \cdot \bxfc^3
\end{split} \label{eq:terrillcubic} 
\end{align}

Insgesamt dienten Vergleichsmessungen von 12 Bieren mit Stammwürzen
von 9 bis 24 Grad Plato, scheinbaren Restextrakten von 1,8 bis
5,6 Grad Plato und scheinbaren Vergärungsgraden von 73 bis 91~\%
als ursprüngliche Datenbasis. Auch kommerzielle Brauereien haben weitere Messwerte erfasst. Mittlerweile verfügt Terrill  über mindestens
60 Vergleichswerte. Messungen mit geringeren Vergärungsgraden wurden
ausgeschlossen, da diese die Genauigkeit des Modells verschlechtert
hätten.
\autocite{Terrill2010a,Terrill2010,Terrill2011,Terrill2013}

\autoref{table:terrillcomp} zeigt den von Terrill veröffentlichte
Ergebnis zwischen den Modell von Bonham und seinem Eigenen. Bei
endvergorenen Proben liefert das Bonham Modell eine wesentlich
schlechtere Näherung. Hierbei ist allerdings zu beachten, dass
Terrill die ursprüngliche Quelle des Bonhams Modells nicht kannte
und deshalb fälschlicherweise auch den zweiten Messwert um
den Würze-Korrekturfaktor angepasst hat. Dieser Fehler soll sich
aber nicht signifikant auf das Ergebnis auswirken.
\autocite{Terrill2010a,Terrill2010,Terrill2011}

\begin{table}[h]
\centering
\begin{tabular}{lrrrrrr}
\toprule
Korrelation &  Max. Abw. [g/100g] & Mittlere Abw. & Standardabw. & < 0,25 [\%] & < 0,50 & < 1,00 \\
\midrule
Bonham & -2,1 & -0,4 & 0,6 & 18 & 41 & 90 \\
Terrill Kubisch & 1,2 & 0,2 & 0,3 & 56 & 85 & 97 \\
Terrill Linear & 1,0 & 0,03 & 0,4 & 63 & 84 & 99 \\
\bottomrule
\end{tabular}
\caption{Vergleich zwischen Bonham und Terrill Modell}
\label{table:terrillcomp}
\end{table}

\subsection*{Gossett}

Von den bisher beschriebenen Korrelationsmodelle wird ein indirekter
Messwert in einen anderen indirekten Messwert transformiert. Gossett
war unzufrieden mit diesem Vorgehen und hat im Jahr 2012 daher ein
stöchiometrisches Modell (\autoref{eq:gossett}, \autoref{eq:gossettabw})
entwickelt, welches direkt den Alkohol in Gewichtsprozent von
berechnet. Er hat hierzu die Auswirkungen der Ethanolkonzentration
auf die Messwerten eines Refraktometers analysiert. Ein Gewichtsprozent
verfälscht diesen bei 20~°C um 0,445~°Bx. 
\autocite{Gossett2012,Gossett2012a,Gossett2012b}

\begin{equation}
C = 100 \cdot \frac{\bxi - \bxf}{100 - 48,4 \cdot 0,445 - 0,582 \cdot \bxf}
\label{eq:gossett} 
\end{equation}

\begin{equation}
\abw\:[\%\:w/w] = \frac{48,4 \cdot C}{100 - 0,582 \cdot C}
\label{eq:gossettabw} 
\end{equation}

Mit dem Gossett Modell lässt sich direkt keine spezifische Dichte
zur Berechnung von \autoref{eq:calcabv} bestimmen. Gossett wendet
hierzu \autoref{eq:bonham} an. Durch Umformung von
\autoref{eq:calcre} und \autoref{eq:calcabw} und Transformation gemäß
\autoref{eq:calcsgtop} ist aber eine Berechnung aus dem
durch \autoref{eq:gossettabw} erhaltenen Alkoholgehalt möglich.

\begin{equation}
\aex\:[g/100g] = \bxic - \frac{\abw \cdot (2,0665 - 1,0665 \cdot \bxic / 100)}{0,8052}
\end{equation}

\subsection*{Novotný}

Um das Jahr 2017 hat sich Novotný mit bestehenden Berechnungsvorschriften
im Heimbraubereich beschäftigt und nach Verbesserungsmöglichkeiten
gesucht. Darunter auch die Terrill Formel, da diese bekannt dafür
ist während der Anfangszeit einer Gärung keine guten Korrelationswerte
zu liefern. Basieren auf einer durch die Budweiser Brauerei beauftragte Forschungsarbeit zur Bewertung von Refraktometern als Messmittel
hat Novotný eine lineare (\autoref{eq:novotnylinear}) und eine quadratische
(\autoref{eq:applywcf}) Gleichung zur Berechnung der spezifischen
Dichte veröffentlicht. Darüber hinaus auch noch Formeln zur
Berechnung des wirklichen Restextrakts (Real Extract,
\rextext, \autoref{eq:novotnyre}) und des Alkoholgehalts
in Gewichtsprozent (\autoref{eq:applywcf}) auf Basis von Refraktometerwerten
und möglichst genauen Näherungsverfahren. \autocite{Novotny2017a,Novotny2017,Savel2009}

\begin{equation} 
\fg\:[g/ml] = -0,002349 \cdot \bxic + 0,006276 \cdot \bxfc + 1
\label{eq:novotnylinear} 
\end{equation}

\begin{align}
\begin{split}
\fg\:[g/ml] &= 1,335 \cdot 10^{-5} \cdot \bxic^2 - 3,239 \cdot 10^{-5} \cdot \bxic \cdot \bxfc + 2,916 \cdot 10^{-5} \cdot \bxfc^2 \\
& \quad - 2,421 \cdot 10^{-3} \cdot \bxic + 6,219 \cdot 10^{-3} \cdot \bxfc + 1
\end{split} \label{eq:novotnyquadratic} 
\end{align}

\begin{equation}
\rex\:[g/100g] = -0,29388 \cdot \bxic + 1.27582 \cdot \bxfc
\label{eq:novotnyre}
\end{equation}

\begin{equation}
\abw\:[\%\:w/w] = 0,67062 \cdot \bxic - 0,66091 \cdot \bxfc
\label{eq:novotnyabw}
\end{equation}

Das Novotný Modell wurde während der Gärung an drei Testsuden mit jeweils
bei 11,6~°P, 17~°P und 20~°P Stammwürze und an den Budweiser
Ergebnissen verifiziert. Laut den Erhobenen Daten liefert das Terrill
Modell im Vergleich während der ersten Tage der Gärung hohe
Abweichungen und auch eine geringfügig höhere Endabweichung.
\autocite{Novotny2017a,Novotny2017}

Eine Implementierung der Novotný Formel findet sich in den Braurechnern
„\href{https://brewfather.app}{Brewfather}“ und
„\href{https://www.brewersfriend.com/refractometer-calculator}{Brewer's Friend}“.

\subsection*{Welches Korrelationsmodell?}

Von den präsentierten Korrelationsmodellen scheinen nur die Modelle von Bonham,
Terrill und Novotný eine größere Verbreitung im Heimbraubereich erfahren
zu haben, wobei das Modell von Terrill selbst heute nach wie vor
als Quasi-Standard gilt. Ohne eine Vergleich auf Basis von Messdaten lässt
sich jedoch keine klare Empfehlung für ein bestimmtes aussprechen. Die
Problematik hierbei besteht, dass kaum Datensätze mit Vergleichswerten
zwischen Refraktometern und Bierspindeln oder genaueren Messgeräten
veröffentlicht wurden. Einer dieser Datensätze stammt von einem vom hobbybrauer.de Forum initiierten Gemeinschaftsexperiments und ein
anderer sind die Messdaten einer aktiven Gärung von Novotnýs 17~°P Testsud.
\autocite{Novotny2017a,Wolf2015}

\autocite{Katman2019}

\autocite{h22lude2020}

\begin{figure}[h]
\centering
\includegraphics[width=14cm]{graph_fermentation.pdf}
\caption{Gärverlauf mit Stammwürze von 17 °P}
\label{fig:novotnygraph}
\end{figure}

\begin{table}[h]
\centering
\begin{tabular}{lrrrr}
\toprule
    Korrelation &  Endabw. [g/100g] &  Max. Abw. &  Mittlere Abw. &  Standardabw. \\
\midrule
         Bonham &              -0,6 &       -0,8 &           -0,5 &           0,1 \\
        Gardner &              -0,5 &       -1,1 &           -0,7 &           0,2 \\
        Gossett &              -0,5 &       -0,6 &           -0,3 &           0,2 \\
 Novotný Linear &              -0,0 &       -0,8 &           -0,2 &           0,2 \\
  Novotný Quad. &              -0,1 &       -0,6 &           -0,3 &           0,2 \\
Terrill Kubisch &               0,2 &        9,0 &            1,3 &           3,2 \\
 Terrill Linear &               0,3 &       -5,9 &           -1,8 &           2,1 \\
Terrill+Novotný &              -0,0 &       -0,8 &           -0,2 &           0,2 \\
\bottomrule
\end{tabular}

\caption{Abweichung des scheinbaren Restextrakts bei Gärverlauf mit Stammwürze von 17 °P}
\label{table:novotnytable}
\end{table}

\begin{figure}[h]
\centering
\includegraphics[width=14cm]{graph_ae.pdf}
\caption{Histogramm der Abweichung des scheinbaren Restextrakts bei endvergorener Probe}
\label{fig:hbfgraph}
\end{figure}

\begin{table}[h]
\centering
\begin{tabular}{lrrrrrr}
\toprule
    Korrelation &  Max. Abw. [g/100g] &  Mittlere Abw. &  Standardabw. &  < 0,25 [\%] &  < 0,5 &  < 1,0 \\
\midrule
         Bonham &                -1,8 &           -0,4 &           0,4 &        39,5 &   62,8 &   90,7 \\
        Gardner &                -1,8 &           -0,4 &           0,4 &        30,2 &   65,1 &   93,0 \\
        Gossett &                -1,7 &           -0,4 &           0,4 &        39,5 &   69,8 &   88,4 \\
 Novotný Linear &                -1,4 &           -0,2 &           0,4 &        55,8 &   81,4 &   97,7 \\
  Novotný Quad. &                -1,5 &           -0,3 &           0,4 &        34,9 &   76,7 &   93,0 \\
Terrill Kubisch &                 1,7 &            0,2 &           0,5 &        46,5 &   74,4 &   95,3 \\
 Terrill Linear &                 1,3 &            0,0 &           0,5 &        46,5 &   69,8 &   95,3 \\
Terrill+Novotný &                 1,3 &            0,1 &           0,4 &        65,1 &   83,7 &   97,7 \\
\bottomrule
\end{tabular}

\caption{Abweichung des scheinbaren Restextrakts bei endvergorener Probe}
\label{table:hbftable}
\end{table}

\section*{Berechnung des Alkoholgehalts und weiterer Kennzahlen}

Aus konsumierter Zuckermenge und der Balling Formel lässt sich

the following values must be known or
obtained: the present or apparent gravity of the beer, the real extract of the beer,
the specific gravity of the alcohol contained within the beer, and the original extract
of the wort (terms defined in Table 7.1). To help solve for these various terms, it is
noted that an arithmetic relationship exists between them through which Carl Ball-
ing first laid down the foundation for brewing calculations


\begin{equation}
\rex\:[g/100g] = 0,1948 \cdot \oex + 0,8052 \cdot \aex
\label{eq:calcre} 
\end{equation}

\begin{equation}
\abw\:[\%\:w/w] = \frac{\oex - \rex}{2,0665 - 1.0665 \cdot \oex / 100}
\label{eq:calcabw}
\end{equation}

\begin{equation}
\abv\:[\%\:v/v] = \frac{\abw \cdot \fg}{0,7907}
\label{eq:calcabv}
\end{equation}

\autocite{Spedding2016}

\begin{equation}
\adf\:[\%]= 100 \cdot \frac{\oex - \aex}{\oex}
\end{equation}

%\setlength{\jot}{2mm}

\begin{equation}
\rdf\:[\%] = 100 \cdot \frac{\oex - \rex}{\oex} \cdot \frac{1}{1 - 0,005161 \cdot \rex}
\end{equation}

\autocite{Speers2015}

Lincoln equations
\autocite{Spedding2016}

Method \href{https://www.asbcnet.org}{American Society of Brewing Chemists (ASBC)} \href{https://www.mebak.org}{Mitteleuropäische Brautechnische Analysenkommission (MEBAK)}

Novotný

bxi = 12
bxf = 6
wcf = 1.04

Referenzimplementierung \url{https://aschet.github.io/refractometer/}

\autoref{eq:applywcf}
\autoref{eq:novotnylinear}
\autoref{eq:calcsgtop}
\autoref{eq:calcre}
\autoref{eq:calcabw}
\autoref{eq:calcabv}


\begin{align*}
\bxi\:[\degree Bx] &= 12 \\
\bxf\:[\degree Bx] &= 6 \\
\wcf &= 1,04 \\
\bxic\:[\degree P] = \frac{12}{1,04} &= 11,54 \\
\bxfc\:[g/100g] = \frac{6}{1,04} &= 5,77 \\
\fg\:[g/ml] = -0,002349 \cdot 11,54 + 0,006276 \cdot 5,77 + 1 &= 1,009 \\
\aex\:[g/100g] = -205,347 \cdot 1,009^2 + 668,72 \cdot 1,009 - 463,37 &= 2,34 \\
\rex\:[g/100g] = 0,1948 \cdot 11,54 + 0,8052 \cdot 2,34 &= 4,13 \\
\abw\:[\%\:w/w] = \frac{11,54 - 4,13}{2,0665 - 1,0665 \cdot 11,54 / 100} &= 3,81 \\
\abv\:[\%\:v/v] = \frac{3,81 \cdot 1,009}{0,7907} &= 4,87
\end{align*}

\autocite{MEBAK2013}

\section*{Zusammenfassung}

\printbibliography[title=Quellen]

\end{document}