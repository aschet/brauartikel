% Copyright 2022 Thomas Ascher
% SPDX-License-Identifier: CC-BY-SA-4.0

\documentclass[a4paper,parskip=half]{scrartcl}

\usepackage[T1]{fontenc}
\usepackage[naustrian]{babel}
\usepackage{csquotes}
\usepackage[regular,condensed,sfdefault]{roboto}
\usepackage{booktabs}
\usepackage{graphicx}
\usepackage{chemformula}
%\usepackage{euler}
\usepackage{amsmath,amsfonts,amssymb}
\usepackage{icomma}
%\usepackage{textcomp}
\usepackage{gensymb}
%\usepackage{mathastext}
\usepackage{float}
\usepackage[style=apa,backend=biber]{biblatex}
\DeclareLanguageMapping{naustrian}{naustrian-apa}

\usepackage[hidelinks,pdfencoding=auto,
  pdfauthor={Thomas Ascher},
  pdfusetitle,
  pdfkeywords={Bier,Bitterung,Bitterausbeute,IBU,Tinseth,Rager,Garetz}]{hyperref}
\usepackage{microtype}
\DisableLigatures{encoding=*, family=*}

\addto\extrasnaustrian{
\def\figureautorefname{Abb.}
\def\tableautorefname{Tab.}
\def\equationautorefname{Gl.}
}

\addto\captionsnaustrian{
\renewcommand{\figurename}{Abb.}
\renewcommand{\tablename}{Tab.}
}

\NewBibliographyString{gethesis}
\DefineBibliographyStrings{naustrian}{
  mathesis = {Masterarbeit},
  gethesis = {Diplomarbeit},
}

\newcommand{\BA}{\mathit{BA}}
\newcommand{\BAKt}{{\mathit{BA}}_{\mathit{Kt}}}
\newcommand{\umin}{\:[\textrm{min}]}
\newcommand{\uden}{\:[\textrm{g/cm³}]}
\newcommand{\uper}{\:[\textrm{\%}]}
\newcommand{\uli}{\:[\textrm{l}]}
\newcommand{\ume}{\:[\textrm{m}]}
\newcommand{\FKd}{F_{\mathit{Kd}}}
\newcommand{\FHR}{F_{\mathit{HR}}}
\newcommand{\FSP}{F_{\mathit{SP}}}
\newcommand{\FAH}{F_{\mathit{AH}}}
\newcommand{\FHF}{F_{\mathit{HF}}}
\newcommand{\FHS}{F_{\mathit{HS}}}
\newcommand{\FFil}{F_{\mathit{FI}}}
\newcommand{\dPfvw}{d_\mathit{Pfvw}}
\newcommand{\dKt}{\overline{d_{\mathit{Kt}}}}

\title{Eine bittere Angelegenheit: IBU Berechnungen}
\author{Thomas Ascher <thomas.ascher@gmx.at>}
\date{\today, \href{http://creativecommons.org/licenses/by-sa/4.0/}{CC BY-SA 4.0}}

\addbibresource{hopfenbittere.bib}

\begin{document}
\maketitle

\section*{Einleitung}

Hopfen, der auch als Seele des 

\parencite{Bastgen2020}
%TODO

Meiste Bittere vom Hopfen, ein Teil von Röstmalzen \parencite[11]{Garetz1994}

\parencite[10]{Garetz1994}
Aroma, Bittere und Haltbarkeit

\parencite[103]{Garetz1994}
Brauer interesserit: Alpha, Beta und Ölgehalt.


\section*{Bitterstoffe im Hopfen}


\parencite[34]{Garetz1994}
Alphasäure Bestandteil am Gewicht des ganzen Hopfens. Herangezogen
zur Bitterberechnung. Betasäure wird ignoriert.


\parencite{MEBAK2020}
Die wichtigsten Bitterstoffe in Würze und Bier sind die iso-alpha-Säuren. Zudem lassen sich, vor allem in Würze, noch beta-Säuren sowie -Säuren nachweisen. Darüber hinaus enthalten Würze und Bier andere Derivate der Hopfenbittersäuren, insbesondere Oxidationsprodukte, die ebenfalls zum Bittergeschmack beitragen.
Die Bitterstoffe, hauptsächlich iso-alpha-Säuren, werden mit iso-Oktan aus der
angesäuerten Probe extrahiert und ihre Konzentration im Auszug spektralphotometrisch bestimmt.


\parencite[11]{Garetz1994}
Alphasäuren: humulone, cohumulone, adhumulone, sind Bittere und kommen in
verschiednen Hopfensorten in unterschiedlichen Verhältnissen vor
Glauben: Mit hohem Cohumulone Gehalt unangenehmere Bittere.
Beta Säuren, nicht Bitter? Ignoriert hinsichtlich Bitterpotential
Alphasäuren schlecht löslich in Würze bei geringeren Temperaturen
und normalem pH-Wert.

\parencite[20]{Garetz1994}
Weibliche Pflanze der Hopfengattung Humulus lupulus, Alpha Beta
Säuren teil des resins, Resin mit Hopfenölen in gelben
Lupulindrüsen in Dolden produziert.

\parencite[100]{Garetz1994} Einige Hopfenaromen entstehen erst
durch Oxidation. Besonders bei Nobelhopfen interessant.

\parencite[119]{Garetz1994} 
Adhumulone nur in kleinen Mengen.

\parencite[120]{Garetz1994} 
nur geringe Bittere, noch schlechter Wasserlöslich, mehrere Säuren,
lupulone, colupulone und adlupulone
Während lagerung und und kochen oxiieren und bilden Bitterstoffe

\parencite[60]{Beechum2017}
Genaue Messung von iso-alpha zeitintensiv und teuer
in 1950 verinfachte methode, die von der ASBC in 1968
als Standard etabliert wurde.
spectrophotometer, Lichtabsorbtion bei Wällenlänge
von 275 Nanometer wenn durch präperierte Probe
Korreliert miit der Konzentration von iso-alpha.
Beeinflusst durch Dry Hops und andere Substanzen

Iso-Alpha acids aren’t the only things that cause an organoleptic
sensation of bitterness, the IBU is an incomplete picture of just how bitter a
beer really is.

\parencite[61]{Beechum2017}
Relative dichte der Kochwürze wird als Konstante betrachtet
Nur für gekochte Hopfen
Nur für Dolen, es gibt verschiedene Korrekturfaktoren

IGOR volunteers Basic Pale Ale, Basic IPA, or
Basic DIPA 
Versuch mit 22 Suden, verschiedene Rezpete
Oregon Brew Lab, sensorisch getestet.

\parencite[62]{Beechum2017}
10 mL of beer with a little bit of Salzsäure and iso-Octan. The resulting
solution is agitated vigorously until it sep-
arates into two to three phases. The 275
nm absorption of the distinct clear phase
at the top of the sample is compared to
pure iso-octane. That comparison gives
you the official IBU number.

APAs. Both the median and mode of the
sample set were spot on with the formula’s
calculated estimate. As the beers increase
in gravity, the wobbliness we expected
begins to emerge. The IPA is still close,
with the average about 5 IBUs below the
predicted value. The DIPA, though, is a
full 21 IBUs (about 28 percent) below the
calculated value.

\parencite[65]{Beechum2017}
we think the “undershoot” is due to the
rapid chilling procedures that are more
common today than they were during the
formulae’s development.
Well,
one is that there’s probably more work
to be done to figure out how the
numbers change with different chill-
ing regimes, kettle geometry, and boil
vigor.

Well,
one is that there’s probably more work
to be done to figure out how the
numbers change with different chill-
ing regimes, kettle geometry, and boil
vigor.


\parencite[55]{Hall1997}
Hopfenbittere aus den Bitterharzen in den gelben Lupulindrüsen.
alpha, beta, gamma. alpha+beta sind weichharze weil
lösbar in Hexan. Gamma nicht darinlösbar darum
hartharze. Alpha-fraktion humulone, cohumulone
adhumulone, prehumulone and posthumu-but repeating such a success can be difficult.
lone
The alpha acids will dissolve
in hot wort, up to 250 mg/L at a pH of 5 and
a temperature of 212 degrees F (100 degrees
C). They are not very soluble in beer, with
its lower pH and temperature, and will pre-
cipitate out if their concentration is higher
than 5 mg/L at a pH of 4 and temperature of
32 degrees F (0 degrees C)
uring the kettle boil, the alpha acids
undergo a molecular rearrangement called
isomerization. The resultant chemicals are
called iso-alpha acids, and there is a corre-
sponding version for each humulone (iso-
humulone, isocohumulone, etc.). The iso-
alpha acids are much more soluble in wort
and beer,
The beta-fraction of the hop resins is
composed of the beta acids and many other
chemicals, including the oxidation products
of the alpha and beta acids that result from
aging (De Clerck, 1957). The beta acids are
known as lupulones and occur in varieties
similar to the humulones.
lupulone,
colupulone, adlupulone, prelupulone
postlupulone
postlupulone
less soluble than the alpha acids (0.7 mg/L
they do contribute some bitterness to beer
through their oxidation products. The bit-
besonders in gealterten Hopfen, wird
als unangenehm empfunden.
The hard resins do not contribute to the
bitterness of the finished beer.

\parencite[35]{Garetz1994}
Nobelhopfen Alpha:Beta = 1:1, nieder cohumulone.
mehr cohumulone harsch, gibt aber hopfen mit hohem cohumulone
Anteil, die nicht als Harsch bekannt sind. Ineratkion
mit anderem.


\section*{Die International Bitterness Unit (IBU)}

\parencite[145-151]{Garetz1994}
Taste Titration 
Bier mit bekannter Bittere,, iso-alpha extrakt hinzufügen
bis gleich bitter wie test bier kalibrierter Extrakt, 0,44 IBU

\parencite[121]{Garetz1994} 
Messung: Spektrometer, mit ISO-Octane lösungsmittel, mixture
wird gemessen bei UV licht, genauer gesagt die abosption
bei gewissen Wellenlängen, absorption proportional
zu IBU (die wird korreliert)

\parencite[120\psq]{Garetz1994}
internationaler Standrard IBU oder kurz BU, amerikaner und
europäer eigene Messmethoden




\section*{Isomerisierung und Bitterausbeute}


 
\parencite[121]{Garetz1994} 
IBU mg iso alpha in einem liter Bier, fertiges bier
ein Teil geht verloren. 


\parencite{Tinseth1997}
IBUs = decimal alpha acid utilization * mg/l of added alpha acids

mg/l of added alpha acids = decimal AA rating * grams hops * 1000
                            -------------------------------------
                              volume of finished beer in liters
                              

\parencite[50]{Holle2010}
Utilisation rate wie viel alphasäuren vom Hopfen in Lösung gehen
und im Bier laden. 10 bis20 verlust

Höhere utilisation bei
vigor boild und temperature (umgebungsdruck, seehöhe)
geringerer relative Dichte
geringere Hopfenmenge
höherer pH Wert, -> zum Teil unangenehmere Bittere

\parencite[159]{Annemueller2015}

Art (zerkleinerungsgrad9 des Hopfenprodukts. Pelletts 35, Presshopfen
30. bei 100°C 90 min.

Kochdauer (je länger des höher), Konzentration der alphasäuren (je höher desto geringer)
Kochtemperatur (höher schnellere Isomerisierung)
Ph-Wert der Würze (je höher, desto höher).
Bruch-/Trumbenge: je höher desto geringere Ausbeute 

\parencite[160]{Annemueller2015}
Verlust von Anstellwürze
Offene Warmgärung 40..50
Druckgärung 10...15

\parencite[162\psq]{Annemueller2015}
Bezieht sie auf Kalte Anstellwürze oder Fertigbier. Leitet sich aus der
Formel für die Bitterstoffausbeute ab. 

AW Anstellwürze
Pfvw = Pfannenvollwürze

BA=IBU AW * 100 / IBU Pfvw

BA = IBU Bier * 100/IBU Pfvw

\parencite[160-164]{Annemueller2015}
Bitterstoffgabe, Bitterstoffbilanz
Betriebliche Bitterstoffausbeute

\parencite[124]{Garetz1994} 

\% u = alpha present / alpha used * 100
efficency of alpha acid as iso alpha acid in final beer

Einflüsse:
Hopfenmenge
Hopenform
Kohtemperatur und Zeit
Volumen
Stammwürze
pH der Würze
Hopfengabe während Gärung
Yest Groth und floullation
Filtrierung

\parencite[125]{Garetz1994} 
Während des Kochens werden säuren und öle extrahiert 
bei dolden innerhalb von 10 bis 15 Minuten, bei Pallets
schneller.
Nach kochen fallen die unisomerisierten alpha säuren
aus, sonst spätestens bei Gärung

Isomerisierung abhängig von Volumen und Temperatur beim
Kochen, nach einem Zeitpunkt dreht sich die Reaktion
um. Nach circa 2 Stunden degraded. 

\parencite[126]{Garetz1994} 
Max. Isomerisierungsrate von 60-75, 8-10 trubverlust

iso absobiert onto Hefezellenwände, 5\% Schwankung auf
Utilisation basierend auf langsame und normale Flukulation
Hefewachstum

Auch mit CO2 ausgetrieben. In Hefedecke, macht bis zu 18\%
aus. Sammelt sich am Rand des Fermenters, verhärten durch
Oxidierung

\parencite[133]{Garetz1994} 
In Kräusen. J , höhere Stammwürze höherer Verlust, 

\parencite[128]{Garetz1994} 
Einige Korrekturfakten aus veröffentlichen Forschungsdaten. 
Messungen einiger gebrauten Biere.
Zu wenig Messdaten erhobens. Feedback erbeten.

10 Minuten zur Extraktion, 5-5.5 pH, vigorous boil

\parencite[130]{Garetz1994} 
Kochzeiten unter 30 min keine Pellet Korrektur
Bei pellets platzen die lupulindrüsen. 

Hopfensack ebenfalls einfluss,geringere utisilierung

\parencite[153]{Garetz1994} 
Beim Kochen mehrere Denkschulen
- bei Kochbeginn
- erste Anzeichen von Würzebruck
-> Bindet hopfenharze, reduziert Ausbeute; primär werden
Tanine gebunden
Vermindert Oberflächenspannung, reduziert überkochen
bei der Zugabe kann es zum überkochen kommen

\parencite[158\psq]{Garetz1994} 
Hopfen sind primärer Quelle für Bitterstoffe, balance
maskiert durch Malzssüße, 
Röstmalze liefern Bitterstoffe
Wasserprofil hat Einwirkung.

BU=Bittering unit (IBU x 1.125)
\parencite[215]{Noonan1996}


\parencite[57]{Hall1997}
optimistically peaks at 35 percent
Storage deterioration: oxidation reduziert bittere durch alpha und erhöht beta
abhängig von Temperatur, Alter, Luft (Hopfenform)

chemische abläufe: oxidiert zu anderen produkten, 
anti-isohumulones doppelt so bitter bis 10\%
höherer pH höherer Rate
Rückreaktion schließendlich zur dekomposition

Physitsche Trennung: In the hot wort at the end of the boilalpha acids are safely ensconced in the fin-
chronological order through the cycle ofthe utilization rate is about 50 percent
heißt und kalttrub aus lösung 7\%
durch co2 entfernent in kräusen, auch behälterrand
Filterung, hefezellen +- 5\%

Staling reactions: oxidation reaktion im fertigen beer, Käse

\parencite[58]{Hall1997}
Schätzung alle Effekte für Bitterausbeute zu beachten schwierig
Starke Schwankungen bei Doldenhopfen
geringer bei Pellets weil geblendet
abhängig von Kochtemperatur abhängig von Seehöhe
undissolved iso-alpha acids and unsaturated wort -> boil vigor, gravity -> viscosity
Hopfenform, pellets > dolden
Wasserprofil bitterness auswirkung
andere effekte -> oxidationsprodukte, röstmalze -> tannine, 

kommerzielle brauereien können messen und batches blenden.

acbc (1992). zentrifuge + spectrophototmeter

Wahrnehmungsschwelle 4 IBU





\parencite[320]{Kunze2004}
BU mg of bitter substance / l Beer
Berechnung auf Volumen kalte Ausschlagwürze contracts (4\%)
bitterness yield, only part of substance
25 und 35\% für kommerzielle Brauereien
Für Hopfengabe müssen für Bitterness Yield, Versuchssude in der Brauerei
Circa ein Drittel der gegebenen Alphasäuren landet im 
100\% Bitterness=aacid * 100 / BY

\parencite[51]{Davidson1997} gleiche Atomzusammenstellung wie
eine andere Substanz aber in anderer Struktur, struktuerelle
Veränderung
- nach 4 Stunden Rekation nocn nicht abgeschlossen
- util = alpha in work / alpha zugegeben

\parencite[12]{Garetz1994}
Durch Kochen geht ein Teil der Alphasäuren in Lösung. Durch Kochen
chemischen Prozess der Isomerisierung. Iso-Alpha Säuren. Sehr löslich.
Bei Kochen geht AS schnell in Lösung, Isomerisierung dauert lange.
Ab 90 Min. Plateu erreicht.
Nur Ungefähr 20\% Alphasäure -> Utilisierung landen als Iso-Alpha im
fertigen Bier.
Verlust: zu kurze Isomerisierugszeit, fällt mit Kühltrub aus,
während Fermentierung und Filterung



\parencite[128]{Garetz1994} 
Max. isomerisierung nach 90 Minuten, max 70\%, vermutlich fallender
ph-Wert reduziert isomerisierungsreaktion.

Höhere Hopfengaben ebenfalls. 40mg/l alpha 

\parencite[129]{Garetz1994} 
Gravity über 1.050 einfluss, Kochtemperatur (abhängig von Seehöhe).


\parencite[78]{Daniels1996}
ph-wert beeinflusst die isomerisierung nicht stark, aber Bitterwahrnahme
hoher ph wert harsche bittere mehr als 50 ppm carbonates

\parencite[49]{Holle2010} Isomerisierung bei Kochtemperatur
aber auch im Whirlpool 90-95


\section*{Hopfenprodukte}

\parencite[80\psq]{Garetz1994}
Doldenhopfen: am wenigsten verarbeitet, manche Brauer glauben
liefert das beste Aroma. Verbraucen viel Lagerplatz.

\parencite[82\psqq]{Garetz1994}
Pellets: pulverisiert und gepresst, verlust von 4 bis 6 \% Alpha
und Ölen.
T90
T45: enriched, teile des Pflanzenmaterials wird mechanisch
entfernt, doppelter Alpha.
Stabilisierte und isomerisierte Plellets, im Normalfall nicht
erhältlich. Ein Teil der Alphasäure in Iso Alphasäure
isomerisiert. Nicht für Dryhopping und Aromagaben.
chemisch verarbeitet, Magnesium hinzugefügt.
Pellets zerfallen zu Pulver
Schwerer zu entfernen, wird im Whilpool entfernt \parencite[87]{Garetz1994}

\parencite[84\psq]{Garetz1994}
Plugs und Bricks: 
Plug gepresster Doldenhopfen, in England gerne für Cask Dry Hopping
mehrere Dolden in Form, braucht einige Minuten bis sie auseinander
brechen. Schwer kleine Poritionen zu entnehmen.

\parencite[88-93]{Garetz1994}
Extrakte, mit mehreren Methoden hergestellt. Chemisch per Lösungsmittel
oder Dampfdestillation. Um an Alphasäure und Öle zu kommen.
Um Brauprozess berechenbarer zu machen und zu verinfachen.
iso-alpha extrakt: am Ende vom Prozess nach kochen oder
nach Gärung -> verluste, nicht sensitiv zu sonnenlicht, chemisch verändert
Hopfenöle: Aroma, Tasting panel hat schlechter gefunden wenn
late hops, wird in England für Casc conditioning, bessere konsistenz
Stark konzentriert, weniger für Anwendung im Heimbereich für
kleine Sude

\parencite[52]{Davidson1997}
nicht pre-isomerisiert
Pre-isomerized Hop Products -> doppelter gehalt
pre-isomerisierte pellets enthalten öle
The reduced (rho) , hexa, and tetra extracts
all provide bitterness and protection against
the sunstruck reaction that produces a
skunky note in beer. Additionally, hexa and
tetra improve foam stability.
The
reduced, hexa and tetra extracts are the
reduced fo rms (by hydrogen addition to
the iso-alpha acids) and the sensory bit-
terness of these compounds differs. Tetra-
iso-extract has a prolonged senso ry bitter-
ness , a lingering bitterness compared to
the normal iso-al pha acids . He xa-iso-
extract is less lingering and prolonged and
is more similar to iso-alpha acids . The
reduced iso-extract also is similar to iso-
alpha acids, in that it tends to be a quick,
clean bitterness.

found isohumulone (with traces of
iso-adhumu lone) was preferred to isoco-
humulone.

\section*{Hopfengaben}

\parencite[15]{Garetz1994}
Spät: Aromaöle verdampschen schnell, verlust nach mehreren Minuten
oder gleich durch Kochen. Oder verändert, nicht mehr gleich
wie im Naturhopfen. 15 Minuten vor Ende. Wenig isomerisierung.

\section*{Effekte der Hopfenalterung}

Nördliche Hemisphere einmal Ernte pro Jahr. Mitte August bis
Anfang September.
\parencite[97]{Garetz1994}

\parencite[97]{Garetz1994} Hopfen verliert Alphasäure und Öle
durch Oxidation. Vakuimieren und gekühlte Aufbewahrung verlangsamen
den Prozess. Auch kein Licht. 

\parencite[103]{Garetz1994} Alphasäure oxidiert an luft, oxidationsprodukte
nicht mehr isomerisierbar. Nicht mehr so bitter. Betssäuren bilden
Bitterstoffe. bei Oxidation.

\parencite[104]{Garetz1994} 
Unter gleichen Lagerbedingungen verlieren Hopfen unterschiedlich an
Alpha wegen unterschiedliche Menge Oxidantien.
-HSI spektroskopisch analyse von alpha und beta. Test für Halbarkeit:
Verlust von Alpha und Betasäure über 6 Montate über Raumtemperatur 20c.
Korrelation zwischen beiden. Der ursprüngliche Alphagehalt und der
zkünftige
Ölverluste nicht ausreichend erforscht.

\parencite[104]{Garetz1994} 
HSI Nummer, nich von Händlern, hauptsächlich als interne Referenz
in den Laboren
Percent Alpha remaining or Lost after 6 Months at 20c
\% Alpha /nach 6 Monaten



\parencite[52]{Davidson1997}
Oxidation von beta säuren -> ähnliche hulupones wie isohomulone


% TODO ev. Berechnung beschreiben
% TODO Diagramm zu Alphaverlust

\section*{Modelle zur Schätzung der Bitterausbeute}

Spätenstens seit der Veröffentlichung des Berechnungsverfahrens nach
Rager im Zymurgy Magazin im Jahr 1990 stand einem breiteren
Publikum die Möglichkeit zur Verfügung, Berechnungen für Hopfengaben auf
Basis der IBU
durchzuführen \parencite[59]{Hall1997}. Im Verlauf der Neunzingerjahre
sind dann noch weitere Modelle zur Schätzung der Bitterausbeute entstanden.
Von diesen hat sich das Tinseth Modell inzwischen als De-facto-Standard im
Heimbraubereich etabliert \parencite[185]{Hieronymus2012}.

Die Umgestaltung bestehender und die Schaffung neuer Bierstile
durch die Craft Beer Szene in den letzten zwanzig Jahren hat auch zu Änderungen
der Abläufe bei Hopfengaben, in der Form der vermehrten Anwendung von Vorderwürze-
und Whirlpoolhopfung, geführt. Die Modelle der Neunzigerjahre sind aber
nicht dafür ausgelegt, große Hopfengaben gegen Ende des Kochprozesses korrekt
zu berücksichtigen. Aktuelle Ansätze versuchen diesen Fehler zu korrigieren.
\parencite[39]{Novotny2018}


\parencite[51]{Holle2010}
ibu = mg hop * util * alpha acid \% / l

\parencite[127]{Garetz1994} 
Keine in professioneller Brauliteratur, Messtechnisch erhoben
werden. Nach einigen Brauversuchen lässt sich also einstellen.
Blending mehrer Batches.

\parencite[76]{Daniels1996}
Große Brauereien können große Hopfenmengen mischen und
Biere um unterschiede auszugleichen.

\parencite[51]{Holle2010}
Einziger verlässliche Weg ist labortechnische Analysen.




\subsection*{Modelle der Neunzigerjahre}

Hall hat einen Großteil der relevanten, ab 1990 veröffentlichten, Modelle analysiert,
deren Daten normalisiert und in ein gemeinsames Berechnungsschema überführt. Dabei ist
pro Hopfengabe zuerst eine initiale Bitterausbeute basierend auf der Kochzeit
($\BAKt$) zu bestimmen und anschließend durch mehrere Korrekturfaktoren anzupassen.
Aus den eingebrachten Alphasäuremengen der einzelnen Gaben und den ermittelten
Bitterausbeuten lässt sich dann ein IBU Wert für ein Rezept berechnen. \parencite[59-65]{Hall1997}

% TODO verweis konzentration

Folgende Korrekturfaktoren sind modellabhängig vorgesehen:

\begin{itemize}
\item Relative Dichte der „Kochwürze“ ($\FKd$)
\item Hopfenrate ($\FHR$)
\item Siedepunkt des Wassers basierend auf der Seehöhe ($\FSP$)
\item Sedimentationsverhalten der Hefe ($\FAH$)
\item Hopfenform: Dolden oder Pellets ($\FHF$)
\item Verwendung eines Hopfensacks ($\FHS$)
\item Filtration ($\FFil$)
\end{itemize}

Die Umrechnung von Extraktgehalt in relative Dichte erfolgt auf Basis
der Plato Tabellen über die Goldiner Gleichungen (\autoref{eq:calcptosg})
oder vergleichbaren Näherungsverfahren \parencite[140\psq]{Spedding2016}.

\begin{equation}
d_{\frac{20}{20}} \uden = \frac{\degree P}{258,6 - \degree P / 258,2 \cdot 227,1} + 1
\label{eq:calcptosg}
\end{equation}

\subsubsection*{Rager (1990)}

Die Grundlage für Ragers Modell bildeten Heimbraubücher von Fred Eckhardt,
Dave Miller und Byron Burch. Er hat die darin enthaltenen Informationen
zu einem Berechnungsverfahren zusammengeführt \parencite[53]{Rager1990}. 
Garetz kritisierte an Ragers Methode, dass nur der Korrekturfaktor
$\FKd$ berücksichtigt wurde \parencite[134]{Garetz1994}.

Für die Modellberechnung ist zuerst die Ausbeute $\BAKt$ über die \autoref{table:ragerbakt}
oder die \autoref{eq:ragerbakt} auf Basis der Kochdauer der jeweiligen Hopfengabe
zu wählen \parencite{Steinmeyer2021}.
Danach ist der Korrekturfaktor $\FKd$ anhand der relativen Dichte der
Pfannenvollwürze ($\dPfvw$) gemäß \autoref{eq:ragerga} und \autoref{eq:ragerfkd}
zu bestimmen \parencite[53]{Rager1990}.
Abschließend erfolgt die Berechnung der Bitterausbeute über \autoref{eq:ragerba}.

\begin{table}[H]
\centering
\begin{tabular}{rr}
\toprule
\multicolumn{1}{c}{\textbf{Kochzeit [min]}} & \multicolumn{1}{c}{\textbf{Ausbeute [\%]}} \\
\midrule
0–5             & 5 \\
6–10            & 6 \\
11–15           & 8 \\
16–20           & 10,1 \\
21–25           & 12,1 \\
26–30           & 15,3 \\
31–35           & 18,8 \\
36–40           & 22,8 \\
41–45           & 26,9 \\
46–50           & 28,1 \\
51–60           & 30 \\
\bottomrule
\end{tabular}
\caption{Bitterausbeute für Dolden nach Rager \parencite[54]{Rager1990}}
\label{table:ragerbakt}
\end{table}

\begin{equation}
\BAKt \uper = 18,11 + \left(13,86 \cdot \tanh{\frac{\textrm{Kochzeit} \umin - 31,32}{18,27}}\right)
\label{eq:ragerbakt}
\end{equation}


\begin{equation}
\mathit{DA} = \begin{cases}
0 \quad \textrm{für} \quad d_{\mathit{Pfvw}} \le 1,050 \uden, \\
\frac{d_{\mathit{Pfvw}} - 0,05}{0,2} \quad \textrm{für} \quad d_{\mathit{Pfvw}} > 1,050 \uden.
\end{cases}
\label{eq:ragerga}
\end{equation}

\begin{equation}
\FKd = \frac{1}{1 + \mathit{DA}}
\label{eq:ragerfkd}
\end{equation}


\begin{equation}
\BA \uper = \BAKt \cdot \FKd
\label{eq:ragerba}
\end{equation}

\subsubsection*{Burch}
Brewing Quality Beers

\subsubsection*{Garetz (1994)}

Das grundlegende Berechnungsverfahren für das von Garetz in 1994
veröffentlichte Modell übernahm er von Rager. Anpassungen sind jedoch an
der Tabelle zur Bestimmung der Ausbeute anhand der Kochzeit und den
Korekturfaktoren erfolgt \parencite[134-144]{Garetz1994}.

Die Berechnung nach Garetz kann in mehreren Ausbaustufen erfolgen,
deren Umfang von den gewählten Korrekturfaktoren abhängt.
Der Minimalumfang ist dabei durch \autoref{eq:garetzba1} definiert
\parencite[137]{Garetz1994}.
Zunächst ist die Ausbeute $\BAKt$ anhand der Kochzeit der jeweiligen 
Hopfengabe über die \autoref{table:garetzbakt} oder die \autoref{eq:garetzbakt} zu bestimmen \parencite{Steinmeyer2021}. Danach erfolgt die
Berechnung der Faktoren $\FKd$, $\FHR$ und $\FSP$ gemäß
\autoref{eq:garetzkd}, \autoref{eq:garetzhr} und \autoref{eq:garetzsp}.
Der Faktor $\FHR$ basiert auf dem Zielwert der gesamten Bittereinheiten
eines Rezepts. Das Modell von Garetz geht nämlich im Standardfalls
davon aus, Hopfenmengen auf Basis gewünschter IBU Werte zu schätzen.
Hier ist es entweder möglich den Faktor zunächst zu ignorieren und
und iterativ den Gesamtwert zu schätzen oder den Faktor bei nur
einer Hopfengabe durch eine quadratische Gleichung zu eliminieren      \parencite[63]{Hall1997}.

\begin{table}[H]
\centering
\begin{tabular}{rr}
\toprule
\multicolumn{1}{c}{\textbf{Kochzeit [min]}} & \multicolumn{1}{c}{\textbf{Ausbeute [\%]}} \\
\midrule
0–10            & 0 \\
11–15           & 2 \\
16–20           & 5 \\
21–25           & 8 \\
26–30           & 11 \\
31–35           & 14 \\
36–40           & 16 \\
41–45           & 18 \\
46–50           & 19 \\
51–60           & 20 \\
61–70           & 21 \\
70–80           & 22 \\
81–90           & 23 \\
\bottomrule
\end{tabular}
\caption{Bitterausbeute für Dolden nach Garetz \parencite[138]{Garetz1994}}
\label{table:garetzbakt}
\end{table}

\begin{equation}
\BAKt \uper = 7,2994 + \left(15,0746 \cdot \tanh{\frac{\textrm{Kochzeit} \umin - 21,86}{24,71}}\right)
\label{eq:garetzbakt}
\end{equation}


\begin{equation}
\mathit{CF} = \begin{cases}
1 \quad \textrm{bei keiner Verdünnung der Kaltwürzemenge}, \\
\frac{\textrm{Kaltwürzemenge} \uli}{\textrm{Pfannevollmenge} \uli} \quad \textrm{bei Verdünnung der Kaltwürzemenge}.
\end{cases}
\label{eq:garetzcf}
\end{equation}

\begin{equation}
d \uden = \begin{cases}
\dPfvw \quad \textrm{für} \quad \mathit{CF} = 1, \\
\left( \left( \dPfvw - 1 \right) \cdot \mathit{CF} \right) + 1 \quad \textrm{für} \quad \mathit{CF} \ne 1.
\end{cases}
\label{eq:garetzbg}
\end{equation}

\begin{equation}
\mathit{DA} = \begin{cases}
0 \quad \textrm{für} \quad d \le 1,050 \uden, \\
\frac{d - 0,05}{0,2} \quad \textrm{für} \quad d > 1,050 \uden.
\end{cases}
\label{eq:garetzga}
\end{equation}

\begin{equation}
\FKd = \frac{1}{1 + DA}
\label{eq:garetzkd}
\end{equation}

\begin{equation}
\FHR = \frac{1}{\left( \frac{\textrm{IBU Rezept}}{260} \cdot \mathit{CF} \right) + 1}
\label{eq:garetzhr}
\end{equation}

\begin{equation}
\FSP = \frac{1}{\left(\frac{\textrm{Seehöhe} \ume \cdot 3,2808}{550} \cdot 0,02 \right) + 1}
\label{eq:garetzsp}
\end{equation}

\begin{equation}
\BA_1 \uper = \BAKt \cdot \FKd \cdot \FHR \cdot \FSP
\label{eq:garetzba1}
\end{equation}

Alle weiteren Faktoren erachtet Garetz als optional. Der Faktor
$\FAH$ (\autoref{eq:garetzah}) ist anhand des Sedimentationsverhaltens
der Hefe zu wählen. Je schneller eine Hefe sedimentiert, desto
weniger Isoalphasäure haftet an den Zellwenden an und desto
mehr Isoalphasäure bleibt in Lösung. Dementsprechend ist die
Bitterausbeute um 5~\% nach unten oder nach oben zu korrigieren. 
Für Weißbier veranschlagt Garetz eine 20~\% höhere Ausbeute,
da die Hefe beim Trinken ebenfalls konsumiert wird.
Bei der Verwendung von Pellets erfolgt eine erhöhung der
Ausbeute um 10~\% bei einer Kochzeit bis 30~Minuten (\autoref{eq:garetzhf}).
Für Hopfensäcke ist eine Reduktion der Ausbeute,
basierend auf Packdichte, um bis zu 20~\% durchzuführen.
(\autoref{eq:garetzhs}). Der Einsatz eines Filtersystems kann
die Ausbeute inetwa um 1,25 bis 2,5~\% reduzieren (\autoref{eq:garetzfil}). 
Garetz empfieht eine messtechnische Erhebung des Faktors $\FFil$. \parencite[140\psq]{Garetz1994}

\begin{equation}
\FAH = \begin{cases}
1,00 \quad \textrm{für normal sedimentierender Hefe}, \\
1,05 \quad \textrm{für schnell sedimentierender Hefe}, \\
0,95 \quad \textrm{für langsam sedimentierender Hefe}, \\
1,20 \quad \textrm{für Weißbierhefe}. \\
\end{cases}
\label{eq:garetzah}
\end{equation}


\begin{equation}
\FHF = \begin{cases}
1,0 \quad \textrm{für Dolden}, \\
1,1 \quad \textrm{für Pellets für} \quad 10 < \textrm{Kochzeit} \umin \le 30. \\
\end{cases}
\label{eq:garetzhf}
\end{equation}

\begin{equation}
\FHS = \begin{cases}
1,0 \quad \textrm{ohne Hopfensack}, \\
0,9 \quad \textrm{für lose gepackten Hopfensack}, \\
0,8 \quad \textrm{für dicht gepackten Hopfensack}. \\
\end{cases}
\label{eq:garetzhs}
\end{equation}

\begin{equation}
\FFil := \left[0,975, 0,9875 \right]
\label{eq:garetzfil}
\end{equation}

\begin{equation}
\BA_2 \uper = BA_1 \cdot \FAH \cdot \FHF \cdot \FHS \cdot \FFil
\label{eq:garetzba2}
\end{equation}

\subsubsection*{Mosher (1994)}

Bereits am Ende der Achzigerjahre hatte Mosher einen aus mehreren
Drehscheiben bestehenden Rechenschieber zur IBU Berechnung erstellt und später
auch vertrieben \parencite{Mosher2022}. Sein in 1994 veröffentlichtes
Buch „\citetitle{Mosher1994}“ enthält zwei Diagramme zur grafischen
Bestimmung der Bitterausbeute für Kochzeiten bis zu drei Stunden
\parencite[160\psq]{Mosher1994}. Holle und \citeauthor{Thesseling2019} verwenden
einen Teil der abgemessenen Diagrammdaten noch immer in aktuelleren
Publikationen für grobe Abschätzungen \parencites[51]{Holle2010}[9]{Thesseling2019}.

Die Bitterausbeute nach Mosher ist basierend auf der relativen Dichte und
der Kochzeit der jeweiligen Hopfengabe zu bestimmen. Für Dolden ist
dabei der Wert aus \autoref{table:mosherbakt} und für Pellets aus
\autoref{table:mosherbaktpellets} zu entnehmen. Ob sich die
relativen Dichte dabei wie bei Rager auf die Pfannenvollwürze bezieht, ist aus Moshers Publikation nicht ersichtlich. Holle geht bei seinen Beispielen
von der Stammwürze aus \parencite[53]{Holle2010}.

\begin{table}[H]
\centering
\begin{tabular}{rrrrrrrr} 
\toprule
\multicolumn{1}{c}{\textbf{Kochzeit [min]}} & \multicolumn{7}{c}{\textbf{Ausbeute [\%]}}  \\
Relative Dichte [g/cm³]:                                        & 1,030 & 1,040 & 1,050 & 1,060 & 1,070 & 1,080  & 1,090                   \\
Extraktgehalt [\%w/w]:                                            & 7,6 & 10 & 12,4 & 14,7 & 17,1 & 19,3  & 21,6                   \\                                            
\midrule

5                                            & 5     & 5     & 4     & 4     & 3     & 3      & 3                          \\
15                                           & 12    & 12    & 11    & 11    & 11    & 10     & 9                          \\
30                                           & 17    & 17    & 16    & 16    & 15    & 15     & 13                         \\
45                                           & 21    & 21    & 20    & 19    & 18    & 17     & 16                         \\
60                                           & 24    & 23    & 23    & 22    & 21    & 20     & 18                         \\
90                                           & 28    & 27    & 26    & 26    & 25    & 23     & 21                         \\
\bottomrule
\end{tabular}
\caption{Bitterausbeute für Dolden nach \citeauthor{Mosher1994} \parencite[51]{Holle2010}}
\label{table:mosherbakt}
\end{table}

\begin{table}[H]
\centering
\begin{tabular}{rrrrrrrr} 
\toprule
\multicolumn{1}{c}{\textbf{Kochzeit [min]}} & \multicolumn{7}{c}{\textbf{Ausbeute [\%]}}  \\
Relative Dichte [g/cm³]:                                        & 1,030 & 1,040 & 1,050 & 1,060 & 1,070 & 1,080  & 1,090                   \\
Extraktgehalt [\%w/w]:                                            & 7,6 & 10 & 12,4 & 14,7 & 17,1 & 19,3  & 21,6                   \\  
\midrule
5                                            & 6     & 6     & 5     & 5     & 4     & 4      & 3                          \\
15                                           & 15    & 15    & 14    & 14    & 13    & 13     & 11                         \\
30                                           & 22    & 21    & 21    & 20    & 19    & 18     & 16                         \\
45                                           & 26    & 26    & 25    & 24    & 23    & 22     & 21                         \\
60                                           & 29    & 28    & 28    & 27    & 26    & 25     & 23                         \\
90                                           & 35    & 34    & 33    & 32    & 31    & 29     & 27                         \\
\bottomrule
\end{tabular}
\caption{Bitterausbeute für Pellets nach \citeauthor{Mosher1994} \parencite[51]{Holle2010}}
\label{table:mosherbaktpellets}
\end{table}

\subsubsection*{Daniels (1996)}

Das Modell von Daniels verwendet Teile der Berechnungsvorschriften
von Rager und Garetz aber eigene Tabellen für die Bestimmung
der Bitterausbeute basierend auf der Kochzeit und des
verwendeten Hopfenprodukts (\autoref{table:danielsbakt} und \autoref{table:danielsbaktpellets}).
Der Faktor $\FKd$ ist nach Rager mit \autoref{eq:ragerfkd} zu berechnen.
Für den Faktor $\FHR$ hat Daniels eine eigene Tabelle vorgesehen,
die hier nicht angeführt wurde, nachdem dessen Berechnung auch durch
\autoref{eq:garetzhr} möglich ist.
Die gesamte Bitterausbeute wird gemäß \autoref{eq:danielsba} gebildet.
Neben Moshers Rechenschieber bilden noch weitere Quellen und eigene
Tests die Datenbasis für Daniels Tabellen. \parencite[80,85-88]{Daniels1996}

\begin{table}[H]
\centering
\begin{tabular}{rr}
\toprule
\multicolumn{1}{c}{\textbf{Kochzeit [min]}} & \multicolumn{1}{c}{\textbf{Ausbeute [\%]}} \\
\midrule
0–9            & 5  \\
10–19          & 12 \\
20–29          & 15 \\
30–44          & 19 \\
45–49          & 22 \\
60–74          & 24 \\
>74            & 27 \\
\bottomrule
\end{tabular}
\caption{Bitterausbeute für Dolden nach \citeauthor{Daniels1996} \parencite[80]{Daniels1996}}
\label{table:danielsbakt}
\end{table}

\begin{table}[H]
\centering
\begin{tabular}{rr}
\toprule
\multicolumn{1}{c}{\textbf{Kochzeit [min]}} & \multicolumn{1}{c}{\textbf{Ausbeute [\%]}} \\
\midrule
0–9            & 6 \\
10–19          & 15 \\
20–29          & 19 \\
30–44          & 24 \\
45–49          & 27 \\
60–74          & 30 \\
>74            & 34 \\
\bottomrule
\end{tabular}
\caption{Bitterausbeute für Pellets nach \citeauthor{Daniels1996} \parencite[80]{Daniels1996}}
\label{table:danielsbaktpellets}
\end{table}

\begin{equation}
\BA \uper = \BAKt \cdot \FKd \cdot \FHR
\label{eq:danielsba}
\end{equation}


\subsubsection*{Noonan (1996)}

Die Bitterausbeute nach Noonan ist basierend auf der relativen Dichte und
der Kochzeit der jeweiligen Hopfengabe zu bestimmen. Für Dolden ist
dabei der Wert aus \autoref{table:noonanbakt} und für Pellets aus
\autoref{table:noonanbaktpellets} zu entnehmen. Ob sich die
relative Dichte dabei wie bei Rager auf die Pfannenvollwürze
bezieht, ist aus Noonans Publikation nicht ersichtlich. Die in der Quelle
fehlerhafte angegebene Bitterausbeute für Pellets bei 90~Minuten und
einem Extraktgehalt von 20,5 bis 23~\%w/w wurde durch korrigiert.

\begin{table}[H]
\centering
\begin{tabular}{rrrrrr} 
\toprule
\multicolumn{1}{c}{\textbf{Kochzeit [min]}} & \multicolumn{5}{c}{\textbf{Ausbeute [\%]}}                                 \\
Spezifische Dichte [g/cm³]:                    & 1,032–50 & 1,050–65 & 1,065–75 & 1,075–85 & 1,085–95  \\
Extraktgehalt [\%w/w]:                    & 8–12,5 & 12,5–16 & 16–18 & 18–20,5 & 20,5–23  \\
\midrule                                             
0                                            & 5        & 4        & 4                            & 4                            & 3                             \\
5                                            & 5        & 5        & 5                            & 4                            & 4                             \\
15                                           & 8        & 8        & 7                            & 7                            & 7                             \\
30                                           & 15       & 14       & 13                           & 13                           & 12                            \\
60                                           & 28       & 26       & 24                           & 23                           & 21                            \\
90                                           & 31       & 28       & 27                           & 26                           & 24                            \\
\bottomrule
\end{tabular}
\caption{Bitterausbeute für Dolden nach \citeauthor{Noonan1996} \parencite[215]{Noonan1996}}
\label{table:noonanbakt}
\end{table}

\begin{table}[H]
\centering
\begin{tabular}{rrrrrr} 
\toprule
\multicolumn{1}{c}{\textbf{Kochzeit [min]}} & \multicolumn{5}{c}{\textbf{Ausbeute [\%]}}                                 \\
Spezifische Dichte [g/cm³]:                    & 1,032–50 & 1,050–65 & 1,065–75 & 1,075–85 & 1,085–95  \\
Extraktgehalt [\%w/w]:                    & 8–12,5 & 12,5–16 & 16–18 & 18–20,5 & 20,5–23  \\
\midrule
0                                            & 5        & 5        & 4                            & 4                            & 3                             \\
5                                            & 6        & 6        & 5                            & 4                            & 4                             \\
15                                           & 12       & 12       & 10                           & 9                            & 8                             \\
30                                           & 18       & 17       & 16                           & 16                           & 15                            \\
60                                           & 31       & 28       & 26                           & 25                           & 23                            \\
90                                           & 33       & 30       & 28                           & 27                           & 25                     \\
\bottomrule
\end{tabular}
\caption{Bitterausbeute für Pellets nach \citeauthor{Noonan1996} \parencite[215]{Noonan1996}}
\label{table:noonanbaktpellets}
\end{table}

\subsubsection*{Tinseth (1997)}

Während Tinseth in den Neunzigerjahren an
der Oregon State University sein Doktoratsstudium in Chemie absolvierte,
betrieb er nebenbei einen kleinen Hopfenhandel mit dem Namen HopTech.
Für dessen
Onlinepräsenz stelle er Informationen zur Hopfenausbeute zusammen,
die er aus Brauliteratur.
Durch kleine Gefälligkeiten wurde es ihm erlaubt die Messgeräte
der USDA Hop Labs für eigene Zwecke zu nutzen. Zum Zeitvertreib
und aus der Vermutung heraus, dass die Ausbeutekurve von Ragers
Modell die stattfindende
chemische Reaktion inkorrekt beschrieb, begann er dann mit
der Entwicklung eines eigenen Berechnungsverfahrens.
Grundlagen für sein Ergebnis waren Messergebnisse
seiner eigenen Brauversuche, Messdaten des Brauprogramms der
Universität und publizierte und nicht publizierten Daten verschiedener
Brauereien. Die von Tinseth erhobenen Daten sind für
Brauvorgänge mit Hopfendolen gültig. Pellets waren zur
damaligen Zeit in guter Qualität schlichtweg nicht für Privatpersonen
verfügbar. \parencites[0:55:45-1:08:00]{Beechum2017a}[2:10-6:30]{Smith2011}

Die Berechnung der Bitterausbeute nach Tinseth erfolgt über
\autoref{eq:tinsethbakt}, \autoref{eq:tinsethgf} und \autoref{eq:tinsethba}
anhand der Kochzeit der jeweiligen Hopfengabe und der mittleren
spezifischen Dichte ($\dKt$) über den gesamten Kochverlauf \parencite{Tinseth1997}.

\begin{equation}
\BAKt \uper = \frac{1 - e^{\left(-0,04 \cdot \textrm{Kochzeit} \umin \right)}}{4,15} \cdot 100
\label{eq:tinsethbakt}
\end{equation}

\begin{equation}
\FKd = 1,65 \cdot 0,000125^{\left(\overline{d_{\mathit{Kt}}} \uden - 1 \right)}
\label{eq:tinsethgf}
\end{equation}

\begin{equation}
\BA \uper = \BAKt \cdot \FKd
\label{eq:tinsethba}
\end{equation}

%\url{https://www.realbeer.com/hops/bcalc_js.html}


Während Tinseths Modell die Isomerisierung der Alphasäure über
die Kochzeit korrekt beschreibt \parencite[43]{Malowicki2005}, ist es
nicht allgemeingültig sondern wie alle anderen Modelle für ein
Brausystem kalibriert. Es sind daher mehrere Faktoren zur Anpassung
vorgesehen (\autoref{eq:tinsethbakt}). Die Konstante 0,04 steuert die
Kurvenform und die Konstante 4,15  die maximale Ausbeute
\parencite{Tinseth1997}.

\subsubsection*{Modellvergleich}

\begin{figure}[H]
\centering
\includegraphics[width=10cm]{graph_utilization.pdf}
\caption{Bitterausbeute bei 10,5 \%w/w Extrakt in Pfannenvollwürze (Ascher, 2022)}
\label{fig:novotnygraph}
\end{figure}

iso octane organisches Lösungsmittel

0:47:20-40
IBUS entsprechen nicht perceived bitterness, andere komponenten

49:45-50:25
sellhammer nicht gravity und sugar theorie, höheres protein

1:03:21- time und gravity weil in literatur
die wichtigsten. how much hops 1:04:20

1.06 pale ale 1.07.40
SELBES rezept gescalt, pale ale, ipa, dipa. ctz, cent cascade
20-43 31,8 mit beersmith berechnet

1:08:00 boil vigor, pellets nicht ausgelegt, nie daten genutzt die pellets
pellets waren nicht von gutre qualität 1:09

1:09:38 58
38, 66
dipa 57,9 l 44 71
1:11:00

1:11:35 within 10 % genauigkeit wenn glück, +30
konsistent, sensorisches eindruck korrelieren
1:12:20

1:18:10
protein level 
\parencite{Beechum2017a}

BSHB

11:10--11:14
easily reliable calculate AA Percent

12:00--12:10
11-30% alpha util

20:55-22:00
From Daten nicht herausrechenbar
10 % Genauigkeit
Nicht einmal genau welche alpha der Hopfen weil nur ein eine geringe 
Probe davon.

22:05-23
within 10 ibu
konsistenz ist wichtiger, feinadjustment, nummer nicht,
korrelation zwischen nummer aus gleichung und sensorischem eindruck
\parencite{Smith2011}

Die Beträge der geschätzten
Bitterausbeuten scheinen insgesamt zu gering zu sein \parencite{Jones1995}.

Die Beträge
der berechneten Bitterausbeuten scheinen insgesamt zu hoch zu sein \parencite{Jones1995}.

\parencite[65]{Hall1997}
Kochbeginn, mitte kochzeit und Ende, Stopfen
45.5
RAGER  57.82
%GARETZ 24.12
MOSHER 35.65
TINSETH  47.54
NOONAN 53.21
DANIELS 46.51

\parencite[65\psqq]{Beechum2017}
Here’s our recommendation: ignore the
number as “concrete reality.” No calcula-
tion is ever going to be perfect for you.
Unless you take the time to dial in your
process, analyze your beers, and create
your own utilization curves, the number
will always be a bit of a lie. Instead, treat it
as a squishy imprecise landmark that only
has meaning for you.
Learn what calculated IBU values of 15,
30, 50, 70, and so on mean to your taste
buds when brewed on your equipment.
Use those landmarks to hit what you want
to taste. That’s the only value to the num-
ber, because, frankly, the number itself is
fairly taste-free.

\parencite{Bonham2001}
Kochzeit , Hopfen bit bekannter Alphasäure und gleicher Brauverfahren
Korrigiert OG
ASBC Mittelwert, 33 Bier: 62.1, 80\% innerhalb einer Abweichung von 15\%, 50\% innerhalb
von  bis 20 Abweichung

Rager: 53.5
Burch: 45.6
Garetz: 41
Mosher: 61
Tinseth: 48
Daniels: 67.9
Noonan: 56.9



\subsection*{Aktuelle Ansätze}

\subsubsection*{Wolf (2012)}

„\href{https://www.maischemalzundmehr.de/index.php?inhaltmitte=toolsiburechner}{Maische, Malz und Mehr}“

die Nachisomerisierungzeit (Zeit zwischen Kochende und Whirlpool) wird zur Kochzeit jeder Hopfengabe hinzugerechnet, allerdings entsprechenend ihrer Dauer und der Geschwindigkeit der Abkühlung mit einem niedrigeren Wert gegenüber der regulären Kochzeit (siehe Theorieteil) 

bei Verwendung von Hopfenpellets wird gegenüber Hopfendolden eine 10% höhere Hopfenausnutzung angenommen
Da die Tinseth-Formel auf Experimenten mit Hopfendolden basiert, wird bei der Verwendung von Hopfenpellets die Hopfenausnutzung mit dem Faktor 1.1 nach oben korrigiert, während bei Verwendung von Hopfendolden nicht korrigiert wird. 

Weiterhin hat Glenn Tinseth seine Formel mit Hopfengaben direkt zu Kochbeginn entwickelt. Durch Ausscheidungen mit dem Würzebruch gehen der Würze jedoch eine gewisse Menge an Bitterstoffen verloren. Der IBU-Rechner ehöht die Hopfenausnutzung bei Gaben nach dem Würzebruch um den Faktor 1.1 gegenüber der Tinseth-Formel. Bei Vorderwürzehopfung bzw. bei Hopfengaben zu Kochbeginn gilt die Tinseth-Formel unverändert. 

bei Hopfengaben nach dem Würzebruch (>15 min nach Kochbeginn, d.h. Gesamtkochzeit minus individuelle Hopfenkochzeit > 15) wird eine 10% höhere Hopfenausnutzung angenommen 

In der Zeit zwischen Kochende und Kühlschlange werden weiterhin alpha-Säuren zu iso-alpha-Säuren isomerisiert. Diese Nachisomerisierung wird von der Tinseth-Formel nicht berücksichtigt. Da sie bei Temperaturen <100°C stattfindet, verläuft sie nicht so schnell wie die reguläre Isomerisierung während des Hopfenkochens. Je niedriger die Temperatur, desto langsamer. Die Nachisomerisierung kann mit dem IBU-Rechner jedoch abgeschätzt werden. Dazu wird die Nachisomerisierungszeit zur Würzekochzeit jeder Hopfengabe hinzugerechnet, allerdings entsprechenend ihrer Dauer und der Geschwindigkeit der Abkühlung der Würze mit einem niedrigeren Wert gegenüber der regulären Kochzeit. Dabei wird davon ausgegangen, dass sich die Geschwindigkeit der Isomerisierung alle 10°C unter 100°C halbiert. Diese Abhängigkeit kann mit einem Faktor ausgedrückt werden, welcher bei 100°C daher 1.0, bei 90°C 0.5 und bei 80°C 0.25 usw. entspricht. über eine Regression kann er rechnerisch auch wie folgt ermittelt werden: 

%Isomerisierungsgeschwindigkeitsfaktor = 0.001 x e^0.069 x Temperatur

Mit der Nachisomerisierungszeit multipliziert ergibt dieser Wert die Effektive Kochzeit während der Nachisomerisierung. Bei einer Nachisomerisierung von 30 min und einer Temperatur von 90°C, entspricht die Zeit der Nachisomerisierung effektiv einer Kochzeit von nur 15 min %(Effektive Kochzeit = 30 min x 0.001 x e^0.069 x 90 = 15 min). In der Realität läuft die Nachisomerisierung allerdings nicht bei einer konstanten Temperatur ab. Die Würze kühlt sich in der Nachisomerisierungszeit ausgehend von ca. 100°C kontinuierlich ab. In dieser Abkühlungsphase verringert sich natürlich der Isomerisierungsgeschwindigkeitsfaktor permanent. Messen wir die Temperatur am Ende der Nachisomerisierungszeit, können wir den Isomerisierungsgeschwindigkeitsfaktor allerdings über den Temperaturbereich der Nachisomerisierung (100°C bis Temperatur am Ende der Nachisomerisierung) integrieren: 

%Gleichung 3: Integrierter Isomerisierungsgeschwindigkeitsfaktor = 0.046 e^(0.031 x Temperatur am Ende der Nachisomerisierung). 

Dieser Integrierte Isomerisierungsgeschwindigkeitsfaktor ist dann der Wert, welcher mit der Nachisomerisierungszeit multipliziert die Zeit ergibt, welche im IBU-Rechner zur regulären Kochzeit hinzugerechnet wird. So werden z.B. bei einer Nachisomerisierungszeit von 30 min, während der die Würze auf 80°C abkühlt (errechneter Integrierter Isomerisierungsgeschwindigkeitsfaktor = 0.53), zu jeder Hopfengabe 16 min (30 min x 0.53 = 16 min) zusätzliche Kochzeit hinzugerechnet. Insbesonder bei reichlichen Hopfengaben gegen Kochende, d.h. wenn der Hopfen noch nicht "ausgelutscht" ist, hat die Nachisomerisierung einen enormen Einfluss auf die Gesamtbittere. 

Bereits seit 2012

\parencite{Wolf2022}

\subsubsection*{Hosom – mIBU (2015)}

\url{https://jphosom.github.io/alchemyoverlord}

\parencite{Hosom2015}

\subsubsection*{Novotný (2016)}

\url{https://www.diversity.beer/2018/02/ibu-spreadsheet-english-version.html}
\parencite{Novotny2016}
\parencite{Novotny2018}

\subsubsection*{Smith (2019)}

Zur Berechnung der Bitterausbeute im Whirlpool wendet \citeauthor{Smith2019}
einen von der Temperatur abhängigen, durch \autoref{eq:smithkfwp} definierten,
Korrekturfaktor an. Dabei ist für Hopfengabe, die erst während des Whirlpools
erfolgen, die Ausbeute für eine äquivalente Kochzeit zu berechnen und dann der
Korrekturfaktor anzuwenden. Bei Hopfengaben, die bereits vor dem Whirlpool erfolgt
sind, ist berücksichtigen, dass bereits ein Teil Alphasäuren isomerisiert wurde.
Der verbleibende nicht isomerisierte Säuregehalt ist die Basis für weitere
Berechnungen. \parencite{Smith2019}

\begin{equation}
\mathit{F}_{\mathit{WP}} = 2,39 \cdot 10^{11} \cdot e^{\left(\frac{-9773}{\mathit{WP-Temperatur}\:[K]} \right)}
\label{eq:smithkfwp}
\end{equation}

\subsubsection*{Hosom – SMPH (2021)}

\parencite{Hosom2021}

\section*{Berechnungsbespiel}

\section*{Zusammenfassung}

\parencite{Janish2019}
\parencite{Hieronymus2012}
\parencite{Nottebohm2020}
\parencite{Pyle1995}
\parencite{Justus2018}
\parencite{Parkin2017}
\parencite{Bishop1964}
\parencite{Nickerson1979}
\parencite{Calado2019}
\parencite{Weiss2019}

\parencite{Bruecklmeier2017}
\parencite{Bruecklmeier2018}

Bieranalyse Fuchs crica 30 € \url{https://bieranalyse.de/}
Weinlabor Krauß: \url{https://www.weinlabor-krauss.de}

\printbibliography[title=Quellen]

\end{document}