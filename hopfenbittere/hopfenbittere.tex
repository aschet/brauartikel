% Copyright 2022 Thomas Ascher
% SPDX-License-Identifier: CC-BY-SA-4.0

\documentclass[a4paper,parskip=half]{scrartcl}

\usepackage[T1]{fontenc}
\usepackage[ngerman]{babel}
\usepackage{csquotes}
\usepackage{chemformula}
\usepackage[regular,condensed,sfdefault]{roboto}
\usepackage{booktabs}
\usepackage{multicol}
\usepackage{graphicx}
\usepackage{float}
\usepackage[style=apa,backend=biber]{biblatex}
\DeclareLanguageMapping{ngerman}{ngerman-apa}
\usepackage[hidelinks,pdfencoding=auto,
  pdfauthor={Thomas Ascher},
  pdfusetitle,
  pdfkeywords={Bier,Bitterung,Bitterausbeute,IBU,Tinsetz,Rager,Garetz}]{hyperref}
\usepackage{microtype}
\DisableLigatures{encoding=*, family=*}

\addto\extrasngerman{
\def\figureautorefname{Abb.}
\def\tableautorefname{Tab.}
\def\equationautorefname{Gl.}
}

\addto\captionsngerman{
\renewcommand{\figurename}{Abb.}
\renewcommand{\tablename}{Tab.}
}

\NewBibliographyString{gethesis}
\DefineBibliographyStrings{ngerman}{
  mathesis = {Masterarbeit},
  gethesis = {Diplomarbeit},
}

\title{Eine bittere Angelegenheit: Berechnungen rund um den Hopfen}
\author{Thomas Ascher <thomas.ascher@gmx.at>}
\date{1. Jänner 2022, \href{http://creativecommons.org/licenses/by-sa/4.0/}{CC BY-SA 4.0}}

\addbibresource{hopfenbittere.bib}

\begin{document}
\maketitle

\section*{Einleitung}

Meiste Bittere vom Hopfen, ein Teil von Röstmalzen \parencite[11]{Garetz1994}

\section*{Hopfenbestandteile}

\parencite[11]{Garetz1994}
Alphasäuren: humulone, cohumulone, adhumulone, sind Bittere und kommen in
verschiednen Hopfensorten in unterschiedlichen Verhältnissen vor
Glauben: Mit hohem Cohumulone Gehalt unangenehmere Bittere.
Beta Säuren, nicht Bitter? Ignoriert hinsichtlich Bitterpotential
Alphasäuren schlecht löslich in Würze bei geringeren Temperaturen
und normalem pH-Wert.

\parencite[20]{Garetz1994}
Weibliche Pflanze der Hopfengattung Humulus lupulus, Alpha Beta
Säuren teil des resins, Resin mit Hopfenölen in gelben
Lupulindrüsen in Dolden produziert.


\section*{Alphasäure und Isomerisierung}

\parencite[12]{Garetz1994}
Durch Kochen geht ein Teil der Alphasäuren in Lösung. Durch Kochen
chemischen Prozess der Isomerisierung. Iso-Alpha Säuren. Sehr löslich.
Bei Kochen geht AS schnell in Lösung, Isomerisierung dauert lange.
Ab 90 Min. Plateu erreicht.
Nur Ungefähr 20\% Alphasäure -> Utilisierung landen als Iso-Alpha im
fertigen Bier.
Verlust: zu kurze Isomerisierugszeit, fällt mit Kühltrub aus,
während Fermentierung und Filterung

\parencite[13]{Garetz1994}
Kettle Utilization, Boil Utilization während Kochen

\parencite[14]{Garetz1994} 
Bitte in IBU, mg of iaa in l Bier, Verluste während des gesamten
Prozesses

\parencite[34]{Garetz1994}
Alphasäure Bestandteil am Gewicht des ganzen Hopfens. Herangezogen
zur Bitterberechnung. Betasäure wird ignoriert.

\parencite[35]{Garetz1994}
Nobelhopfen Alpha:Beta = 1:1, nieder cohumulone.
mehr cohumulone harsch, gibt aber hopfen mit hohem cohumulone
Anteil, die nicht als Harsch bekannt sind. Ineratkion
mit anderem.

\section*{Hopfenprodukte}


\parencite{Annemueller2015}
\parencite{Beechum2017}
\parencite{Janish2019}
\parencite{Hieronymus2012}
\parencite{Nottebohm2020}
\parencite{Hall1997}
\parencite{Rager1990}
\parencite{Daniels1996}
\parencite{Mosher1994}
\parencite{Holle2010}

\printbibliography[title=Quellen]

\end{document}