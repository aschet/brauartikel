% Copyright 2022 Thomas Ascher
% SPDX-License-Identifier: CC-BY-SA-4.0

\documentclass[a4paper,parskip=half]{scrartcl}

\usepackage[T1]{fontenc}
\usepackage[naustrian]{babel}
\usepackage{csquotes}
\usepackage[regular,condensed,sfdefault]{roboto}
\usepackage{booktabs}
\usepackage{graphicx}
\usepackage{chemformula}
\usepackage{amsmath,amsfonts,amssymb}
\usepackage{icomma}
\usepackage{textcomp}
\usepackage{gensymb}
\usepackage[italic,symbolgreek]{mathastext}
\usepackage{float}
\usepackage[style=apa,backend=biber]{biblatex}
\DeclareLanguageMapping{naustrian}{naustrian-apa}
\usepackage[hidelinks,pdfencoding=auto,
  pdfauthor={Thomas Ascher},
  pdfusetitle,
  pdfkeywords={Bier,Bitterung,Bitterausbeute,IBU,Tinseth,Rager,Garetz}]{hyperref}
\usepackage{microtype}
\DisableLigatures{encoding=*, family=*}

\addto\extrasnaustrian{
\def\figureautorefname{Abb.}
\def\tableautorefname{Tab.}
\def\equationautorefname{Gl.}
}

\addto\captionsnaustrian{
\renewcommand{\figurename}{Abb.}
\renewcommand{\tablename}{Tab.}
}

\NewBibliographyString{gethesis}
\DefineBibliographyStrings{naustrian}{
  mathesis = {Masterarbeit},
  gethesis = {Diplomarbeit},
}

\title{Eine bittere Angelegenheit: IBU Berechnungen}
\author{Thomas Ascher <thomas.ascher@gmx.at>}
\date{1. Jänner 2022, \href{http://creativecommons.org/licenses/by-sa/4.0/}{CC BY-SA 4.0}}

\addbibresource{hopfenbittere.bib}

\begin{document}
\maketitle

\section*{Einleitung}

\parencite{Bastgen2020}
%TODO

Meiste Bittere vom Hopfen, ein Teil von Röstmalzen \parencite[11]{Garetz1994}

\parencite[10]{Garetz1994}
Aroma, Bittere und Haltbarkeit

\parencite[103]{Garetz1994}
Brauer interesserit: Alpha, Beta und Ölgehalt.


\section*{Hopfenbestandteile}

\parencite[11]{Garetz1994}
Alphasäuren: humulone, cohumulone, adhumulone, sind Bittere und kommen in
verschiednen Hopfensorten in unterschiedlichen Verhältnissen vor
Glauben: Mit hohem Cohumulone Gehalt unangenehmere Bittere.
Beta Säuren, nicht Bitter? Ignoriert hinsichtlich Bitterpotential
Alphasäuren schlecht löslich in Würze bei geringeren Temperaturen
und normalem pH-Wert.

\parencite[20]{Garetz1994}
Weibliche Pflanze der Hopfengattung Humulus lupulus, Alpha Beta
Säuren teil des resins, Resin mit Hopfenölen in gelben
Lupulindrüsen in Dolden produziert.

\parencite[100]{Garetz1994} Einige Hopfenaromen entstehen erst
durch Oxidation. Besonders bei Nobelhopfen interessant.

\parencite[119]{Garetz1994} 
Adhumulone nur in kleinen Mengen.

\parencite[120]{Garetz1994} 
nur geringe Bittere, noch schlechter Wasserlöslich, mehrere Säuren,
lupulone, colupulone und adlupulone
Während lagerung und und kochen oxiieren und bilden Bitterstoffe

\parencite[60]{Beechum2017}
Genaue Messung von iso-alpha zeitintensiv und teuer
in 1950 verinfachte methode, die von der ASBC in 1968
als Standard etabliert wurde.
spectrophotometer, Lichtabsorbtion bei Wällenlänge
von 275 Nanometer wenn durch präperierte Probe
Korreliert miit der Konzentration von iso-alpha.
Beeinflusst durch Dry Hops und andere Substanzen

Iso-Alpha acids aren’t the only things that cause an organoleptic
sensation of bitterness, the IBU is an incomplete picture of just how bitter a
beer really is.

Glenn Tinseth Modell ist der de facto standard
Messungen und Berechnungen wurden für sein Brausystem und Prozess entwickelt

\parencite[61]{Beechum2017}
Spezifische dichte der Kochwürze wird als Konstante betrachtet
Nur für gekochte Hopfen
Nur für Dolen, es gibt verschiedene Korrekturfaktoren

IGOR volunteers Basic Pale Ale, Basic IPA, or
Basic DIPA 
Versuch mit 22 Suden, verschiedene Rezpete
Oregon Brew Lab, sensorisch getestet.

\parencite[62]{Beechum2017}
10 mL of beer with a little bit of Salzsäure and iso-Octan. The resulting
solution is agitated vigorously until it sep-
arates into two to three phases. The 275
nm absorption of the distinct clear phase
at the top of the sample is compared to
pure iso-octane. That comparison gives
you the official IBU number.

APAs. Both the median and mode of the
sample set were spot on with the formula’s
calculated estimate. As the beers increase
in gravity, the wobbliness we expected
begins to emerge. The IPA is still close,
with the average about 5 IBUs below the
predicted value. The DIPA, though, is a
full 21 IBUs (about 28 percent) below the
calculated value.

\parencite[65]{Beechum2017}
we think the “undershoot” is due to the
rapid chilling procedures that are more
common today than they were during the
formulae’s development.
Well,
one is that there’s probably more work
to be done to figure out how the
numbers change with different chill-
ing regimes, kettle geometry, and boil
vigor.

Well,
one is that there’s probably more work
to be done to figure out how the
numbers change with different chill-
ing regimes, kettle geometry, and boil
vigor.

\parencite[65\psqq]{Beechum2017}
Here’s our recommendation: ignore the
number as “concrete reality.” No calcula-
tion is ever going to be perfect for you.
Unless you take the time to dial in your
process, analyze your beers, and create
your own utilization curves, the number
will always be a bit of a lie. Instead, treat it
as a squishy imprecise landmark that only
has meaning for you.
Learn what calculated IBU values of 15,
30, 50, 70, and so on mean to your taste
buds when brewed on your equipment.
Use those landmarks to hit what you want
to taste. That’s the only value to the num-
ber, because, frankly, the number itself is
fairly taste-free.

\parencite[55]{Hall1997}
Hopfenbittere aus den Bitterharzen in den gelben Lupulindrüsen.
alpha, beta, gamma. alpha+beta sind weichharze weil
lösbar in Hexan. Gamma nicht darinlösbar darum
hartharze. Alpha-fraktion humulone, cohumulone
adhumulone, prehumulone and posthumu-but repeating such a success can be difficult.
lone
The alpha acids will dissolve
in hot wort, up to 250 mg/L at a pH of 5 and
a temperature of 212 degrees F (100 degrees
C). They are not very soluble in beer, with
its lower pH and temperature, and will pre-
cipitate out if their concentration is higher
than 5 mg/L at a pH of 4 and temperature of
32 degrees F (0 degrees C)
uring the kettle boil, the alpha acids
undergo a molecular rearrangement called
isomerization. The resultant chemicals are
called iso-alpha acids, and there is a corre-
sponding version for each humulone (iso-
humulone, isocohumulone, etc.). The iso-
alpha acids are much more soluble in wort
and beer,
The beta-fraction of the hop resins is
composed of the beta acids and many other
chemicals, including the oxidation products
of the alpha and beta acids that result from
aging (De Clerck, 1957). The beta acids are
known as lupulones and occur in varieties
similar to the humulones.
lupulone,
colupulone, adlupulone, prelupulone
postlupulone
postlupulone
less soluble than the alpha acids (0.7 mg/L
they do contribute some bitterness to beer
through their oxidation products. The bit-
besonders in gealterten Hopfen, wird
als unangenehm empfunden.
The hard resins do not contribute to the
bitterness of the finished beer.



\section*{AAU, HBU}

\parencite[122\psq]{Garetz1994} 
ibu nicht messen
alpha mal hopfenmenge
Dave line
Abhängig von Sudmenge, beschreiben keine Verluste oder bittere
des finalzen biers.

\parencite[40]{Favre1990}
AAC=HBU=aa\%*oz
Für Pellets 10 bis 15 Prozent höher

\parencite[60]{Papazian2003}
MBU=aa\%*oz

\parencite[55\psq]{Hall1997}
specifying simply “three ounces of hops,”
Dave line 1985
(1991) HBU Papazian
Alpha Acid Unit
Homebrew Bittering Unit
wide variations in the bitterness level.

the volume of the wort and the boil time
Many recipes sidestep this problem by spec
ifying the volume and boil time explicitly.


\section*{Bitterausbeute}

\parencite[50]{Holle2010}
Utilisation rate wie viel alphasäuren vom Hopfen in Lösung gehen
und im Bier laden. 10 bis20 verlust

Höhere utilisation bei
vigor boild und temperature (umgebungsdruck, seehöhe)
geringerer spezifischen Dichte
geringere Hopfenmenge
höherer pH Wert, -> zum Teil unangenehmere Bittere

\parencite[159]{Annemueller2015}

Art (zerkleinerungsgrad9 des Hopfenprodukts. Pelletts 35, Presshopfen
30. bei 100°C 90 min.

Kochdauer (je länger des höher), Konzentration der alphasäuren (je höher desto geringer)
Kochtemperatur (höher schnellere Isomerisierung)
Ph-Wert der Würze (je höher, desto höher).
Bruch-/Trumbenge: je höher desto geringere Ausbeute 

\parencite[160]{Annemueller2015}
Verlust von Anstellwürze
Offene Warmgärung 40..50
Druckgärung 10...15

\parencite[162\psq]{Annemueller2015}
Bezieht sie auf Kalte Anstellwürze oder Fertigbier. Leitet sich aus der
Formel für die Bitterstoffausbeute ab. 

AW Anstellwürze
Pfvw = Pfannenvollwürze

BA=IBU AW * 100 / IBU Pfvw

BA = IBU Bier * 100/IBU Pfvw

\parencite[160-164]{Annemueller2015}
Bitterstoffgabe, Bitterstoffbilanz
Betriebliche Bitterstoffausbeute

\parencite[124]{Garetz1994} 

\% u = alpha present / alpha used * 100
efficency of alpha acid as iso alpha acid in final beer

Einflüsse:
Hopfenmenge
Hopenform
Kohtemperatur und Zeit
Volumen
Stammwürze
pH der Würze
Hopfengabe während Gärung
Yest Groth und floullation
Filtrierung

\parencite[125]{Garetz1994} 
Während des Kochens werden säuren und öle extrahiert 
bei dolden innerhalb von 10 bis 15 Minuten, bei Pallets
schneller.
Nach kochen fallen die unisomerisierten alpha säuren
aus, sonst spätestens bei Gärung

Isomerisierung abhängig von Volumen und Temperatur beim
Kochen, nach einem Zeitpunkt dreht sich die Reaktion
um. Nach circa 2 Stunden degraded. 

\parencite[126]{Garetz1994} 
Max. Isomerisierungsrate von 60-75, 8-10 trubverlust

iso absobiert onto Hefezellenwände, 5\% Schwankung auf
Utilisation basierend auf langsame und normale Flukulation
Hefewachstum

Auch mit CO2 ausgetrieben. In Hefedecke, macht bis zu 18\%
aus. Sammelt sich am Rand des Fermenters, verhärten durch
Oxidierung

\parencite[133]{Garetz1994} 
In Kräusen. J , höhere Stammwürze höherer Verlust, 

\parencite[128]{Garetz1994} 
Einige Korrekturfakten aus veröffentlichen Forschungsdaten. 
Messungen einiger gebrauten Biere.
Zu wenig Messdaten erhobens. Feedback erbeten.

10 Minuten zur Extraktion, 5-5.5 pH, vigorous boil

\parencite[130]{Garetz1994} 
Kochzeiten unter 30 min keine Pellet Korrektur
Bei pellets platzen die lupulindrüsen. 

Hopfensack ebenfalls einfluss,geringere utisilierung

\parencite[153]{Garetz1994} 
Beim Kochen mehrere Denkschulen
- bei Kochbeginn
- erste Anzeichen von Würzebruck
-> Bindet hopfenharze, reduziert Ausbeute; primär werden
Tanine gebunden
Vermindert Oberflächenspannung, reduziert überkochen
bei der Zugabe kann es zum überkochen kommen

\parencite[158\psq]{Garetz1994} 
Hopfen sind primärer Quelle für Bitterstoffe, balance
maskiert durch Malzssüße, 
Röstmalze liefern Bitterstoffe
Wasserprofil hat Einwirkung.

BU=Bittering unit (IBU x 1.125)
\parencite[215]{Noonan1996}


\parencite[57]{Hall1997}
optimistically peaks at 35 percent
Storage deterioration: oxidation reduziert bittere durch alpha und erhöht beta
abhängig von Temperatur, Alter, Luft (Hopfenform)

chemische abläufe: oxidiert zu anderen produkten, 
anti-isohumulones doppelt so bitter bis 10\%
höherer pH höherer Rate
Rückreaktion schließendlich zur dekomposition

Physitsche Trennung: In the hot wort at the end of the boilalpha acids are safely ensconced in the fin-
chronological order through the cycle ofthe utilization rate is about 50 percent
heißt und kalttrub aus lösung 7\%
durch co2 entfernent in kräusen, auch behälterrand
Filterung, hefezellen +- 5\%

Staling reactions: oxidation reaktion im fertigen beer, Käse

\parencite[58]{Hall1997}
Schätzung alle Effekte für Bitterausbeute zu beachten schwierig
Starke Schwankungen bei Doldenhopfen
geringer bei Pellets weil geblendet
abhängig von Kochtemperatur abhängig von Seehöhe
undissolved iso-alpha acids and unsaturated wort -> boil vigor, gravity -> viscosity
Hopfenform, pellets > dolden
Wasserprofil bitterness auswirkung
andere effekte -> oxidationsprodukte, röstmalze -> tannine, 

kommerzielle brauereien können messen und batches blenden.

acbc (1992). zentrifuge + spectrophototmeter

Wahrnehmungsschwelle 4 IBU

\parencite[59]{Hall1997}
geheimsamer Framework, fehler ausgebessert
basierend auf Kochzeit. Darauf werden weitere
Korrekturfaktoren angewendet.



\section*{Alphasäure und Isomerisierung}

\parencite[51]{Davidson1997} gleiche Atomzusammenstellung wie
eine andere Substanz aber in anderer Struktur, struktuerelle
Veränderung
- nach 4 Stunden Rekation nocn nicht abgeschlossen
- util = alpha in work / alpha zugegeben

\parencite{MEBAK2020}
Die wichtigsten Bitterstoffe in Würze und Bier sind die iso-alpha-Säuren. Zudem lassen sich, vor allem in Würze, noch beta-Säuren sowie -Säuren nachweisen. Darüber hinaus enthalten Würze und Bier andere Derivate der Hopfenbittersäuren, insbesondere Oxidationsprodukte, die ebenfalls zum Bittergeschmack beitragen.
Die Bitterstoffe, hauptsächlich iso-alpha-Säuren, werden mit iso-Oktan aus der
angesäuerten Probe extrahiert und ihre Konzentration im Auszug spektralphotometrisch bestimmt.



\parencite[120\psq]{Garetz1994}
Bittere historisch beschrieben mit gegebener Hopfenmenge

internationaler Standrard IBU oder kurz BU, amerikaner und
europäer eigene Messmethoden
 
\parencite[121]{Garetz1994} 
IBU mg iso alpha in einem liter Bier, fertiges bier
ein Teil geht verloren. 

Messung: Spektrometer, mit ISO-Octane lösungsmittel, mixture
wird gemessen bei UV licht, genauer gesagt die abosption
bei gewissen Wellenlängen, absorption proportional
zu IBU (die wird korreliert)

\parencite[12]{Garetz1994}
Durch Kochen geht ein Teil der Alphasäuren in Lösung. Durch Kochen
chemischen Prozess der Isomerisierung. Iso-Alpha Säuren. Sehr löslich.
Bei Kochen geht AS schnell in Lösung, Isomerisierung dauert lange.
Ab 90 Min. Plateu erreicht.
Nur Ungefähr 20\% Alphasäure -> Utilisierung landen als Iso-Alpha im
fertigen Bier.
Verlust: zu kurze Isomerisierugszeit, fällt mit Kühltrub aus,
während Fermentierung und Filterung

\parencite[13]{Garetz1994}
Kettle Utilization, Boil Utilization während Kochen

\parencite[14]{Garetz1994} 
Bitte in IBU, mg of iaa in l Bier, Verluste während des gesamten
Prozesses

\parencite[34]{Garetz1994}
Alphasäure Bestandteil am Gewicht des ganzen Hopfens. Herangezogen
zur Bitterberechnung. Betasäure wird ignoriert.

\parencite[35]{Garetz1994}
Nobelhopfen Alpha:Beta = 1:1, nieder cohumulone.
mehr cohumulone harsch, gibt aber hopfen mit hohem cohumulone
Anteil, die nicht als Harsch bekannt sind. Ineratkion
mit anderem.

\parencite[128]{Garetz1994} 
Max. isomerisierung nach 90 Minuten, max 70\%, vermutlich fallender
ph-Wert reduziert isomerisierungsreaktion.

Höhere Hopfengaben ebenfalls. 40mg/l alpha 

\parencite[129]{Garetz1994} 
Gravity über 1.050 einfluss, Kochtemperatur (abhängig von Seehöhe).


\parencite[78]{Daniels1996}
ph-wert beeinflusst die isomerisierung nicht stark, aber Bitterwahrnahme
hoher ph wert harsche bittere mehr als 50 ppm carbonates

\parencite[49]{Holle2010} Isomerisierung bei Kochtemperatur
aber auch im Whirlpool 90-95


\section*{Hopfenprodukte}

\parencite[80\psq]{Garetz1994}
Doldenhopfen: am wenigsten verarbeitet, manche Brauer glauben
liefert das beste Aroma. Verbraucen viel Lagerplatz.

\parencite[82\psqq]{Garetz1994}
Pellets: pulverisiert und gepresst, verlust von 4 bis 6 \% Alpha
und Ölen.
T90
T45: enriched, teile des Pflanzenmaterials wird mechanisch
entfernt, doppelter Alpha.
Stabilisierte und isomerisierte Plellets, im Normalfall nicht
erhältlich. Ein Teil der Alphasäure in Iso Alphasäure
isomerisiert. Nicht für Dryhopping und Aromagaben.
chemisch verarbeitet, Magnesium hinzugefügt.
Pellets zerfallen zu Pulver
Schwerer zu entfernen, wird im Whilpool entfernt \parencite[87]{Garetz1994}

\parencite[84\psq]{Garetz1994}
Plugs und Bricks: 
Plug gepresster Doldenhopfen, in England gerne für Cask Dry Hopping
mehrere Dolden in Form, braucht einige Minuten bis sie auseinander
brechen. Schwer kleine Poritionen zu entnehmen.

\parencite[88-93]{Garetz1994}
Extrakte, mit mehreren Methoden hergestellt. Chemisch per Lösungsmittel
oder Dampfdestillation. Um an Alphasäure und Öle zu kommen.
Um Brauprozess berechenbarer zu machen und zu verinfachen.
iso-alpha extrakt: am Ende vom Prozess nach kochen oder
nach Gärung -> verluste, nicht sensitiv zu sonnenlicht, chemisch verändert
Hopfenöle: Aroma, Tasting panel hat schlechter gefunden wenn
late hops, wird in England für Casc conditioning, bessere konsistenz
Stark konzentriert, weniger für Anwendung im Heimbereich für
kleine Sude

\parencite[52]{Davidson1997}
nicht pre-isomerisiert
Pre-isomerized Hop Products -> doppelter gehalt
pre-isomerisierte pellets enthalten öle
The reduced (rho) , hexa, and tetra extracts
all provide bitterness and protection against
the sunstruck reaction that produces a
skunky note in beer. Additionally, hexa and
tetra improve foam stability.
The
reduced, hexa and tetra extracts are the
reduced fo rms (by hydrogen addition to
the iso-alpha acids) and the sensory bit-
terness of these compounds differs. Tetra-
iso-extract has a prolonged senso ry bitter-
ness , a lingering bitterness compared to
the normal iso-al pha acids . He xa-iso-
extract is less lingering and prolonged and
is more similar to iso-alpha acids . The
reduced iso-extract also is similar to iso-
alpha acids, in that it tends to be a quick,
clean bitterness.

found isohumulone (with traces of
iso-adhumu lone) was preferred to isoco-
humulone.

\section*{Hopfengaben}

\parencite[15]{Garetz1994}
Spät: Aromaöle verdampschen schnell, verlust nach mehreren Minuten
oder gleich durch Kochen. Oder verändert, nicht mehr gleich
wie im Naturhopfen. 15 Minuten vor Ende. Wenig isomerisierung.

\section*{Effekte der Hopfenalterung}

\parencite[50]{Holle2010}
Je älter der Hopfen desto geringer der alpha, bestimmt über
HSI.

Nördliche Hemisphere einmal Ernte pro Jahr. Mitte August bis
Anfang September.
\parencite[97]{Garetz1994}

\parencite[97]{Garetz1994} Hopfen verliert Alphasäure und Öle
durch Oxidation. Vakuimieren und gekühlte Aufbewahrung verlangsamen
den Prozess. Auch kein Licht. 

\parencite[103]{Garetz1994} Alphasäure oxidiert an luft, oxidationsprodukte
nicht mehr isomerisierbar. Nicht mehr so bitter. Betssäuren bilden
Bitterstoffe. bei Oxidation.

\parencite[104]{Garetz1994} 
Unter gleichen Lagerbedingungen verlieren Hopfen unterschiedlich an
Alpha wegen unterschiedliche Menge Oxidantien.
-HSI spektroskopisch analyse von alpha und beta. Test für Halbarkeit:
Verlust von Alpha und Betasäure über 6 Montate über Raumtemperatur 20c.
Korrelation zwischen beiden. Der ursprüngliche Alphagehalt und der
zkünftige
Ölverluste nicht ausreichend erforscht.

\parencite[104]{Garetz1994} 
HSI Nummer, nich von Händlern, hauptsächlich als interne Referenz
in den Laboren
Percent Alpha remaining or Lost after 6 Months at 20c
\% Alpha /nach 6 Monaten

\parencite[145-151]{Garetz1994}
Taste Titration 
Bier mit bekannter Bittere,, iso-alpha extrakt hinzufügen
bis gleich bitter wie test bier kalibrierter Extrakt, 0,44 IBU

\parencite[52]{Davidson1997}
Oxidation von beta säuren -> ähnliche hulupones wie isohomulone


% TODO ev. Berechnung beschreiben
% TODO Diagramm zu Alphaverlust

\section*{Modelle zur Schätzung der Bitterausbeute}

Die Umrechnung zwischen
Massenanteil in °P und der relative Dichte
erfolgt auf Basis der Plato Tabellen über die Goldiner Gleichungen
(\autoref{eq:calcptosg}) oder vergleichbaren
Näherungsverfahren.
\parencite[140\psq]{Spedding2016}


\begin{equation}
d_{\frac{20}{20}}\:[g/cm^3] = \frac{\degree P}{258,6 - \degree P / 258,2 \cdot 227,1} + 1
\label{eq:calcptosg}
\end{equation}


\parencite[320]{Kunze2004}
BU mg of bitter substance / l Beer
Berechnung auf Volumen kalte Ausschlagwürze contracts (4\%)
bitterness yield, only part of substance
25 und 35\% für kommerzielle Brauereien
Für Hopfengabe müssen für Bitterness Yield, Versuchssude in der Brauerei
Circa ein Drittel der gegebenen Alphasäuren landet im 
100\% Bitterness=aacid * 100 / BY

\parencite[51]{Holle2010}
ibu = mg hop * util * alpha acid \% / l

\parencite[127]{Garetz1994} 
Keine in professioneller Brauliteratur, Messtechnisch erhoben
werden. Nach einigen Brauversuchen lässt sich also einstellen.
Blending mehrer Batches.

\parencite[76]{Daniels1996}
Große Brauereien können große Hopfenmengen mischen und
Biere um unterschiede auszugleichen.

\parencite[51]{Holle2010}
Einziger verlässliche Weg ist labortechnische Analysen.



\subsection*{Kritik}

\parencite{Bonham2001}
Kochzeit , Hopfen bit bekannter Alphasäure und gleicher Brauverfahren
Korrigiert OG
ASBC Mittelwert, 33 Bier: 62.1, 80\% innerhalb einer Abweichung von 15\%, 50\% innerhalb
von  bis 20 Abweichung

Rager: 53.5
Burch: 45.6
Garetz: 41
Mosher: 61
Tinseth: 48
Daniels: 67.9
Noonan: 56.9

\parencite{Jones1995}
Rager
. A lot of us have used his formula with mixed success. Most of us felt
the utilizations were on the high side resulting in too low of an IBU calculation.

Then along came Mark Garetz with his book on "Using Hops" that included yet another formula to
calculated IBUs. His formulas and techniques stirred things up a bit and most brewers felt his IBU
numbers were too high or the utilizations were too low.

The one thing both of these authors did not have was hard data to back up their claims. I think each had
done a few tests, but not really extensive enough, in my opinion, to really convince me that the
calculations were right.

t Glenn has the resources, friends
and equipment to better explore and measure hop utilization. He has friends and tools at the USDA hop
labs at Oregon State University.

knew Glenn was working on this project as we spoke about it over 3 years ago.
The following plot shows that Tinseth's utilization numbers generally fall between the
old Rager and Garetz utilizations.

oth Rager’s and Garetz’s plots follow a familiar "S" shaped curve. Tinseth observes that
it doesn't seem to accurately reflect what's happening in the brew kettle.
I like the look of Glenn's curve. It fits and looks more like a classical chemical reaction as well as a lot of
other things in nature. Let’s take a look at the new IBU calculation as see how it works.

\subsection*{Klassische Modelle}

\parencite[59]{Hall1997}
Rager war erste Methode in Heimbrauliteratur, weit verbreitet.

\parencite[60]{Hall1997}
Garetz niedrigere Utilisierung und neue Korrekturfaktoren.

Tinseth: zwischen Rager und Garetz.

\parencite[60]{Hall1997}
U% = U%bt Fbg Fhf Fhr Fbp Fst Fhb Fyf Ffil
boil gravity
hop form
hop rate
boiling-
point temperature, storage losses, hop bags
yeast flocculation rate and filtration
Basisformel + Korrekturfaktoren

\parencite[62]{Hall1997}
hop utilization rates decrease with increas-
ing boil gravity above 1.050

Correcting the utilization to
account for the hop form also is common.
Leaf hops or hop plugs do not need a cor-
rection, but hops in the pellet form are
reported to have an increased utilization.
The Garetz method sets Fhf equal to 1.1 for
pellets boiled from 10 to 30 minutes, and
unity otherwise. The Mosher method sets
Fhf equal to 1.33 for pellets in general, inde-
pendent of gravity and boil time. Noonan
again uses a table, which gives Fhf between
1.0 and 1.5 for pellets, with maximum val-
ues centering around 15 minutes of boil
time and low boil gravities. Daniels does not
give a value for F hf , although he recom-
mends using something between 1 and 1.25
for pellets. The other methods do not give
a correction factor for hop form, but any of
the above methods may be used with them.

\parencite[63]{Hall1997}
Hopping rate:
Iterativ berechnen oder quadratische
Formel. 
faktor Zulsetzt anwenden

\parencite[65]{Hall1997}
Kochbeginn, mitte kochzeit und Ende, Stopfen
45.5
RAGER  57.82
GARETZ 24.12
MOSHER 35.65
TINSETH  47.54
NOONAN 53.21
DANIELS 46.51

\subsubsection*{Rager}

\parencite[53]{Rager1990}
Grundlage Fred Eckhardt,
Dave Miller and Byron Burch because

A gravity adjustment calculation
is used for high-gravity beers or when
the brew kettle is too small to boil the
full batch. It is calculated by taking
the larger of gravity of boil (GB) minus
.050, divided by 0.2. If neither gravity
is greater than 1.050, then (GA)
equals 0.

12.4 °P

Note: In formulas, percent utilization
must be expressed as a decimal.

\%U Percent utilization
\%A Percent alpha acid
Wgz· Weight in grams
VL Volume in liters
GA Gravity adjustment
GB Gravity of boiling wort
IBU (International) Bittering Units

GA=(GB)- .050 / 0,2

\parencite[54]{Rager1990}

\begin{table}[H]
\centering
\begin{tabular}{rr}
\toprule
\multicolumn{1}{c}{\textbf{Kochzeit [min]}} & \multicolumn{1}{c}{\textbf{Ausbeute [\%]}} \\
\midrule
0–5             & 5 \\
6–10            & 6 \\
11–15           & 8 \\
16–20           & 10,1 \\
21–25           & 12,1 \\
26–30           & 15,3 \\
31–35           & 18,8 \\
36–40           & 22,8 \\
41–45           & 26,9 \\
46–50           & 28,1 \\
51–60           & 30 \\
\bottomrule
\end{tabular}
\caption{Bitterausbeute für Dolden nach \citeauthor{Rager1990} \parencite[54]{Rager1990}}
\label{table:ragerutil}
\end{table}

%((final volume x (final gravity – 1)) / estimated boil volume) + 1 = estimated boil gravity



Näherungsweise lässt sich die Bitterausbeute von \autoref{table:ragerutil}
auch über die \autoref{eq:ragerutil} berechnen \parencite{Steinmeyer2021}. 

\begin{equation}
\mathit{BA}_{\mathit{Tab.}}\:[\%] = 18,11 + \left(13,86 \cdot \tanh{\frac{\mathit{Kochzeit}\:[min] - 31,32}{18,27}}\right)
\label{eq:ragerutil}
\end{equation}

\begin{equation}
\mathit{DA} = \frac{\mathit{GB} - 0,05}{0,2}
\label{eq:ragerga}
\end{equation}

%\newcommand{\bxfc}{\mathit{Rc}_f}
%\newcommand{\bxfctext}{$\smash{\bxfc}$}

\begin{equation}
\mathit{IBU} = \frac{\%U \cdot \%A \cdot W\:[g] \cdot 1000}{V\:[l] \cdot (1 + \mathit{GA})}
\label{eq:rageribu}
\end{equation}

\begin{equation}
W\:[g] = \frac{V\:[l] \cdot (1 + \mathit{GA}) \cdot \mathit{IBU}}{\%U \cdot \%A \cdot 1000}
\label{eq:ragerw}
\end{equation}

\parencite[134]{Garetz1994} 
Nur ein Korrekturfaktor: boil Gravity, Ausbeutewerte sehr optimistisch
gewählt. 

\subsubsection*{Burch}
Brewing Quality Beers

\subsubsection*{Garetz}

g=v*ca*ibu desired/util*alpha*0,1

ibu=g*util*alpha*0,1/vol*CA

\parencite[134-144]{Garetz1994} 
Basiert auf Ragers Formel. Angepasst. Combined
Adjustments statt 1+GA.

Concentration factor
cf=fin vol/boil vol
boil gravity bg=(CF*(starting gravity -1))+1
gravity factor
GF=((BG-1.050/0.2)+1 = GA+1
Hoping rate factor
HF=((CF*desired ibu/260)+1  -> HCF
TF((eleation in feet/550)*0.02+1)
Seehöhe = 1. Negative nummer wenn unter Seehöhe

CA = GF*HF*TF

\begin{table}[H]
\centering
\begin{tabular}{rr}
\toprule
\multicolumn{1}{c}{\textbf{Kochzeit [min]}} & \multicolumn{1}{c}{\textbf{Ausbeute [\%]}} \\
\midrule
0–10            & 0 \\
11–15           & 2 \\
16–20           & 5 \\
21–25           & 8 \\
26–30           & 11 \\
31–35           & 14 \\
36–40           & 16 \\
41–45           & 18 \\
46–50           & 19 \\
51–60           & 20 \\
61–70           & 21 \\
70–80           & 22 \\
81–90           & 23 \\
\bottomrule
\end{tabular}
\caption{Bitterausbeute für Dolden nach \citeauthor{Garetz1994} \parencite[138]{Garetz1994}}
\label{table:garetzutil}
\end{table}

Näherungsweise lässt sich die Bitterausbeute von \autoref{table:garetzutil}
auch über die \autoref{eq:garetzutil} berechnen \parencite{Steinmeyer2021}. 

\begin{equation}
\mathit{BA}_{\mathit{Tab.}}\:[\%] = 7,2994 + \left(15,0746 \cdot \tanh{\frac{\mathit{Kochzeit}\:[min] - 21,86}{24,71}}\right)
\label{eq:garetzutil}
\end{equation}


statt decimal 0,21 der Prozentzahl 
g = v in Liter * CA * IBU / util  alpha * 0,1

TF auf Seehöhe vernachlässigbar
CF bei gesamter Kochmenge, sonst nur ein Teil gekocht, eliminiert BG

CA komplexer * YF *PF *Bf *FF

\parencite[140\psq]{Garetz1994} 
Schnelle flokulierung 5\% höhere utilisierung. Hopfenmenge *0,95
sonst * 1,05. Bei Hefeweizen * 0.8 weil an Hefezellen anhaftet.

von 10 bis kleiner 30 minuten Kochzeit Pelletfaktor, 0.9 10\% höher

Hopfensack
frei 10 * geringer * 1,1
eng 20 * geringer * 1,2
1,25 bis 2,25 geringer * 1.0125

HF Faktor bei Berechnung der IBU nicht vorhanden, schätzung :(

\parencite[134-144]{Garetz1994} 

IBU Messung von Labor durchführen lassen
\parencite[145]{Garetz1994} 

Berücksichtigt Ausbeute während Whirlpool nicht weil geringe
Isomerisierung für Heimbrauer.
\parencite[167]{Garetz1994} 

\subsubsection*{Mosher}

2 charts, für pellets und dolden, 3h 1,000-1120,nennt keine quellen
\parencite[160\psq]{Mosher1994}

\parencite{Mosher2022} Bereits ende der Achzigerjahre. Mechanischer
Rechner mit drehschreiben.

\parencite[51]{Holle2010}
Von Mosher bestimmte Werte finden sich hier

\parencite[9]{Thesseling2019}

Die Bitterausbeute ist basierend auf der Stammwürze für die entsprechende
Kochzeit zu bestimmen. Für Dolden ist dabei der Wert aus \autoref{table:mosherutil}
und für Pellets aus \autoref{table:mosherutilpellets} zu entnehmen.


\begin{table}[H]
\centering
\begin{tabular}{rrrrrrrr} 
\toprule
\multicolumn{1}{c}{\textbf{Kochzeit [min]}} & \multicolumn{7}{c}{\textbf{Ausbeute [\%]}}  \\
Relative Dichte [g/cm³]:                                        & 1,030 & 1,040 & 1,050 & 1,060 & 1,070 & 1,080  & 1,090                   \\
Stammwürze [°P]:                                            & 7,6 & 10 & 12,4 & 14,7 & 17,1 & 19,3  & 21,6                   \\                                            
\midrule

5                                            & 5     & 5     & 4     & 4     & 3     & 3      & 3                          \\
15                                           & 12    & 12    & 11    & 11    & 11    & 10     & 9                          \\
30                                           & 17    & 17    & 16    & 16    & 15    & 15     & 13                         \\
45                                           & 21    & 21    & 20    & 19    & 18    & 17     & 16                         \\
60                                           & 24    & 23    & 23    & 22    & 21    & 20     & 18                         \\
90                                           & 28    & 27    & 26    & 26    & 25    & 23     & 21                         \\
\bottomrule
\end{tabular}
\caption{Bitterausbeute für Dolden nach \citeauthor{Mosher1994} \parencite[51]{Holle2010}}
\label{table:mosherutil}
\end{table}

\begin{table}[H]
\centering
\begin{tabular}{rrrrrrrr} 
\toprule
\multicolumn{1}{c}{\textbf{Kochzeit [min]}} & \multicolumn{7}{c}{\textbf{Ausbeute [\%]}}  \\
Relative Dichte [g/cm³]:                                        & 1,030 & 1,040 & 1,050 & 1,060 & 1,070 & 1,080  & 1,090                   \\
Stammwürze [°P]:                                            & 7,6 & 10 & 12,4 & 14,7 & 17,1 & 19,3  & 21,6                   \\  
\midrule
5                                            & 6     & 6     & 5     & 5     & 4     & 4      & 3                          \\
15                                           & 15    & 15    & 14    & 14    & 13    & 13     & 11                         \\
30                                           & 22    & 21    & 21    & 20    & 19    & 18     & 16                         \\
45                                           & 26    & 26    & 25    & 24    & 23    & 22     & 21                         \\
60                                           & 29    & 28    & 28    & 27    & 26    & 25     & 23                         \\
90                                           & 35    & 34    & 33    & 32    & 31    & 29     & 27                         \\
\bottomrule
\end{tabular}
\caption{Bitterausbeute für Pellets nach \citeauthor{Mosher1994} \parencite[51]{Holle2010}}
\label{table:mosherutilpellets}
\end{table}

% TODO tabelle
\parencite[53]{Holle2010}
Spezifische Dichte und Volumen der gekühlten Ausschlagswürze

\subsubsection*{Tinseth}

\parencite{Tinseth1995}
\url{https://www.realbeer.com/hops/bcalc_js.html}

access to some handy tools and knowledgable friends at the USDA hop labs and the Flavor Perception labs at Oregon State University. Over the years I've read a lot of scientific papers on the subject and had quite a few worts and beers analyzed.

different than the curves attributed to Jackie Rager in Zymurgy or to Mark Garetz in his book. That's because I believe, backed up by the brewing literature, that alpha acid isomerization is a first order, or more likely, a pseudo first order chemical reaction

The "S" shaped curves most homebrewers are familiar with don't seem to accurately reflect what's happening in the brew kettle. What follows is a primer on how to calculate IBUs using my numbers.

IBUs = decimal alpha acid utilization * mg/l of added alpha acids

mg/l of added alpha acids = decimal AA rating * grams hops * 1000
                            -------------------------------------
                              volume of finished beer in liters
                              



The Bigness factor accounts for reduced utilization due to higher wort gravities. Use an average gravity value for the entire boil to account for changes in the wort volume.

\begin{equation}
\mathit{GF} = 1,65 \cdot 0,000125^{\left( \overline{d_{\frac{20}{20}}}\:[g/cm^3] - 1 \right)}
\label{eq:tinsethgf}
\end{equation}

\begin{equation}
\mathit{TF} = \frac{1 - e^{\left( -0,04 \cdot \mathit{Kochzeit}\:[min] \right)}}{4,15}
\label{eq:tinsethtf}
\end{equation}

\begin{equation}
\mathit{BA\:[\%]} = \mathit{GF} \cdot \mathit{TF} \cdot 100
\label{eq:tinsethutil}
\end{equation}

decimal alpha acid utilization = Bigness factor * Boil Time factor

The Boil Time factor accounts for the change in utilization due to boil time:


The numbers 1.65 and 0.000125 are empirically derived to fit my data. The number 0.04 controls the shape of the utilization vs. time curve. The factor 4.15 controls the maximum utilization value--make it smaller if your kettle utilization is higher than mine.

I'd suggest fiddling with 4.15 if necessary to match your system; only play with the other three if you like to muck around. I make no guarantees if you do.
                              

\subsubsection*{Daniels}

\parencite[80]{Daniels1996}
Setzt auf Berechnungsvorschrift von \citeauthor{Rager1990}
einfache

eigenen Tests und mehreren Quellen, insbesondere Randy Mosher
Hop-Go-Round

\begin{table}[H]
\centering
\begin{tabular}{rr}
\toprule
\multicolumn{1}{c}{\textbf{Kochzeit [min]}} & \multicolumn{1}{c}{\textbf{Ausbeute [\%]}} \\
\midrule
0–9            & 5  \\
10–19          & 12 \\
20–29          & 15 \\
30–44          & 19 \\
45–49          & 22 \\
60–74          & 24 \\
>74            & 27 \\
\bottomrule
\end{tabular}
\caption{Bitterausbeute für Dolden nach \citeauthor{Daniels1996} \parencite[80]{Daniels1996}}
\label{table:danielsutil}
\end{table}

\begin{table}[H]
\centering
\begin{tabular}{rr}
\toprule
\multicolumn{1}{c}{\textbf{Kochzeit [min]}} & \multicolumn{1}{c}{\textbf{Ausbeute [\%]}} \\
\midrule
0–9            & 6 \\
10–19          & 15 \\
20–29          & 19 \\
30–44          & 24 \\
45–49          & 27 \\
60–74          & 30 \\
>74            & 34 \\
\bottomrule
\end{tabular}
\caption{Bitterausbeute für Pellets nach \citeauthor{Daniels1996} \parencite[80]{Daniels1996}}
\label{table:danielsutilpellets}
\end{table}

\parencite[85]{Daniels1996}
IBU für 30 Minuten vermessen und dann eigene util zusammenstellen

\parencite[86]{Daniels1996}
Um mehrere Faktoren verweist auf den HCF nach Garetz

\subsubsection*{Noonan}

Die Bitterausbeute ist basierend auf der Stammwürze für die entsprechende
Kochzeit zu bestimmen. Für Dolden ist dabei der Wert aus \autoref{table:noonanutil}
und für Pellets aus \autoref{table:noonanutilpellets} zu entnehmen.
Die in Quelle fehlerhafte Bitterausbeute für Pellets bei 90~Minuten und einer
Stammwürze von 20,5 bis 23~°P wurde korrigiert.

\begin{table}[H]
\centering
\begin{tabular}{rrrrrr} 
\toprule
\multicolumn{1}{c}{\textbf{Kochzeit [min]}} & \multicolumn{5}{c}{\textbf{Ausbeute [\%]}}                                 \\
Relative Dichte [g/cm³]:                    & 1,032–50 & 1,050–65 & 1,065–75 & 1,075–85 & 1,085–95  \\
Stammwürze [°P]:                    & 8–12,5 & 12,5–16 & 16–18 & 18–20,5 & 20,5–23  \\
\midrule                                             
0                                            & 5        & 4        & 4                            & 4                            & 3                             \\
5                                            & 5        & 5        & 5                            & 4                            & 4                             \\
15                                           & 8        & 8        & 7                            & 7                            & 7                             \\
30                                           & 15       & 14       & 13                           & 13                           & 12                            \\
60                                           & 28       & 26       & 24                           & 23                           & 21                            \\
90                                           & 31       & 28       & 27                           & 26                           & 24                            \\
\bottomrule
\end{tabular}
\caption{Bitterausbeute für Dolden nach \citeauthor{Noonan1996} \parencite[215]{Noonan1996}}
\label{table:noonanutil}
\end{table}

\begin{table}[H]
\centering
\begin{tabular}{rrrrrr} 
\toprule
\multicolumn{1}{c}{\textbf{Kochzeit [min]}} & \multicolumn{5}{c}{\textbf{Ausbeute [\%]}}                                 \\
Relative Dichte [g/cm³]:                    & 1,032–50 & 1,050–65 & 1,065–75 & 1,075–85 & 1,085–95  \\
Stammwürze [°P]:                    & 8–12,5 & 12,5–16 & 16–18 & 18–20,5 & 20,5–23  \\
\midrule
0                                            & 5        & 5        & 4                            & 4                            & 3                             \\
5                                            & 6        & 6        & 5                            & 4                            & 4                             \\
15                                           & 12       & 12       & 10                           & 9                            & 8                             \\
30                                           & 18       & 17       & 16                           & 16                           & 15                            \\
60                                           & 31       & 28       & 26                           & 25                           & 23                            \\
90                                           & 33       & 30       & 28                           & 27                           & 25                     \\
\bottomrule
\end{tabular}
\caption{Bitterausbeute für Pellets nach \citeauthor{Noonan1996} \parencite[215]{Noonan1996}}
\label{table:noonanutilpellets}
\end{table}

\subsection*{Aktuelle Modelle}

\subsubsection*{Wolf}

https://www.maischemalzundmehr.de/index.php?inhaltmitte=toolsiburechner
Bereits seit 2012

\parencite{Wolf2022}

\subsubsection*{Hosom – mIBU}

\url{https://jphosom.github.io/alchemyoverlord}

\parencite{Hosom2015}

\subsubsection*{Novotný}

\url{https://www.diversity.beer/2018/02/ibu-spreadsheet-english-version.html}
\parencite{Novotny2016}
\parencite{Novotny2018}

\subsubsection*{Smith}

\parencite{Smith2019}

If we calculate the hop utilization for an equivalent boil hop using the time, volume and gravity of the wort for the addition, we can then apply the “whirlpool utilization” to this number to estimate the overall hop utilization. Here’s a look at the whirlpool utilization with temperature assuming 100% would be an equivalent boil hop:

   % Formula: Utilization = 2.39 * 10^11 * e^(-9773/T) where T is in Kelvin
    Boiling: 100 C (212 F) – Utilization is 100%
    At 90 C (194 F) – Utilization is 49%
    At 80 C (176 F) – Utilization is 23%
    At 70 C (158 F) – Utilization is 10%
    At 60 C (140 F) – Utilization is 4.3%
    At 50 C (122 F) – Utilization is 1.75%

\subsubsection*{Hosom – SMPH}

\parencite{Hosom2021}

\section*{Hopfenprodukte}


\parencite{Janish2019}
\parencite{Hieronymus2012}
\parencite{Nottebohm2020}
\parencite{Hall1997}
\parencite{Pyle1995}
\parencite{Justus2018}
\parencite{Parkin2017}
\parencite{Bishop1964}
\parencite{Nickerson1979}
\parencite{Calado2019}

\parencite{Bruecklmeier2017}
\parencite{Bruecklmeier2018}

Bieranalyse Fuchs crica 30 € \url{https://bieranalyse.de/}
Weinlabor Krauß: \url{https://www.weinlabor-krauss.de}

\printbibliography[title=Quellen]

\end{document}