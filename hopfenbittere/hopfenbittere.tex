% Copyright 2022 Thomas Ascher
% SPDX-License-Identifier: CC-BY-SA-4.0

\documentclass[a4paper,parskip=half]{scrartcl}

\usepackage[T1]{fontenc}
\usepackage[ngerman]{babel}
\usepackage{csquotes}
\usepackage{chemformula}
\usepackage[regular,condensed,sfdefault]{roboto}
\usepackage{booktabs}
\usepackage{graphicx}
\usepackage{chemformula}
\usepackage{amsmath,amsfonts,amssymb}
\usepackage{icomma}
\usepackage{textcomp}
\usepackage{gensymb}
\usepackage[italic,symbolgreek]{mathastext}
\usepackage{float}
\usepackage[style=apa,backend=biber]{biblatex}
\DeclareLanguageMapping{ngerman}{ngerman-apa}
\usepackage[hidelinks,pdfencoding=auto,
  pdfauthor={Thomas Ascher},
  pdfusetitle,
  pdfkeywords={Bier,Bitterung,Bitterausbeute,IBU,Tinsetz,Rager,Garetz}]{hyperref}
\usepackage{microtype}
\DisableLigatures{encoding=*, family=*}

\addto\extrasngerman{
\def\figureautorefname{Abb.}
\def\tableautorefname{Tab.}
\def\equationautorefname{Gl.}
}

\addto\captionsngerman{
\renewcommand{\figurename}{Abb.}
\renewcommand{\tablename}{Tab.}
}

\NewBibliographyString{gethesis}
\DefineBibliographyStrings{ngerman}{
  mathesis = {Masterarbeit},
  gethesis = {Diplomarbeit},
}

\title{Eine bittere Angelegenheit: Berechnungen rund um den Hopfen}
\author{Thomas Ascher <thomas.ascher@gmx.at>}
\date{1. Jänner 2022, \href{http://creativecommons.org/licenses/by-sa/4.0/}{CC BY-SA 4.0}}

\addbibresource{hopfenbittere.bib}

\begin{document}
\maketitle

\section*{Einleitung}

Meiste Bittere vom Hopfen, ein Teil von Röstmalzen \parencite[11]{Garetz1994}

\parencite[10]{Garetz1994}
Aroma, Bittere und Haltbarkeit

\parencite[103]{Garetz1994}
Brauer interesserit: Alpha, Beta und Ölgehalt.

\section*{Hopfenbestandteile}

\parencite[11]{Garetz1994}
Alphasäuren: humulone, cohumulone, adhumulone, sind Bittere und kommen in
verschiednen Hopfensorten in unterschiedlichen Verhältnissen vor
Glauben: Mit hohem Cohumulone Gehalt unangenehmere Bittere.
Beta Säuren, nicht Bitter? Ignoriert hinsichtlich Bitterpotential
Alphasäuren schlecht löslich in Würze bei geringeren Temperaturen
und normalem pH-Wert.

\parencite[20]{Garetz1994}
Weibliche Pflanze der Hopfengattung Humulus lupulus, Alpha Beta
Säuren teil des resins, Resin mit Hopfenölen in gelben
Lupulindrüsen in Dolden produziert.

\parencite[100]{Garetz1994} Einige Hopfenaromen entstehen erst
durch Oxidation. Besonders bei Nobelhopfen interessant.

\parencite[119]{Garetz1994} 
Adhumulone nur in kleinen Mengen.

\parencite[120]{Garetz1994} 
nur geringe Bittere, noch schlechter Wasserlöslich, mehrere Säuren,
lupulone, colupulone und adlupulone
Während lagerung und und kochen oxiieren und bilden Bitterstoffe

\section*{AAU, HBU}

\parencite[122\psq]{Garetz1994} 
ibu nicht messen
alpha mal hopfenmenge
Dave line
Abhängig von Sudmenge, beschreiben keine Verluste oder bittere
des finalzen biers.

\section*{Bitterausbeute}

\parencite[124]{Garetz1994} 

\% u = alpha present / alpha used * 100
efficency of alpha acid as iso alpha acid in final beer

Einflüsse:
Hopfenmenge
Hopenform
Kohtemperatur und Zeit
Volumen
Stammwürze
pH der Würze
Hopfengabe während Gärung
Yest Groth und floullation
Filtrierung

\parencite[125]{Garetz1994} 
Während des Kochens werden säuren und öle extrahiert 
bei dolden innerhalb von 10 bis 15 Minuten, bei Pallets
schneller.
Nach kochen fallen die unisomerisierten alpha säuren
aus, sonst spätestens bei Gärung

Isomerisierung abhängig von Volumen und Temperatur beim
Kochen, nach einem Zeitpunkt dreht sich die Reaktion
um. Nach circa 2 Stunden degraded. 

\parencite[126]{Garetz1994} 
Max. Isomerisierungsrate von 60-75, 8-10 trubverlust

iso absobiert onto Hefezellenwände, 5\% Schwankung auf
Utilisation basierend auf langsame und normale Flukulation
Hefewachstum

Auch mit CO2 ausgetrieben. In Hefedecke, macht bis zu 18\%
aus. Sammelt sich am Rand des Fermenters, verhärten durch
Oxidierung

\parencite[133]{Garetz1994} 
In Kräusen. J , höhere Stammwürze höherer Verlust, 

\parencite[128]{Garetz1994} 
Einige Korrekturfakten aus veröffentlichen Forschungsdaten. 
Messungen einiger gebrauten Biere.
Zu wenig Messdaten erhobens. Feedback erbeten.

10 Minuten zur Extraktion, 5-5.5 pH, vigorous boil

\parencite[130]{Garetz1994} 
Kochzeiten unter 30 min keine Pellet Korrektur
Bei pellets platzen die lupulindrüsen. 

Hopfensack ebenfalls einfluss,geringere utisilierung

\section*{Alphasäure und Isomerisierung}

\parencite[120\psq]{Garetz1994}
Bittere historisch beschrieben mit gegebener Hopfenmenge

internationaler Standrard IBU oder kurz BU, amerikaner und
europäer eigene Messmethoden
 
\parencite[121]{Garetz1994} 
IBU mg iso alpha in einem liter Bier, fertiges bier
ein Teil geht verloren. 

Messung: Spektrometer, mit ISO-Octane lösungsmittel, mixture
wird gemessen bei UV licht, genauer gesagt die abosption
bei gewissen Wellenlängen, absorption proportional
zu IBU (die wird korreliert)

\parencite[12]{Garetz1994}
Durch Kochen geht ein Teil der Alphasäuren in Lösung. Durch Kochen
chemischen Prozess der Isomerisierung. Iso-Alpha Säuren. Sehr löslich.
Bei Kochen geht AS schnell in Lösung, Isomerisierung dauert lange.
Ab 90 Min. Plateu erreicht.
Nur Ungefähr 20\% Alphasäure -> Utilisierung landen als Iso-Alpha im
fertigen Bier.
Verlust: zu kurze Isomerisierugszeit, fällt mit Kühltrub aus,
während Fermentierung und Filterung

\parencite[13]{Garetz1994}
Kettle Utilization, Boil Utilization während Kochen

\parencite[14]{Garetz1994} 
Bitte in IBU, mg of iaa in l Bier, Verluste während des gesamten
Prozesses

\parencite[34]{Garetz1994}
Alphasäure Bestandteil am Gewicht des ganzen Hopfens. Herangezogen
zur Bitterberechnung. Betasäure wird ignoriert.

\parencite[35]{Garetz1994}
Nobelhopfen Alpha:Beta = 1:1, nieder cohumulone.
mehr cohumulone harsch, gibt aber hopfen mit hohem cohumulone
Anteil, die nicht als Harsch bekannt sind. Ineratkion
mit anderem.

\parencite[128]{Garetz1994} 
Max. isomerisierung nach 90 Minuten, max 70\%, vermutlich fallender
ph-Wert reduziert isomerisierungsreaktion.

Höhere Hopfengaben ebenfalls. 40mg/l alpha 

\parencite[129]{Garetz1994} 
Gravity über 1.050 einfluss, Kochtemperatur (abhängig von Seehöhe).



\section{Hopfenprodukte}



\parencite[80\psq]{Garetz1994}
Doldenhopfen: am wenigsten verarbeitet, manche Brauer glauben
liefert das beste Aroma. Verbraucen viel Lagerplatz.

\parencite[82\psqq]{Garetz1994}
Pellets: pulverisiert und gepresst, verlust von 4 bis 6 \% Alpha
und Ölen.
T90
T45: enriched, teile des Pflanzenmaterials wird mechanisch
entfernt, doppelter Alpha.
Stabilisierte und isomerisierte Plellets, im Normalfall nicht
erhältlich. Ein Teil der Alphasäure in Iso Alphasäure
isomerisiert. Nicht für Dryhopping und Aromagaben.
chemisch verarbeitet, Magnesium hinzugefügt.
Pellets zerfallen zu Pulver
Schwerer zu entfernen, wird im Whilpool entfernt \parencite[87]{Garetz1994}

\parencite[84\psq]{Garetz1994}
Plugs und Bricks: 
Plug gepresster Doldenhopfen, in England gerne für Cask Dry Hopping
mehrere Dolden in Form, braucht einige Minuten bis sie auseinander
brechen. Schwer kleine Poritionen zu entnehmen.

\parencite[88-93]{Garetz1994}
Extrakte, mit mehreren Methoden hergestellt. Chemisch per Lösungsmittel
oder Dampfdestillation. Um an Alphasäure und Öle zu kommen.
Um Brauprozess berechenbarer zu machen und zu verinfachen.
iso-alpha extrakt: am Ende vom Prozess nach kochen oder
nach Gärung -> verluste, nicht sensitiv zu sonnenlicht, chemisch verändert
Hopfenöle: Aroma, Tasting panel hat schlechter gefunden wenn
late hops, wird in England für Casc conditioning, bessere konsistenz
Stark konzentriert, weniger für Anwendung im Heimbereich für
kleine Sude

\section*{Hopfengaben}

\parencite[15]{Garetz1994}
Spät: Aromaöle verdampschen schnell, verlust nach mehreren Minuten
oder gleich durch Kochen. Oder verändert, nicht mehr gleich
wie im Naturhopfen. 15 Minuten vor Ende. Wenig isomerisierung.

\section*{Effekte der Hopfenalterung}

Nördliche Hemisphere einmal Ernte pro Jahr. Mitte August bis
Anfang September.
\parencite[97]{Garetz1994}

\parencite[97]{Garetz1994} Hopfen verliert Alphasäure und Öle
durch Oxidation. Vakuimieren und gekühlte Aufbewahrung verlangsamen
den Prozess. Auch kein Licht. 

\parencite[103]{Garetz1994} Alphasäure oxidiert an luft, oxidationsprodukte
nicht mehr isomerisierbar. Nicht mehr so bitter. Betssäuren bilden
Bitterstoffe. bei Oxidation.

\parencite[104]{Garetz1994} 
Unter gleichen Lagerbedingungen verlieren Hopfen unterschiedlich an
Alpha wegen unterschiedliche Menge Oxidantien.
-HSI spektroskopisch analyse von alpha und beta. Test für Halbarkeit:
Verlust von Alpha und Betasäure über 6 Montate über Raumtemperatur 20c.
Korrelation zwischen beiden. Der ursprüngliche Alphagehalt und der
zkünftige
Ölverluste nicht ausreichend erforscht.

\parencite[104]{Garetz1994} 
HSI Nummer, nich von Händlern, hauptsächlich als interne Referenz
in den Laboren
Percent Alpha remaining or Lost after 6 Months at 20c
\% Alpha /nach 6 Monaten

\parencite[145-151]{Garetz1994}
Taste Titration 
Bier mit bekannter Bittere,, iso-alpha extrakt hinzufügen
bis gleich bitter wie test bier kalibrierter Extrakt, 0,44 IBU

% TODO ev. Berechnung beschreiben
% TODO Diagramm zu Alphaverlust

\section*{Modelle zur Schätung der Bitterausbeute}

\parencite[127]{Garetz1994} 
Keine in professioneller Brauliteratur, Messtechnisch erhoben
werden. Nach einigen Brauversuchen lässt sich also einstellen.
Blending mehrer Batches.

\subsection*{Rager}

\parencite[134]{Garetz1994} 
Nur ein Korrekturfaktor: boil Gravity, Ausbeutewerte sehr optimistisch
gewählt. 

\subsection*{Garetz}

g=v*ca*ibu desired/util*alpha*0,1

ibu=g*util*alpha*0,1/vol*CA

\parencite[134-144]{Garetz1994} 
Basiert auf Ragers Formel. Angepasst. Combined
Adjustments statt 1+GA.

Concentration factor
cf=fin vol/boil vol
boil gravity bg=(CF*(starting gravity -1))+1
gravity factor
GF=((BG-1.050/0.2)+1 = GA+1
Hoping rate factor
HF=((CF*desired ibu/260)+1
TF((eleation in feet/550)*0.02+1)
Seehöhe = 1. Negative nummer wenn unter Seehöhe

CA = GF*HF*TF

\begin{table}[H]
\centering
\begin{tabular}{rr}
\toprule
Kochzeit [min.] & Ausbeute [\%] \\
\midrule
0–10            & 0 \\
11–15           & 2 \\
16–20           & 5 \\
21–25           & 8 \\
26–30           & 11 \\
31–35           & 14 \\
36–40           & 16 \\
41–45           & 18 \\
46–50           & 19 \\
51–60           & 20 \\
61–70           & 21 \\
70–80           & 22 \\
81–90           & 23 \\
\bottomrule
\end{tabular}
\caption{Bitterausbeute nach Garetz \parencite[138]{Garetz1994}}
\label{table:garetzutil}
\end{table}


statt decimal 0,21 der Prozentzahl 
g = v in Liter * CA * IBU / util  alpha * 0,1

TF auf Seehöhe vernachlässigbar
CF bei gesamter Kochmenge, sonst nur ein Teil gekocht, eliminiert BG

CA komplexer * YF *PF *Bf *FF

\parencite[140\psq]{Garetz1994} 
Schnelle flokulierung 5\% höhere utilisierung. Hopfenmenge *0,95
sonst * 1,05. Bei Hefeweizen * 0.8 weil an Hefezellen anhaftet.

von 10 bis kleiner 30 minuten Kochzeit Pelletfaktor, 0.9 10\% höher

Hopfensack
frei 10 * geringer * 1,1
eng 20 * geringer * 1,2
1,25 bis 2,25 geringer * 1.0125

HF Faktor bei Berechnung der IBU nicht vorhanden, schätzung :(

\parencite[134-144]{Garetz1994} 

IBU Messung von Labor durchführen lassen
\parencite[145]{Garetz1994} 




\subsection*{Tinseth}

\url{https://www.realbeer.com/hops/bcalc_js.html}

\subsection*{Novotný}

\url{https://www.diversity.beer/2018/02/ibu-spreadsheet-english-version.html}

\section*{Hopfenprodukte}


\parencite{Annemueller2015}
\parencite{Beechum2017}
\parencite{Janish2019}
\parencite{Hieronymus2012}
\parencite{Nottebohm2020}
\parencite{Hall1997}
\parencite{Rager1990}
\parencite{Daniels1996}
\parencite{Mosher1994}
\parencite{Holle2010}
\parencite{Tinseth1995}
\parencite{Jones1995}
\parencite{Novotny2016}
\parencite{Brueckelmeier2018}

\printbibliography[title=Quellen]

\end{document}