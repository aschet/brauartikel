% Copyright 2021 Thomas Ascher
% SPDX-License-Identifier: CC-BY-SA-4.0

\documentclass[a4paper,parskip=half]{scrartcl}

\usepackage[T1]{fontenc}
\usepackage[ngerman]{babel}
\usepackage{csquotes}
\usepackage{chemformula}
\usepackage[regular,condensed,sfdefault]{roboto}
\usepackage{booktabs}
\usepackage{graphicx}
\usepackage{float}
\usepackage[style=apa,backend=biber]{biblatex}
\DeclareLanguageMapping{ngerman}{ngerman-apa}
\usepackage[hidelinks,pdfencoding=auto,
  pdfauthor={Thomas Ascher},
  pdfusetitle,
  pdfkeywords={Bier,Bierstil,Witbier}]{hyperref}
\usepackage{microtype}
\DisableLigatures{encoding=*, family=*}

\addto\extrasngerman{
\def\figureautorefname{Abb.}
\def\tableautorefname{Tab.}
\def\equationautorefname{Gl.}
}

\addto\captionsngerman{
\renewcommand{\figurename}{Abb.}
\renewcommand{\tablename}{Tab.}
}

\NewBibliographyString{gethesis}
\DefineBibliographyStrings{ngerman}{%
  mathesis = {Masterarbeit},
  gethesis = {Diplomarbeit},
}

\title{Stilportrait: Witbier / Biere blanché}
\author{Thomas Ascher <thomas.ascher@gmx.at>}
\date{29. Dezember 2021, \href{http://creativecommons.org/licenses/by-sa/4.0/}{CC BY-SA 4.0}}

\addbibresource{witbier.bib}

\begin{document}
\maketitle

\section*{Kurzbeschreibung}

Bei Witbier handelt es sich um ein traditionelles belgisches, mit Gewürzen
aromatisiertes, weißliches und naturtrübes obergäriges Bier auf Weizenbasis, das
üblicherweise sehr kalt getrunken wird. \parencite[1\psq]{Strottner1999}
Die Celis Brauerei empfiehlt eine Trinktemperatur von circa 3,5 bis 5,5~°C
und die Verwendung eines oktagonalen Witbier-Glases \parencite{CelisBrewery2021}.

Typische Vertreter dieses Bierstils sind \parencite{Roncoroni2018}:

\begin{itemize}
\item Brugs – Brouwerij De Gouden Boom
\item Celis White – Brouwerij Van Steenberge
\item Hoegaarden – Brouwerij De Kluis
\item Watou's Wit – Leroy Breweries
\end{itemize}

\section*{Sensorischer Eindruck}

Die BJCP und die Brewers Association Stil-Richtlinien beschreiben
Witbier als ein leicht säuerliches Bier mit hoher Karbonisierung,
leichtem bis mittlerem Körper, geringen Malznoten, geringer Bittere
und kein bis geringes Hopfenaroma. Darüber hinaus sind hintergründige
Zitrusnoten sowie fruchtige Ester und phenolische Hefearomen (Gewürznelke)
vorhanden. Das Erscheinungsbild ist charakterisiert durch eine helle
bis leicht goldene Farbe, eine durch Stärke und Hefe induzierte Trübung
und einem dichten Schaum. \parencites[48\psq]{BJCP2015}[24]{BA2021}

\section*{Stilparameter}

In \autoref{table:parameters} sind die vom \citeauthor{BJCP2015} und die von
\citeauthor{Strottner1999} genannten brautechnischen Parameter erfasst.
Die Angaben der Brewers Association Stil-Richtlinien sind annähernd
deckungsgleich \parencites[24]{BA2021}. Einige am Markt erhältliche
Witbiere weisen allerdings nur einen gemessenen Bittergehalt von
2 bis 5~IBU auf \parencite{Roncoroni2018}.

\begin{table}[H]
\centering
\begin{tabular}{lrr}
\toprule
Parameter                    & \citeauthor{BJCP2015} & \citeauthor{Strottner1999} \\
\midrule
Alkoholgehalt [\%v/v]        & 4,5–5,5 & 4–6,9 \\
Stammwürze [°P]              & 11–12,9 & 9,4–14,9 \\
Scheinbarer Restextrakt [\%w/w] & 2,1–3,1 & 1,5–3,2 \\
Scheinbarer Vergärungsgrad [\%] & –       & 74,9–86,7 \\
Bittere [IBU]                & 8–20    & 9–18 \\
Farbe [EBC]                  & 4–8     & 6–14 \\
\bottomrule
\end{tabular}
\caption{Stilparameter \parencites[49]{BJCP2015}{Strottner1999}}
\label{table:parameters}
\end{table}

Es existieren nur wenige Angaben hinsichtlich Karbonisierung.
\citeauthor{Zainasheff2007} nennt hierbei einen \ch{CO2} Gehalt von
4,9 bis 5,9~g/l bzw. 2,5 bis 3~vols \parencite{Zainasheff2007}.
Die BeerSmith Software empfiehlt sogar bis zu 6,7~g/l bzw. 3,4~vols.

\section*{Historische Entwicklung}

Der Ursprungsort des Witbiers ist das durch den Weizenanbau geprägte
Flämisch-Brabant im Zentrum von Belgien \parencite[44]{Roncoroni2018}.
Weitere Produktionsstandorte waren auch die Provinzen Hennegau und
Limburg \parencite[118]{Strottner1999}.
Belege dafür reichen zumindest bis in das 16. Jahrhundert
zurück, zu einer Zeit, als die kleine Stadt Hoegaarden größere
Teile des belgischen Biermarkts belieferte \parencite[46]{Mulder2020}. 
Durch gewährte Steuerprivilegien existierten dort zu Glanzzeiten 35
Brauereien auf nur 2000 Einwohner \parencite[27]{Sparrow2002}.

In Flämisch-Brabant wurde bereits ab dem Mittelalter Bier auf
Basis von Weizenrohfrucht gebraut. Damals aber noch mit
spontaner Gärung und Grut statt Hopfen zur Bitterung.
Besondere Bedeutung erlangten dabei die Witbiere aus Löwen
und aus dem 20 Kilometer entfernten Hoegaarden. In Löwen
entstand um 1425 eine Universität mit einer heute renommierten Brauschule
und im Jahr 1453 in Hoegaarden ein Kloster. Die dort ansässigen
Mönche haben später als Erste aromatisiertes Witbier mit
Curaçao-Orangen-Schalen und Koriander gebraut. Flämisch-Brabant war
früher Teil der Niederlande. Durch den während der
Kolonialzeit vorherrschenden Gewürzhandel standen
Brauereien deshalb neue exotische Zutaten zur Verfügung.
\parencite[1,4]{Strottner1999}

Während des 17. und 18. Jahrhunderts war Witbier das beliebteste Bier
in der Stadt Brüssel \parencite{Zainasheff2007}.
Mehrere Umstände haben dann aber schlussendlich dazu geführt, dass
es für fast ein Jahrzehnt gänzlich vom Markt verschwunden ist.
Nach der Französischen Revolution den Brauereien in
Flämisch-Brabant gewährte Steuerprivilegien wieder entzogen
\parencite[44]{Roncoroni2018}. Während des Ersten Weltkrieges
musste dann aufgrund von Weizenrationierungen die Produktion gänzlich
eingestellt werden. Nach beiden Weltkriegen hat sich
das Konsumverhalten bedingt durch den amerikanischen
Einfluss zugunsten von untergärigen Bieren verschoben \parencite[4]{Strottner1999}. Letztendlich schloss im Jahr 1957 die letzte
Witbier produzierende Brauerei Tomsin ihre Tore \parencite[44]{Roncoroni2018}.

Der in den Fünfzigerjahren in Hoegaarden ansässige Milchhändler Pierre
Celis hatte gelegentlich bei Brauvorgängen bei Tomsin ausgeholfen. Im
Jahr 1966 begann er nach Beratung mit dem ehemaligen
Braumeister von Tomsin und der Gründung der Brauerei Celis
(später De Kluis) wieder mit der Produktion von Witbier. Die von
Celis geschaffene und von historischen Vorlagen abweichende Interpretation
gilt heute als Prototyp dieses Bierstils.
\parencite[37,49]{Hieronymus2010} 

Ab den frühen Achtzigerjahren erfährt der Konsum von Witbier – es
existieren mittlerweile mehrere produzierende Brauereien – national und
international einen starken Aufschwung. Allerdings war dieser
ab den Neunzigerjahren in Belgien selbst wieder rückläufig.
Celis selbst sieht sich im Jahr 1985 nach einem Feuer im Malzhaus der Brauerei
De Kluis dazu gezwungen, Unternehmensanteile an Artois (heute
AB InBev) zu veräußern. Nach dem Verkauf seiner restlichen Anteile
emigriert er in die USA und gründet dort die Brauerei Celis in Austin,
Texas. Nach mehreren Besitzerwechseln und dem Tod von Celis in 2011
hat dessen Tochter die Brauerei 2017 in Austin wiedereröffnet.
\parencites[1]{Strottner1999}[37,49]{Hieronymus2010}{Meewes2017}

Die Herstellungsmethoden und Zusammensetzung von Witbieren haben
sich über die Zeit zum Teil stark verändert und den heutigen
Marktanforderungen gerecht zu werden. Insgesamt wurden diese
Biere primär lokal ausgeschenkt und waren nur wenige
Wochen aufgrund der zunehmenden Säurebildung genießbar.
\parencite[118]{Strottner1999}

Historische Brauprozesse bestanden bis zu einundzwanzig
Einzelschritten mit sechs Teilmaischen und dauerten bis zu
17 Stunden. Dabei wurde oft nur ein geringer Endvergärungsgrad
von circa 50~\% erreicht. Durch eine geringe Nutzung von gealtertem
Hopfen, kurzen Kochzeiten, um helle Biere zu produzieren 
und geringe Anstellraten bei der Gärung waren die Biere besonders
anfällig für Infektionen durch Milchsäurebakterien und andere Mikroben.
Deshalb erfolgte eine Auslieferung an Kunden teilweise direkt nach
der Hauptgärung. Errungenschaften wie die Verwendung von
Hefereinkulturen statt spontaner Gärung und der Einsatz von
Kälteanlagen nach dem Ersten Weltkrieg haben die Produktionsbedingungen
nachträglich verbessert. \parencite[38–41]{Hieronymus2010}

Eine detaillierte Vorgehensweise für traditionelle Herstellungsprozessen
ist bei \citeauthor{Hieronymus2010}, \citeauthor{Mulder2020} und
\citeauthor{Strottner1999} beschrieben.

\section*{Schüttungszusammenstellung}

Extraktlieferanten für Witbier sind helle Gerstenmalze wie Pilsner Malz,
helle Weizenmalze, Zucker und Rohfrucht der Getreidearten Weizen, Roggen,
Dinkel, Hafer und Buchweizen \parencite[14]{Strottner1999}. Im 19.
Jahrhundert war der Einsatz von luftgetrocknetem Grünmalz gängig
\parencite[38]{Hieronymus2010}. \autoref{table:grains} zeigt die von \citeauthor{Strottner1999}
ermittelten Schüttungsanteile. Moderne Witbiere haben einen Weizenanteil
von 30 bis 50~\% \parencite[45]{Mulder2020}. Historische
Schüttungen weisen einen Haferanteil von bis zu 15~\% auf, wobei
einen Großteil davon auch Hafermalz bilden kann \parencite[45]{Hieronymus2010}.
Der Einsatz von Hafer ist optional. Die Hoegaarden Schüttung
besteht z. B. aus 55~\% Gerstenmalz und 45~\% Weizenrohfrucht \parencite[43]{Strottner1999}.

\begin{table}[H]
\centering
\begin{tabular}{lr}
\toprule
Malz/Rohfrucht      & Anteil [\%] \\
\midrule
Gerstenmalz           & 50–60 \\
Weizenmalz            & 25–50 \\
Weizenrohfrucht       & 10–50 \\
Hafer (optional)      & 0,1–5 \\
Buchweizen (optional) & 0,1–5 \\
\bottomrule
\end{tabular}
\caption{Typische Schüttungsanteile \parencite[13]{Strottner1999}}
\label{table:grains}
\end{table}

Rohfrucht erfordert eine spezielle Behandlung bei der Maischeführung. Damit
die Stärke in der Rohfrucht für Enzyme zugänglich wird, muss diese zuerst,
abhängig von der Getreideart, für längere Zeit auf Verkleisterungstemperatur
gebracht werden. Es ist daher einfacher, den Rohfruchtanteil in der Schüttung
durch vorverkleisterte Weizen- und Haferflocken zu ersetzen.

% TODO Mosher, geschmackliche Veränderung

\section*{Maischeführung}

Die mehrstufige Infusion ist heute das meist verwendete Maischeverfahren
für Witbiere. \autoref{table:mashtable}
zeigt einen exemplarischen Rastverlauf. Die enthaltene
Eiweißrast dient zur Bildung von höhermolekularen Stickstoffsubstanzen die
Biertrübung begünstigten. \parencite[12,16]{Strottner1999}

\begin{table}[H]
\centering
\begin{tabular}{lrr}
\toprule
Rast              & Temperatur [°C] & Dauer [Min.] \\
\midrule
Einmaischen       & 38–56          & –     \\
Eiweißrast        & 50–56          & 10–35 \\
Maltoserast       & 60–64          & 20–60 \\
Verzuckerungsrast & 75              & –     \\
\bottomrule
\end{tabular}
\caption{Exemplarischer Rastverlauf \parencite[16]{Strottner1999}}
\label{table:mashtable}
\end{table}

Eine Zugabe von Rohfrucht erhört die Viskosität der Würze \parencite[9]{Strottner1999}.
Dadurch können Probleme beim Läutern entstehen, die durch Einsatz von
Reisspelzen gemindert werden können.

\section*{Hopfen und Hopfengaben}

\parencite[13]{Strottner1999}
Kochzeit60 bis 120 Minuten
Bitterstoffe10 bis 18 EBC

\parencite[1]{Strottner1999}
Als Hopfen wird meist
Aromahopfen in Dolden oder als Pellets verwendet.


\parencite[2]{Strottner1999}
- Würzekochung soll wegen des Erhalts des hohen Eiweißgehaltes kurz und schonend sein
- meist nur eine Hopfengabe; der Bitterstoffgehalt ist niedrig.

\parencite[17]{Strottner1999}
Bei der Blanche-Herstellung sollte die Kochung kurz und schonend sein. Eine
Kochdauer von z. B. einer Stunde bei 100°C und einem Würze-pH von etwa 5,2
würde den erwünschten hohen Eiweißgehalt erhalten.

\parencite[17]{Strottner1999}
Die Hopfen- und Gewürzegabe erfolgt etwa 15 bis 30 Minuten vor Kochende. Die
Blanche ist meist schwach gehopft, die Bittere liegt allgemein bei 10 bis 18 BE
(EBC)-Einheiten.

\parencite{Zainasheff2007}
BJCP style guide hints at low-hop flavor and aroma being acceptable, you’re better
off with neither. Hop flavor and aroma in this beer seems to battle with the other subtle spice notes.

\section*{Gewürze}

\parencite[26]{Strong2021}
- American versions being aggressive with the spicing

- Fresh coriander seeds have a lemony, slightly
earthy character. This citrusy complexity adds to the general
fruitiness.


\parencite[29]{Sparrow2002}
The coriander
must be ground or crushed if whole. Fresh
orange peel may be zested—the skin
removed from the rind—or added as whole
peels. They should be added within the final
10-15 minutes of the boil.

Author and homebrew spicemaster
Randy Mosher recommends the use of the
Indian variety of coriander, as it is softer,
more citrusy and much less celery/vegetal
than the ordinary kind. Look for it at Indian
groceries. I have found the dried orange peel
sold to homebrewers contributes an
unpleasant (and unwanted) bitterness to the
beer.

\parencite[29]{Sparrow2002}
currently used in the classic wit are 20 g/hl
(0.7 oz/hl or 0.13 oz/5 gallons.) of coriander
and 25 g/hl (0.9 oz/hl or 0.17 oz/5 gallons)

\parencite[48]{BJCP2015}
Coriander of certain origins
might give an inappropriate ham or celery character. 

\parencite[26]{Strong2021}
- Spices should be fresh; old or oxidized versions can have an unwanted celery, ham, or soapy

\parencite[13]{Strottner1999}
Korianderzugabe 10 bis 50 g/hl
Orangenschalenzugabe 13 bis 80 g/hl

Curaçao-Orangen oder Pomeranzen


\parencite[1]{Strottner1999}
Als Gewürze werden
Bitterorangenschalen,
Koriander,
Holunderblüten,
Süßholz
und
Muskatnuß

\parencite[2]{Strottner1999}
- Um das volle Aroma zu erhalten, fügt man die Gewürze oft erst kurz vor dem
Kochende zu.

\parencite[40]{Hieronymus2010}
- coriander seeds and orange peels (Curaçao)
- andere Gewürze finden kaum historische Erwähnung
- Coriander nur in geringen Mengen 0,02 grams per Liter

\parencite[73]{Hieronymus2010}
coriander provides citrusy aroma and flavor
to wit beer, while orange peel lends a balancing herbal bitterness.

Mash in at 124° F (51° C) for 15 minutes
Saccharification at 145° F (63° C) for 30 minutes
Saccharification at 154° F (68° C) for 15 minutes
Mash out at 169 to 172° F (76 to 78° C)

\parencite[17]{Strottner1999}
Die Hopfen- und Gewürzegabe erfolgt etwa 15 bis 30 Minuten vor Kochende.
Die Gewürze sollten nicht zu lange gekocht werden, weil dabei viel Aroma verloren
geht und gleichzeitig eine adstringierende Bittere entsteht. Verschiedene Brauer
maischen die Gewürze bei niedrigeren Temperaturen getrennt ein und fügen diesen
Auszug nach Kochende, z.B. im Whirlpool, der Würze zu

\parencite[18]{Strottner1999}
Man kann die Gewürze auch erst bei der Hauptgärung zugeben. Dies hat den Vorteil,
daß die leichtflüchtigen, ätherischen Öle und Aromastoffe besser erhalten bleiben.
Die Infektionsgefahr ist jedoch sehr groß (7).

\parencite[62]{Hieronymus2010}
Brouwerij Bavik, Gewürze bei Kochgbeginn, dafür größere Mengen, Infekte vermeiden

\parencite{Zainasheff2007}
, you can always bump it up by boiling some spices in a little water and adding them in, or adding dry
spices post fermentation.

Coriander is probably the trickiest of the witbier spices to balance properly. Not only does the spice intensity
vary considerably among suppliers and sources, but how you add it makes a big difference, too. I gently crush
the coriander with the back of a heavy spoon to expose the inside of the seeds, which gives it a fairly strong,
spicy character versus whole seeds. The level of coriander is probably the area most brewers overshoot,
resulting in a really peppery beer. The desired result is a gentle, background spicing, not an overwhelming one.
If you have fairly fresh coriander, start with 0.4 oz (11 g) per 5-gallon (19-L) batch added during the last five
minutes of the boil

\parencite[27]{Strong2021}
fferent types of citrus or orange va-
rieties, and also with coriander of dif-
ferent origin (Moroccan versus Indian).
These experiments are interesting, but
remember the classic profile as a ref-
erence. Curacao orange peel is always
dried, never fresh. If using fresh citrus
peel, take care to avoid any white pith.
There are several Belgian witbier
yeast strains available. Wyeast 3944
(Belgian Witbier), Wyeast 3942 (Belgian
Wheat), Wyeast 3463 (Forbidden Fruit),
White Labs WLP400 (Belgian Wit), and
White Labs WLP410 (Belgian Wit II)

\section*{Hefe}

Für die Gärung von Witbier kommen primär obergängige Hefestämme
zum Einsatz die zur Bildung des phenolischen Hefearomas
Gewürznelke (4-Vinyl Guaiacol) neigen \parencite[46]{Roncoroni2018}.
Hefen, die diesen Voraussetzungen entsprechen, sind
basierend auf Herstellerempfehlungen mit typischer Gärtemperatur,
in \autoref{table:yeasts} angeführt. Wyeast Hefen, die die
Bezeichnung PC bzw. „Private Collection“ in der Produktnummer enthalten,
werden nur gelegentlich aufgelegt. White Labs und Wyeast empfehlen
auch die Verwendung diverser Hefen für Saison- und Abteibiere,
die über ein passendes Aromaprofil verfügen.

\begin{table}[H]
\centering
\begin{tabular}{lllll}
\toprule
Hersteller      & Nummer  & Bezeichnung          & Form    & Gärtemp. [°C] \\
\midrule
Fermentis       & WB-06   & –                    & Trocken & 18–24        \\
Lallemand       & –       & WIT                  & Trocken & 17–22        \\
Mangrove Jack's & M21     & Belgian Wit          & Trocken & 18–25        \\
White Labs      & WLP400  & Belgian Wit          & Flüssig & 19–23        \\
White Labs      & WLP410  & Belgian Wit II       & Flüssig & 19–23        \\
Wyeast          & 3463 PC & Forbidden Fruit      & Flüssig & 17–25        \\
Wyeast          & 3942 PC & Belgian Wheat        & Flüssig & 18–23        \\
Wyeast          & 3944    & Belgian Witbier      & Flüssig & 17–25        \\
\bottomrule
\end{tabular}
\caption{Witbier Hefen verschiedener Hersteller (Ascher, 2021)}
\label{table:yeasts}
\end{table}

% TODO Hefestripping

\section*{Gärführung und Reifung}

\parencite[39]{Hieronymus2010}
- trübung hoher Proteingehalt von Weizen und Wintergerste.
 Durch Bakterien. Getrunken während noch Hefe in schwebe war,
 typisch Staubhefen. Kartoffelstärke um Effekt zu verstärken

\parencite[31]{Sparrow2002}
Primary fermentation at Hoegaarden is carried out between (18-25 C)

\parencite[13]{Strottner1999}
Gärtemperatur12 bis 25°C
Gärdauer3 bis 10 Tage
Zucker als Speise

\parencite[14]{Strottner1999}
Warmlagerungsphase
22 bis 25°C
7 bis 18Tage
Kaltlagerungsphase
0 bis 2°C
7 bis 21 Tage

Hefezugabe


\parencite[2]{Strottner1999}
- obergäriger Hefe vergoren
- Hauptgärung dauert 3 bis 6 Tage
- Nachgärung im Tank oder in der Flasche statt
- Einige Brauer fügen vor der Abfüllung dem Bier zur geschmacklichen Abrundung Milchsäure und Essigsäure zu.

\parencite[18]{Strottner1999}
Die auf etwa 20°C abgekühlte Würze wird, ohne Kühltrubabscheidung, mit
obergäriger Hefe bei einer Gärtemperatur zwischen 18 bis 25°C innerhalb von
3 bis 6 Tagen vergoren. 
Flaschengärung erzeugten, Blanche setzt sich die Hefe mit einem
Teil der Eiweißtrübung im Laufe der Zeit als unansehnlicher brauner, mehr oder
weniger fester Satz ab.

Nach einer Speisezugabe vermischt man das Bier vorzugsweise
mit einer Staubhefe und lagert die Flaschen dann 7 bis 14 Tage bei 22 bis 25°C für
die Nachgärung.

Um den Geschmack der alten traditionellen Blanche besser zu treffen, fügen einige
Brauer dem Bier vor der Abfüllung Milchsäure und Essigsäure zu.





\section*{Brauwasseraufbereitung}

Sowohl weiches (3 bis 8~°dH) als auch hartes Wasser (20~°dH) findet
Anwendung bei der Witbier-Herstellung. Während des Maischevorgangs
wird der pH-Wert aber zumeist auf einen Wert zwischen 5,2 und
5,4 mit Milchsäure oder Phosphorsäure eingestellt. \parencite[14]{Strottner1999}

\section*{Beispielrezept}

\section*{Produktion}

\parencite[52]{Hieronymus2010}
Hoegaarden White contains 40 percent unmalted wheat and 60 per-
cent malted barley, with the mash gradually heated from 122° F (50° C)
to 167° F (75° C). Lautering that used to take four to six hours now lasts
little more than two hours. After a one-hour boil brewers cool the wort
to 64° F (18° C) and pitch yeast, allowing fermentation to rise to 77°
F (25° C) over the course of five days.

\parencite[52-53]{Hieronymus2010}
Brouwerij Van Steenberge employs a similar recipe, but not the
same process, to brew Celis White. The brewery uses milled white
wheat (between 40 and 50 percent of the grist), otherwise intended for
baking, along with malted barley. The mash begins at 122° F (50° C)
and is gradually heated, but only to 149° F (65° C) for saccharifica-
tion. Hops are added with 20 minutes left in the boil, coriander and
Curaçao at the end. Fermentation starts at
60° F (16° C) and won’t be al-
lowed to rise above 64° F (18°
C). In contrast, strong Belgian
ales like Van Steenberge’s Pi-
raat may ferment as high at 79°
F (26° C). Primary usually lasts
eight to ten days. conditioning at about 41° F (5° C) takes four weeks.

\parencite[53-54]{Hieronymus2010}
CELIS WHITE (VAN STEENBERGE)
Malts: Pilsener, unmalted wheat
Spices: Coriander, orange peel
Hops: Saaz
Primary Fermentation: Yeast pitched at 61° F (16° C), rises to 61 to 64° F (16
to 18° C), 8 to 10 days
Secondary Fermentation: 41 to 43° F (5° to 6° C), 4 weeks
Also Noteworthy: Bottle conditioned
Celis White conditions at 75° F (24° C) for two weeks after bottling.

\parencite[56-58]{Hieronymus2010}
ALLAGASH WHITE
Hops: Perle, Tettnang, Saaz
The single-infusion mash ranges from 153° to 155° F (67° to 68° C)
depending on the quality of the malt. Hops are added 15 and 45 minutes
into the 90-minute boil, then in the whirlpool along with a spice bag
containing coriander, Curaçao orange peel, and a secret addition. “We
experimented with every way you could think of to add the spices,”
Perkins said
rimary Fermentation: Yeast pitched at 65° F (18° C), finishes at 68° F (20°
C), 7 days
Secondary Fermentation: Begins at 60° F (15.5° C) and drops to 50° F (10°
C), 4 to 7 days

Fermentation begins at 65° F (18° C) with a house strain used
primarily for the White, and the wort is aerated with a stone. Allagash



%% Strottner



Brewers Gold
Nugget

43 -> 22,5 Gramm pro Hektoliter fertiges Bier und 29 Gramm Pomeranzenschalen
55 Prozent Gerstenmalz und 45 Prozent
Die Würzekochung dauert 90 Minuten. Eine Stunde vor
Kochende findet die einzige Hopfengabe mit Nugget Hopfe
Hauptgärung dauert 4 bis 5 Tage bei 25°C.


Die Schüttung teilt sich auf 50 Anteile Gerstenmalz und 50 Anteile Weizenrohfrucht

Sorten Stirian und Goldings
49 -> 60 Prozent Gerstenmalz und 40 Prozent Weizenrohfrucht.
celis , 20 bis 22°C in liegenden Tanks
Die Hopfengabe
findet gleich am Anfang der Kochung statt.

Die Würzekochung dauert eine Stunde. Es werden zwei Hopfengaben gegeben: die
erste gleich nach Kochbeginn mit Hallertauer Pellets und die Zweite, mit Stirian
Pellets, zusammen mit der Gewürzgabe 15 Minuten vor Kochende.

20 Minuten vor Kochende werden Northern
Brewer Doldenhopfen

12 ° anstelltemperatur -> Die Würze wird bei 12°C mit obergäriger Hefe angestellt.

\printbibliography[title=Quellen]

\end{document}