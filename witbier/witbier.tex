% Copyright 2021 Thomas Ascher
% SPDX-License-Identifier: CC-BY-SA-4.0

\documentclass[a4paper,parskip=half]{scrartcl}

\usepackage[T1]{fontenc}
\usepackage[ngerman]{babel}
\usepackage{csquotes}
\usepackage{chemformula}
\usepackage[regular,condensed,sfdefault]{roboto}
\usepackage{booktabs}
\usepackage{graphicx}
\usepackage{float}
\usepackage[style=apa,backend=biber]{biblatex}
\DeclareLanguageMapping{ngerman}{ngerman-apa}
\usepackage[hidelinks,pdfencoding=auto,
  pdfauthor={Thomas Ascher},
  pdfusetitle,
  pdfkeywords={Bier,Bierstil,Witbier}]{hyperref}
\usepackage{microtype}

\addto\extrasngerman{
\def\figureautorefname{Abb.}
\def\tableautorefname{Tab.}
\def\equationautorefname{Gl.}
}

\addto\captionsngerman{
\renewcommand{\figurename}{Abb.}
\renewcommand{\tablename}{Tab.}
}

\NewBibliographyString{gethesis}
\DefineBibliographyStrings{ngerman}{%
  mathesis = {Masterarbeit},
  gethesis = {Diplomarbeit},
}

\title{Stilportrait: Witbier / Biere blanché}
\author{Thomas Ascher <thomas.ascher@gmx.at>}
\date{20. Oktober 2021, \href{http://creativecommons.org/licenses/by-sa/4.0/}{CC BY-SA 4.0}}

\addbibresource{witbier.bib}

\begin{document}
\maketitle

\section*{Kurzbeschreibung}

Bei Witbier handelt es sich um ein traditionelles belgisches, mit Gewürzen
aromatisiertes, weißliches und naturtrübes obergäriges Bier auf Weizenbasis, das
üblicherweise sehr kalt getrunken wird. \parencite[1\psq]{Strottner1999}
Die Celis Brauerei empfiehlt eine Trinktemperatur von circa 3,5 bis 5,5~°C
und die Verwendung eines oktagonalen Witbier-Glases \parencite{CelisBrewery2021}.

Typische Vertreter dieses Bierstils sind \parencite{Roncoroni2018}:

\begin{itemize}
\item Brugs -- Brouwerij De Gouden Boom
\item Celis White -- Brouwerij Van Steenberge
\item Hoegaarden -- Brouwerij De Kluis
\item Watou's Wit -- Leroy Breweries
\end{itemize}

\section*{Sensorischer Eindruck}

Die BJCP und die Brewers Association Stil-Richtlinien beschreiben
Witbier als ein leicht säuerliches Bier mit hoher Karbonisierung,
leichtem bis mittlerem Körper, geringen Malznoten, geringer Bittere
und kein bis geringes Hopfenaroma. Darüber hinaus sind hintergründige
Zitrusnoten sowie fruchtige Ester und phenolische Hefearomen (Gewürznelke)
vorhanden. Das Erscheinungsbild ist charakterisiert durch eine helle
bis leicht goldene Farbe, eine durch Stärke und Hefe induzierte Trübung
und einem dichten Schaum. \parencites[48\psq]{BJCP2015}[24]{BA2021}

\section*{Brautechnische Parameter}

In \autoref{table:parameters} sind die vom \citeauthor{BJCP2015} und die von
\citeauthor{Strottner1999} genannten brautechnischen Parameter erfasst.
Die Angaben der Brewers Association Stil-Richtlinien sind annähernd
deckungsgleich \parencites[24]{BA2021}. Einige am Markt erhältliche
Witbiere weisen allerdings nur einen gemessenen Bittergehalt von
2 bis 5~IBU auf \parencite{Roncoroni2018}.

\begin{table}[H]
\centering
\begin{tabular}{lrr}
\toprule
Parameter                    & \citeauthor{BJCP2015} & \citeauthor{Strottner1999} \\
\midrule
Alkoholgehalt [\%v/v]        & 4,5--5,5 & 4--6,9 \\
Stammwürze [°P]              & 11--12,9 & 9,4--14,9 \\
Scheinb. Restextrakt [\%w/w] & 2,1--3,1 & 1,8--3,2 \\
Scheinb. Vergärungsgrad [\%] & --       & 74,9--86,7 \\
Bittere [IBU]                & 8--20    & 6--18 \\
Farbe [EBC]                  & 4--8     & 5--17 \\
\bottomrule
\end{tabular}
\caption{Parameter \parencites[49]{BJCP2015}{Strottner1999}}
\label{table:parameters}
\end{table}

Es existieren nur wenige Angaben hinsichtlich Karbonisierung.
\citeauthor{Zainasheff2007} nennt hierbei einen \ch{CO2} Gehalt von
4,9 bis 5,9~g/l bzw. 2,5 bis 3~vols \parencite{Zainasheff2007}.
Die BeerSmith Software empfiehlt sogar bis zu 6,7~g/l bzw. 3,40~vols.

\section*{Historische Entwicklung}

Ursprungsort des Witbiers ist das durch den Weizenanbau geprägte
Flämisch-Brabant im Zentrum von Belgien \parencite[44]{Roncoroni2018}.
Weitere historische Produktionsstandorte waren auch die
Regionen Flandern, Hainaut und Limburg \parencite[118]{Strottner1999}.
Belege dafür reichen zumindest bis in das 16. Jahrhundert
zurück, zu einer Zeit als die kleine Stadt Hoegaarden größere
Teile des belgischen Biermarkts belieferte \parencite[46]{Mulder2020}. 
Durch gewährte Steuerprivilegien existierten alleine in
der Stadt Hoegaarden zu Glanzzeiten 35 Brauereien auf nur 2000
Einwohner \parencite[27]{Sparrow2002}.

In Flämisch-Brabant wurde bereits ab dem Mittelalter Bier auf
Basis von Weizenrohfrucht gebraut. Damals aber noch mit
spontaner Gärung und Grut statt Hopfen zur Bitterung.
Besondere Bedeutung erlangten dabei die Witbiere aus Löwen (Leuven)
und aus dem 20 Kilometer entfernten Hoegaarden. In Löwen
entstand um 1425 eine Universität mit einer heute renommierten Brauschule
und im Jahr 1453 in Hoegaarden ein Kloster. Die dort ansässigen
Mönche haben später als Erste aromatisiertes Witbier mit
Curaçao-Orangen-Schalen und Koriander gebraut. Flämisch-Brabant war
früher Teil der Niederlande. Durch den während der
Kolonialzeit vorangetriebenen Gewürzhandel standen
Brauereien deshalb neue exotische Zutaten zur Verfügung.
\parencite[1,4]{Strottner1999}

Im 17. und 18. Jahrhundert war Witbier der beliebteste Bier
in der Stadt Brüssel \parencite{Zainasheff2007}.
Mehrere Umstände haben dann aber schlussendlich dazu geführt, dass
es kurzzeitig gänzlich vom Markt verschwunden ist.
Zuerst wurde nach der französischen Revolution den Brauereien in
Flämisch-Brabant gewährte Steuerprivilegien wieder entzogen
\parencite[44]{Roncoroni2018}. Während des ersten Weltkrieges
musste dann aufgrund von Weizenrationierungen die Produktion gänzlich
eingestellt werden. Nach den beiden Weltkriegen hat sich
das Konsumverhalten, auch bedingt durch den amerikanischen
Einfluss, zugunsten von untergärigen Bieren verschoben \parencite[4]{Strottner1999}. Letztendlich schloss im Jahr 1957 die letzte
Witbier produzierende Brauerei Tomsin ihre Tore \parencite[44]{Roncoroni2018}.

Der in den fünfzigerjahren in Hoegaarden ansässige Milchhändler Pierre
Celis hatte gelegentlich bei Brauvorgängen bei Tomsin ausgeholfen. Im
Jahr 1966 begann er nach Beratung mit dem ehemaligen
Braumeister von Tomsin und der Gründung der Brauerei Celis
(später De Kluis) selbst mit der Produktion von Witbier. Die von
Celis geschaffene und von historischen Vorlagen abweichende Interpretation
gilt heute als Prototyp dieses Bierstils.
\parencite[37,49]{Hieronymus2010} 

Ab den frühen achtziger Jahren erfährt der Konsum von Witbier -- es
existieren mittlerweile mehrere produzierende Brauereien -- national und
international einen starken Aufschwung. Allerdings ist dieser
ab den neunziger Jahr in Belgien selbst wieder rückläufig.
Celis selbst sieht sich im Jahr 1985 nach einem Feuer im Malzhaus der Brauerei
De Kluis dazu gezwungen Unternehmensanteile an Artois (heute
AB InBev) zu veräußern. Nach dem Verkauf seiner restlichen Anteile
emigriert er in die USA und gründet dort die Brauerei Celis in Austin,
Texas. Nach mehreren Unternehmensveräußerungen und dem Tod von Celis in 2011
hat dessen Tochter die Brauerei 2017 wiedereröffnet. \parencites[1]{Strottner1999}[37,49]{Hieronymus2010}{Meewes2017}

Die Herstellungsmethoden und Zusammensetzung von Witbieren haben
sich über die Zeit zum Teil stark verändert und den heutigen
Marktgegebenheiten gerecht zu werden. Insgesamt wurden diese
Biere primär lokal ausgeschenkt und waren nur wenige
Wochen aufgrund der zunehmenden Säurebildung genießbar.
\parencite[118]{Strottner1999}

Historische Brauprozesse bestanden bis zu einundzwanzig
Einzelschritten mit sechs Teilmaischen und dauerten bis zu
17 Stunden an. Dabei wurde oft nur ein geringe Endvergärungsgrad
von circa 50 \% erreicht. Durch eine geringe Nutzung von
gealtertem Hopfen und kurzen Kochzeiten waren die
Biere besonders anfällig für Infektionen durch
Milchsäurebakterien und andere Mikroben. Deshalb erfolgte
eine Auslieferung an Kunden teilweise direkt nach
der Hauptgärung.
\parencite[38--41]{Hieronymus2010}




\parencite[38]{Hieronymus2010}
- typischerweise im sommer gebraut und konsumiert
- waren oft mit Lactobacillus und manchmal mit Pediococcus infiziert
- Hoegaarden sehr sauer im Gegensatz zum Leuven, Noten von saurer Milch
- 19 sehr hell gebraut mit Wind-Malz -> Luft Trocknung, nur kurz gekocht
  auch nur ein Teil der Würze, antibaktierller Schutz kaum gegeben,
  geringe Hopfengaben von gealtertem Hopfen

\parencite[39]{Hieronymus2010}
- trübung hoher Proteingehalt von Weizen und Wintergerste.
 Durch Bakterien. Getrunken während noch Hefe in schwebe war,
 typisch Staubhefen. Kartoffelstärke um Effekt zu verstärken

- komplizierte historische Brauprozesse bis zu 17 Stunden, geringer
  Endvergärungsgrad um die 50 \%



\parencite[9]{Strottner1999}
bedingt durch den Anstieg der Viskosität der Würze durch die
Zugabe von Weizenrohfrucht, versuchten die ursprünglichen Blanche-Hersteller mit
äußerst komplizierten Maischverfahren zu lösen.




\section*{Schüttung}

\parencite[45]{Mulder2020}
Nowadays, Belgian-style white beer usually contains 30 to 50 percent wheat.


\parencite[29]{Sparrow2002}
ording to the
Hoegaarden brewery, the grain bill is typi-
cally 55 percent malted barley and 45 per-
cent unmalted wheat.

\parencite[28]{Sparrow2002}
Wit was 50 percent malted barley, 40-45 percent unmalted wheat and 5-10 percent oats

\parencite{Zainasheff2007}
melanoidin rich malt like Munich it adds a slight bready note that is warmer in character
eep the amount to 5%

\parencite[46]{Roncoroni2018}
oats not uncommon
45 weizen or weizenmalz  
zugabe von hafer
Hefe und Stärke haze

\parencite[14]{Strottner1999}
Als Extraktlieferanten werden sehr helle Gerstenmalze, meist des Pilsener Typs,
helle Weizenmalze, Zuckersirup, sowie Rohfrucht der folgenden Getreidearten
benutzt : Weizen, Roggen, Dinkel, Hafer, Buchweizen (7).

\parencite[16]{Strottner1999}
- Einmaischen bei 54°C mit einer Rast von 20 Minuten.
- Die Temperaturstufe von 35°C wird weggelassen, um den beta-Glucan- und Eiweißabbau zu hemmen. 
- Eiweißrast begünstigt die Bildung von höhermolekularen Stickstoffsubstanzen.
  Die Trübung der Blanche wird neben der Hefe von einem hohen Gehalt an
  hochmolekularem Stickstoff gebildet.
- Die zweite Rast findet bei 64°C statt. Bei dieser Maltoserast wird der
   keine  unterteilte Rast
- Die Verzuckerungsrast liegt bei 75°C, wobei die alpha-Amylase die Stärke zu Dextrine


\parencite[13]{Strottner1999}
Schüttungsanteil :
Gerstenmalz50 bis 60 %
Weizenrohfrucht10 bis 50 %
Weizenmalz25 bis 50 %
Hafer / Buchweizen0,1 bis 5 %

Einmaischtemperatur38 bis 56°C
Intensität der Eiweißrast50 bis 56°C
10 bis 35 Minuten
Intensität der Maltoserast
60 bis 64°C
20 bis 60 Minuten
Maischdauer60 bis 150 Minuten

\parencite[12]{Strottner1999}
Bei der Blanche-Herstellung wird heute meist ein Infusionsmaischverfahren
angewendet.

\parencite[1]{Strottner1999}
Extraktlieferanten dienen sehr helle Gersten- und Weizenmalze, Zuckersirup,
Weizen, Roggen, Dinkel, Hafer und Buchweizen.

\parencite[40]{Hieronymus2010}
Also Known as Leuvens Witbier or Leuvensch Wit
Grains: Malted barley (45 to 55 percent), unmalted wheat (44 to 56
percent), unmalted oats (6 to 12 percent)

Ab zwanzistem Jahrhundert typische Zusammenstellung
Grist: 40 percent unmalted wheat, 50 percent malt, 10 percent unmalted
oats.
\parencite[43]{Hieronymus2010}

\parencite[45]{Hieronymus2010}
Bière de Hougaerde (or Hoegaerde)
Also Known As Hoegaardse Bier
Lacambre indicated that in 1851 the grist included 50 to 60 percent
wind-malted barley, germinated slowly, 20 percent unmalted wheat, and
10 to 15 percent unmalted oats. Other reports mention malted oats.

1933 Verlinden reported a grist we are more used to, with a balance be-
tween malted barley and wheat: 43 percent unmalted wheat, 44 percent
malted barley, 8 percent malted oats, and 4 percent unmalted oats.

Das Malz sollte
sehr hell sein, meist wurde das Grünmalz nur an der Luft getrocknet.

\parencite[46]{Hieronymus2010}
- Geringe Einmaischetemperatur 15 °C
- 42 für 20 min
- 55 für 45
- Kurze rasten bei 65 und 70
- bei 74 °C 45 min
- für eine stunde gekocht

- Nach Kühlung in Holzfässer, ein Warmen raum gelagert, oder offene Gärung zuerst

- Teilweise mit spontaner Gärung

- Hougaerde Bier: spontaner Gärung im, lactic bevor alkoholische Gärung

\parencite[11]{Strottner1999}
Diese Blanche wurde ähnlich hergestellt wie die Leuvener Blanche mit dem
wesentlichen Unterschied, daß man keine Hefe zufügte und die Gärung spontan im
Faß stattfand

\section*{Maischeführung}

\parencite[26]{Strong2021}
e Clerck described
cereal mashes and mixed mashing methods with wheat
boiled separately and multiple mash rests. Much of these
processes seem to be dealing with making sure the wheat is
properly gelatinized before it is converted, and then to avoid
scorching of the viscous mash.
Modern approaches can use pre-gelatinized flaked wheat
(and oats) to avoid the laborious mashing processes. Some
flavor could be compromised by using these ingredients; in
his book, Radical Brewing, Randy Mosher recommends per-
forming a cereal mash to better preserve that character. It’s
definitely a trade-off, but | consider it to be a very advanced
method. | wouldn't use it on my first attempt at the style;
better to understand the fermentation and spicing first. Most
commercial breweries today seem to be using a step mash
process.


\section*{Hopfen und Hopfengaben}

\parencite[13]{Strottner1999}
Kochzeit60 bis 120 Minuten
Bitterstoffe10 bis 18 EBC

\parencite[1]{Strottner1999}
Als Hopfen wird meist
Aromahopfen in Dolden oder als Pellets verwendet.


\parencite[2]{Strottner1999}
- Würzekochung soll wegen des Erhalts des hohen Eiweißgehaltes kurz und schonend sein
- meist nur eine Hopfengabe; der Bitterstoffgehalt ist niedrig.

\parencite[17]{Strottner1999}
Bei der Blanche-Herstellung sollte die Kochung kurz und schonend sein. Eine
Kochdauer von z.B. einer Stunde bei 100°C und einem Würze-pH von etwa 5,2
würde den erwünschten hohen Eiweißgehalt erhalten.

\parencite[17]{Strottner1999}
Die Hopfen- und Gewürzegabe erfolgt etwa 15 bis 30 Minuten vor Kochende. Die
Blanche ist meist schwach gehopft, die Bittere liegt allgemein bei 10 bis 18 BE
(EBC)-Einheiten.

\parencite{Zainasheff2007}
BJCP style guide hints at low-hop flavor and aroma being acceptable, you’re better
off with neither. Hop flavor and aroma in this beer seems to battle with the other subtle spice notes.



\section{Hefe, Gärführung und Reifung}

\parencite[31]{Sparrow2002}
Primary fermentation at Hoegaarden is carried out between (18-25 C)

\parencite[46]{Roncoroni2018}
Hefe strain clove like aroma 4-vinyl guaiacol, phenolisch off

\parencite[13]{Strottner1999}
Gärtemperatur12 bis 25°C
Gärdauer3 bis 10 Tage
Zucker als Speise

\parencite[14]{Strottner1999}
Warmlagerungsphase
22 bis 25°C
7 bis 18Tage
Kaltlagerungsphase
0 bis 2°C
7 bis 21 Tage

Hefezugabe

\begin{table}[]
\centering
\begin{tabular}{lllll}
\toprule
Hersteller      & Nummer  & Bezeichnung          & Form    & Gärtemp. [°C] \\
\midrule
Fermentis       & WB-06   & --                   & Trocken & 18--24        \\
Lallemand       & --      & WIT                  & Trocken & 17--22        \\
Mangrove Jack's & M21     & Belgian Wit          & Trocken & 18--25        \\
White Labs      & WLP400  & Belgian Wit          & Flüssig & 19--23        \\
White Labs      & WLP410  & Belgian Wit II       & Flüssig & 19--23        \\
Wyeast          & 3463 PC & Forbidden Fruit      & Flüssig & 17--25        \\
Wyeast          & 3942 PC & Belgian Wheat        & Flüssig & 18--23        \\
Wyeast          & 3944    & Belgian Witbier      & Flüssig & 17--25        \\
\bottomrule
\end{tabular}
\caption{Hefeempfehlungen verschiedener Hersteller (Ascher, 2021)}
\label{table:yeasts}
\end{table}

\parencite[2]{Strottner1999}
- obergäriger Hefe vergoren
- Hauptgärung dauert 3 bis 6 Tage
- Nachgärung im Tank oder in der Flasche statt
- Einige Brauer fügen vor der Abfüllung dem Bier zur geschmacklichen Abrundung Milchsäure und Essigsäure zu.

\parencite[18]{Strottner1999}
Die auf etwa 20°C abgekühlte Würze wird, ohne Kühltrubabscheidung, mit
obergäriger Hefe bei einer Gärtemperatur zwischen 18 bis 25°C innerhalb von
3 bis 6 Tagen vergoren. 
Flaschengärung erzeugten, Blanche setzt sich die Hefe mit einem
Teil der Eiweißtrübung im Laufe der Zeit als unansehnlicher brauner, mehr oder
weniger fester Satz ab.

Nach einer Speisezugabe vermischt man das Bier vorzugsweise
mit einer Staubhefe und lagert die Flaschen dann 7 bis 14 Tage bei 22 bis 25°C für
die Nachgärung.

Um den Geschmack der alten traditionellen Blanche besser zu treffen, fügen einige
Brauer dem Bier vor der Abfüllung Milchsäure und Essigsäure zu.



\section*{Gewürze}

\parencite[26]{Strong2021}
- American versions being aggressive with the spicing

- Fresh coriander seeds have a lemony, slightly
earthy character. This citrusy complexity adds to the general
fruitiness.


\parencite[29]{Sparrow2002}
The coriander
must be ground or crushed if whole. Fresh
orange peel may be zested—the skin
removed from the rind—or added as whole
peels. They should be added within the final
10-15 minutes of the boil.

Author and homebrew spicemaster
Randy Mosher recommends the use of the
Indian variety of coriander, as it is softer,
more citrusy and much less celery/vegetal
than the ordinary kind. Look for it at Indian
groceries. I have found the dried orange peel
sold to homebrewers contributes an
unpleasant (and unwanted) bitterness to the
beer.

\parencite[29]{Sparrow2002}
currently used in the classic wit are 20 g/hl
(0.7 oz/hl or 0.13 oz/5 gallons.) of coriander
and 25 g/hl (0.9 oz/hl or 0.17 oz/5 gallons)

\parencite[48]{BJCP2015}
Coriander of certain origins
might give an inappropriate ham or celery character. 

\parencite[26]{Strong2021}
- Spices should be fresh; old or oxidized versions can have an unwanted celery, ham, or soapy

\parencite[13]{Strottner1999}
Korianderzugabe 10 bis 50 g/hl
Orangenschalenzugabe 13 bis 80 g/hl

Curaçao-Orangen oder Pomeranzen


\parencite[1]{Strottner1999}
Als Gewürze werden
Bitterorangenschalen,
Koriander,
Holunderblüten,
Süßholz
und
Muskatnuß

\parencite[2]{Strottner1999}
- Um das volle Aroma zu erhalten, fügt man die Gewürze oft erst kurz vor dem
Kochende zu.

\parencite[40]{Hieronymus2010}
- coriander seeds and orange peels (Curaçao)
- andere Gewürze finden kaum historische Erwähnung
- Coriander nur in geringen Mengen 0,02 grams per Liter

\parencite[73]{Hieronymus2010}
coriander provides citrusy aroma and flavor
to wit beer, while orange peel lends a balancing herbal bitterness.

Mash in at 124° F (51° C) for 15 minutes
Saccharification at 145° F (63° C) for 30 minutes
Saccharification at 154° F (68° C) for 15 minutes
Mash out at 169 to 172° F (76 to 78° C)

\parencite[17]{Strottner1999}
Die Hopfen- und Gewürzegabe erfolgt etwa 15 bis 30 Minuten vor Kochende.
Die Gewürze sollten nicht zu lange gekocht werden, weil dabei viel Aroma verloren
geht und gleichzeitig eine adstringierende Bittere entsteht. Verschiedene Brauer
maischen die Gewürze bei niedrigeren Temperaturen getrennt ein und fügen diesen
Auszug nach Kochende, z.B. im Whirlpool, der Würze zu

\parencite[18]{Strottner1999}
Man kann die Gewürze auch erst bei der Hauptgärung zugeben. Dies hat den Vorteil,
daß die leichtflüchtigen, ätherischen Öle und Aromastoffe besser erhalten bleiben.
Die Infektionsgefahr ist jedoch sehr groß (7).

\parencite[62]{Hieronymus2010}
Brouwerij Bavik, Gewürze bei Kochgbeginn, dafür größere Mengen, Infekte vermeiden

\parencite{Zainasheff2007}
, you can always bump it up by boiling some spices in a little water and adding them in, or adding dry
spices post fermentation.

Coriander is probably the trickiest of the witbier spices to balance properly. Not only does the spice intensity
vary considerably among suppliers and sources, but how you add it makes a big difference, too. I gently crush
the coriander with the back of a heavy spoon to expose the inside of the seeds, which gives it a fairly strong,
spicy character versus whole seeds. The level of coriander is probably the area most brewers overshoot,
resulting in a really peppery beer. The desired result is a gentle, background spicing, not an overwhelming one.
If you have fairly fresh coriander, start with 0.4 oz (11 g) per 5-gallon (19-L) batch added during the last five
minutes of the boil

\parencite[27]{Strong2021}
fferent types of citrus or orange va-
rieties, and also with coriander of dif-
ferent origin (Moroccan versus Indian).
These experiments are interesting, but
remember the classic profile as a ref-
erence. Curacao orange peel is always
dried, never fresh. If using fresh citrus
peel, take care to avoid any white pith.
There are several Belgian witbier
yeast strains available. Wyeast 3944
(Belgian Witbier), Wyeast 3942 (Belgian
Wheat), Wyeast 3463 (Forbidden Fruit),
White Labs WLP400 (Belgian Wit), and
White Labs WLP410 (Belgian Wit II)

\section*{Brauwasseraufbereitung}

Sowohl weiches (3 bis 8~°dH) als auch hartes Wasser (20~°dH) findet
Anwendung bei der Witbier-Herstellung. Während des Maischevorgangs
wird der pH-Wert aber zumeist auf einen Wert zwischen 5,2 und
5,4 mit Milchsäure oder Phosphorsäure eingestellt. \parencite[14]{Strottner1999}

\section*{Beispielrezept}

\section*{Produktion}

\parencite[52]{Hieronymus2010}
Hoegaarden White contains 40 percent unmalted wheat and 60 per-
cent malted barley, with the mash gradually heated from 122° F (50° C)
to 167° F (75° C). Lautering that used to take four to six hours now lasts
little more than two hours. After a one-hour boil brewers cool the wort
to 64° F (18° C) and pitch yeast, allowing fermentation to rise to 77°
F (25° C) over the course of five days.

\parencite[52-53]{Hieronymus2010}
Brouwerij Van Steenberge employs a similar recipe, but not the
same process, to brew Celis White. The brewery uses milled white
wheat (between 40 and 50 percent of the grist), otherwise intended for
baking, along with malted barley. The mash begins at 122° F (50° C)
and is gradually heated, but only to 149° F (65° C) for saccharifica-
tion. Hops are added with 20 minutes left in the boil, coriander and
Curaçao at the end. Fermentation starts at
60° F (16° C) and won’t be al-
lowed to rise above 64° F (18°
C). In contrast, strong Belgian
ales like Van Steenberge’s Pi-
raat may ferment as high at 79°
F (26° C). Primary usually lasts
eight to ten days. conditioning at about 41° F (5° C) takes four weeks.

\parencite[53-54]{Hieronymus2010}
CELIS WHITE (VAN STEENBERGE)
Malts: Pilsener, unmalted wheat
Spices: Coriander, orange peel
Hops: Saaz
Primary Fermentation: Yeast pitched at 61° F (16° C), rises to 61 to 64° F (16
to 18° C), 8 to 10 days
Secondary Fermentation: 41 to 43° F (5° to 6° C), 4 weeks
Also Noteworthy: Bottle conditioned
Celis White conditions at 75° F (24° C) for two weeks after bottling.

\parencite[56-58]{Hieronymus2010}
ALLAGASH WHITE
Hops: Perle, Tettnang, Saaz
The single-infusion mash ranges from 153° to 155° F (67° to 68° C)
depending on the quality of the malt. Hops are added 15 and 45 minutes
into the 90-minute boil, then in the whirlpool along with a spice bag
containing coriander, Curaçao orange peel, and a secret addition. “We
experimented with every way you could think of to add the spices,”
Perkins said
rimary Fermentation: Yeast pitched at 65° F (18° C), finishes at 68° F (20°
C), 7 days
Secondary Fermentation: Begins at 60° F (15.5° C) and drops to 50° F (10°
C), 4 to 7 days

Fermentation begins at 65° F (18° C) with a house strain used
primarily for the White, and the wort is aerated with a stone. Allagash



%% Strottner



Brewers Gold
Nugget

43 -> 22,5 Gramm pro Hektoliter fertiges Bier und 29 Gramm Pomeranzenschalen
55 Prozent Gerstenmalz und 45 Prozent
Die Würzekochung dauert 90 Minuten. Eine Stunde vor
Kochende findet die einzige Hopfengabe mit Nugget Hopfe
Hauptgärung dauert 4 bis 5 Tage bei 25°C.


Die Schüttung teilt sich auf 50 Anteile Gerstenmalz und 50 Anteile Weizenrohfrucht

Sorten Stirian und Goldings
49 -> 60 Prozent Gerstenmalz und 40 Prozent Weizenrohfrucht.
celis , 20 bis 22°C in liegenden Tanks
Die Hopfengabe
findet gleich am Anfang der Kochung statt.

Die Würzekochung dauert eine Stunde. Es werden zwei Hopfengaben gegeben: die
erste gleich nach Kochbeginn mit Hallertauer Pellets und die Zweite, mit Stirian
Pellets, zusammen mit der Gewürzgabe 15 Minuten vor Kochende.

20 Minuten vor Kochende werden Northern
Brewer Doldenhopfen

12 ° anstelltemperatur -> Die Würze wird bei 12°C mit obergäriger Hefe angestellt.

\printbibliography[title=Quellen]

\end{document}