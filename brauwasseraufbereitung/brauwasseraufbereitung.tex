% !TeX program = pdflatex
% !TEX encoding = UTF-8 Unicode
% !TEX spellcheck = de_DE
% !BIB program = biber

\documentclass[10pt,a4paper,DIV=12,parskip=half]{scrarticle}
\usepackage[T1]{fontenc}
\usepackage[ngerman]{babel}
\usepackage{csquotes}
\usepackage{mathpazo}
\usepackage{amsmath}
\usepackage[output-decimal-marker={,}]{siunitx}
\usepackage{gensymb}
\usepackage{graphicx}
\usepackage{float}
\usepackage[section]{placeins}
\usepackage[style=apa,backend=biber]{biblatex}

\usepackage[hidelinks,pdfencoding=auto,
pdfauthor={Thomas Ascher},
pdfusetitle,
pdfkeywords={Bier,Farbe,sRGB,Simulation,EBC}]{hyperref}
\usepackage{microtype}

\setkomafont{disposition}{\normalfont\bfseries}

\addto\extrasngerman{
	\def\figureautorefname{Abb.}
	\def\tableautorefname{Tab.}
	\def\equationautorefname{Gl.}
}

\addto\captionsngerman{
	\renewcommand{\figurename}{Abb.}
	\renewcommand{\tablename}{Tab.}
}

\title{Brauwasseraufbereitung im Heimbereich}
\author{Thomas Ascher <thomas.ascher@gmx.at>}
\date{\today, \href{http://creativecommons.org/licenses/by-sa/4.0/}{CC BY-SA 4.0}}

\addbibresource{brauwasseraufbereitung.bib}

\begin{document}
\maketitle

\section*{Einleitung}

Wasser ist Hauptbestandteil von Bier und an allen Teilen der Bierherstellung beteiligt, angefangen beim Mälzen. Maische, Kochen, Gärung, Reinigung, Kühlung. \autocite[3]{EBC2001}.

Biersorten aufgrund von Wasserzusammensetzung entstanden: Burton-on-Trent, München, Pilsen, Dortmund. \autocite[3]{EBC2001}.

Verunreinigungen, Minerale Zusammensetzung wichtig. \autocite[3]{EBC2001}.

Good practice: 4,6 hl pro hl Bier \autocite[5]{EBC2001}.

\section*{Brauwasseranforderungen}

microbiologisch, organisch und inorganische anteile für potable water. \autocite[7]{EBC2001}.

Mineralgehalt macht important contribution for sensory profile Hat sich in der Vergangenheit stark auf die regional entwickelten Bierstile ausgewirkt. \autocite[7]{EBC2001}.

Lager: weiches mineralarmes wasser , wenig carbonate -> pH reduktion notwendig, calcium ionen weniger wichtig bei hellem malz \autocite[9]
Pale ales and bitters: bicarbonate level < 60 mg/liter calcium > 125 mg/liter Sulphate > chloride um bittere aromen zu unterstützen.

Milds and stouts: carbonates < 60mg/l calcium weniger, 75mg/l for milds, <30mg/l stouts

\section*{Effekte von Ionen bei der Bierherstellung}

Ionenkonzentrationen haben Auswirkungen effekt auf Brewing performance und Bierqualität. 
\autocite[9]{EBC2001}

Hydrogen und hydroxyl, H+ OH- \autocite[9]{EBC2001}:
Immer präsent, pH-Wert ist bestimmt durch das Verhältnis dieser Ionen. Reaktionen sind pH-Wert abhängig. Geringer pH-Wert:
Mehr total soluble nitrogen and free amino nitrogen
Increased extract yield
Increased wort run-off rate
Decreased extraction of tannins and hop bittering compounds
increased fermentability.
Empfohlen zwischen 6.0 und 7.5.

Calcium Ca²+ \autocite[10]{EBC2001}: Spielt wichtige Rolle bei Wasserhärte (temporär und permanent), Ca


\printbibliography[title=Quellen]

\end{document}

