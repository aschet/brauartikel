% Copyright 2022 Thomas Ascher
% SPDX-License-Identifier: CC-BY-SA-4.0

\documentclass[a4paper,parskip=half]{scrartcl}

\usepackage[T1]{fontenc}
\usepackage{mathpazo}
\usepackage[naustrian]{babel}
\usepackage{csquotes}
\usepackage{booktabs}
\usepackage{graphicx}
\usepackage{chemformula}
\usepackage{icomma}
\usepackage{gensymb}
\usepackage{float}
\usepackage[section]{placeins}
\usepackage[style=apa,backend=biber]{biblatex}

\usepackage[hidelinks,pdfencoding=auto,
  pdfauthor={Thomas Ascher},
  pdfusetitle,
  pdfkeywords={Bier,Farbe,SRM,IBU,EBC,Morey}]{hyperref}
\usepackage{microtype}

\setkomafont{disposition}{\normalfont\bfseries}

\addto\extrasnaustrian{
\def\figureautorefname{Abb.}
\def\tableautorefname{Tab.}
\def\equationautorefname{Gl.}
}

\addto\captionsnaustrian{
\renewcommand{\figurename}{Abb.}
\renewcommand{\tablename}{Tab.}
}

\NewBibliographyString{gethesis}
\DefineBibliographyStrings{naustrian}{
  mathesis = {Masterarbeit},
  gethesis = {Diplomarbeit},
}

\newcommand{\BA}{\mathit{BA}}
\newcommand{\umin}{\:[\textrm{min}]}
\newcommand{\uden}{\:[\text{g/cm³}]}
\newcommand{\uper}{\:[\text{\%}]}
\newcommand{\uli}{\:[\text{l}]}
\newcommand{\ume}{\:[\text{m}]}
\newcommand{\ucon}{\:[\text{mg/l}]}
\newcommand{\FKd}{F_{\mathit{d}}}

\title{Es wird bunt: Bierfarbe aus Schüttung berechnen}
\author{Thomas Ascher <thomas.ascher@gmx.at>}
\date{\today, \href{http://creativecommons.org/licenses/by-sa/4.0/}{CC BY-SA 4.0}}

\addbibresource{bierfarbe.bib}

\begin{document}
\maketitle

\section*{Einleitung}

„\href{https://bieranalyse.de}{Bieranalyse Fuchs}“

Die Einstellung des durch Hopfen eingebrachten Bitterstoffgehalts eines Biers, den wir als IBU-Angabe kennen, ist ein wesentlicher Bestandteil des Brauprozesses und der Rezeptentwicklung. Typischerweise helfen uns heute Braurechner bei Bestimmung von benötigten Hopfenmengen. Aber wie funktionieren deren Berechnungen? Dieser Artikel soll darüber Aufschluss geben und stellt historische Berechnungsverfahren aus dem Heimbraubereich vor.


\begin{table}[H]
\centering
\begin{tabular}{rrr}
\toprule
\multicolumn{1}{c}{\textbf{Beschreibung}} & \multicolumn{1}{c}{\textbf{EBC}} & \multicolumn{1}{c}{\textbf{SRM}} \\
\midrule
Stroh & - & 2-3 \\
Gelb & - & 3-4 \\
Gold & - & 5-6 \\
Berstein & - & 6-9 \\
Dunkles Bernstein / Helles Kupfer & - & 10-14 \\
Kupfer & - & 14-17 \\
Dunkles Kupfer / Helles Braun & - & 17-18 \\
Braun & - & 19-22 \\
Dunkles Braun & - & 22-30 \\
Sehr dunkles Braun & - & 30-35 \\
Schwarz & - & 30+ \\
Blickdichtes Schwarz & - & 40+ \\
\bottomrule
\end{tabular}
\caption{TODO \parencite{BJCP2015}}
\label{table:bjcpcolor}
\end{table}



\parencite[11]{Noonan1996}
Color may not be stated as SRM/ASBC/°L units.
Where EBC units are given, the formula ASBC = (°EBC
+ 1.2)/2.65 gives reasonable but not entirely accurate
transposition. The color for English malts is often given as
Color IOB, EBC method. IOB color is only approximately
80 percent of EBC color. Where IOB units are given, the
formula ASBC = [(IOB/.80) + 1.2]/2.65 can be used.
HCU color units are a summary measurement of
wort color that is equal to the sum of the SRM/°Lovibond
of all the malts used to make a wort. HCU units and SRM
wort color are comparable up to about 10 °SRM; after that,
corrections must be made. See the conversion chart in
table 15 to convert from HCU color to approximate °SRM.
The color ranges for most malt types vary widely,
depending on the country of origin and the maltster. From
the maltster’s point of view, it is of some advantage to
have a “unique” color range, since it requires any brewer
using the malt to make serious adjustments in the brew
house program in order to switch suppliers. To some
extent, this diversity also benefits brewers, but on the
whole, color variation in the same nominal malt types is a
problem for the brewer seeking predictable results.


\begin{table}[H]
\centering
\begin{tabular}{rr}
\toprule
\multicolumn{1}{c}{\textbf{SRM}} & \multicolumn{1}{c}{\textbf{MCU}} \\
\midrule
1–10 & 1-10 \\
10,5 & 10,8 \\
11   & 11,6 \\
11,5 & 12,4 \\
12   & 13,3 \\
12,5 & 14,1 \\
13   & 14,9 \\
13,5 & 17,7 \\
14   & 18,6 \\
14,5 & 20,5 \\
15   & 22,4 \\
15,5 & 24,3 \\
16   & 26,2 \\
16,5 & 28,1 \\
17   & 30 \\
17,5 & 32,9 \\
18   & 35,8 \\
18,5 & 38,8 \\
19   & 41,9 \\
19,5 & 45 \\
20   & 47,8 \\
\bottomrule
\end{tabular}
\caption{TODO \parencite[206]{Noonan1996}}
\label{table:noonanmcu}
\end{table}



\section*{Zusammenfassung}

\parencite{BJCP2015}
\parencite{Noonan1996}
\parencite{Bruecklmeier2018}
\parencite{Holle2010}
\parencite{Mosher1994}
\parencite{Daniels1996}


Die relevanten Eckpunkte dieses Artikels sind:

\begin{itemize}
\item TODO
\end{itemize}

\printbibliography[title=Quellen]

\end{document}