% Copyright 2022 Thomas Ascher
% SPDX-License-Identifier: CC-BY-SA-4.0

\documentclass[a4paper,parskip=half]{scrartcl}

\usepackage[T1]{fontenc}
\usepackage{mathpazo}
\usepackage[naustrian]{babel}
\usepackage{csquotes}
\usepackage{booktabs}
\usepackage{graphicx}
\usepackage{chemformula}
\usepackage{icomma}
\usepackage{gensymb}
\usepackage{float}
\usepackage[section]{placeins}
\usepackage[style=apa,backend=biber]{biblatex}

\usepackage[hidelinks,pdfencoding=auto,
  pdfauthor={Thomas Ascher},
  pdfusetitle,
  pdfkeywords={Bier,Farbe,SRM,IBU,EBC,Morey}]{hyperref}
\usepackage{microtype}

\setkomafont{disposition}{\normalfont\bfseries}

\addto\extrasnaustrian{
\def\figureautorefname{Abb.}
\def\tableautorefname{Tab.}
\def\equationautorefname{Gl.}
}

\addto\captionsnaustrian{
\renewcommand{\figurename}{Abb.}
\renewcommand{\tablename}{Tab.}
}

\NewBibliographyString{gethesis}
\DefineBibliographyStrings{naustrian}{
  mathesis = {Masterarbeit},
  gethesis = {Diplomarbeit},
}

\newcommand{\MCU}{\mathit{MCU}}
\newcommand{\SRM}{\mathit{SRM}}
\newcommand{\umin}{\:[\textrm{min}]}
\newcommand{\uden}{\:[\text{g/cm³}]}
\newcommand{\uper}{\:[\text{\%}]}
\newcommand{\uli}{\:[\text{l}]}
\newcommand{\ume}{\:[\text{m}]}
\newcommand{\ucon}{\:[\text{mg/l}]}
\newcommand{\FKd}{F_{\mathit{d}}}

\title{Es wird bunt: Bierfarbe aus der Schüttung berechnen}
\author{Thomas Ascher <thomas.ascher@gmx.at>}
\date{\today, \href{http://creativecommons.org/licenses/by-sa/4.0/}{CC BY-SA 4.0}}

\addbibresource{bierfarbe.bib}

\begin{document}
\maketitle

\section*{Einleitung}

„\href{https://bieranalyse.de}{Bieranalyse Fuchs}“

%In \autoref{table:parameters} sind die vom \citeauthor{BJCP2015} und die von
%\citeauthor{Strottner1999} genannten brautechnischen Parameter erfasst.
%Die Angaben der Brewers Association Stil-Richtlinien sind annähernd
%deckungsgleich \parencites[24]{BA2021}. 

\begin{table}[H]
\centering
\begin{tabular}{lrr}
\toprule
\multicolumn{1}{c}{\textbf{Farbbeschreibung}} & \multicolumn{1}{c}{\textbf{SRM}} & \multicolumn{1}{c}{\textbf{EBC}} \\
\midrule
Sehr hell & 1–1,5 & 2–3  \\
Stroh & 2–3 & 4–6 \\
Blass & 4 & 8 \\
Gold & 5–6 & 10–12 \\
Helles Bernstein & 7 & 14 \\
Bernstein & 8 & 16 \\
Dunkles Bernstein & 9 & 18 \\
Kupfer/Granatrot & 10–12 & 20–24 \\
Hellbraun & 13–15 & 26–30 \\
Braun/Rotbraun/Kastanienbraun & 16–17 & 32–33 \\
Dunkelbraun & 18–24 & 35–47 \\
Sehr dunkel & 25–39 & 49–77 \\
Schwarz & 40+ & 79 \\
\bottomrule
\end{tabular}
\caption{Farbeindruck TODO \parencite{BA2021}}
\label{table:bacolor}
\end{table}

\parencite{BA2021}


\parencite[11]{Noonan1996}
Color may not be stated as SRM/ASBC/°L units.
Where EBC units are given, the formula ASBC = (°EBC
+ 1.2)/2.65 gives reasonable but not entirely accurate
transposition. The color for English malts is often given as
Color IOB, EBC method. IOB color is only approximately
80 percent of EBC color. Where IOB units are given, the
formula ASBC = [(IOB/.80) + 1.2]/2.65 can be used.
HCU color units are a summary measurement of
wort color that is equal to the sum of the SRM/°Lovibond
of all the malts used to make a wort. HCU units and SRM
wort color are comparable up to about 10 °SRM; after that,
corrections must be made. See the conversion chart in
table 15 to convert from HCU color to approximate °SRM.
The color ranges for most malt types vary widely,
depending on the country of origin and the maltster. From
the maltster’s point of view, it is of some advantage to
have a “unique” color range, since it requires any brewer
using the malt to make serious adjustments in the brew
house program in order to switch suppliers. To some
extent, this diversity also benefits brewers, but on the
whole, color variation in the same nominal malt types is a
problem for the brewer seeking predictable results.

\section*{Messung von Bier- und Malzfarbe}

\parencite{KrausWeyermann2021a}
Erster systematischer Ansatz
Josep W. Lovibon 1885 kolorimetischen Komperator, Farbe Bier in Glas bewerten
Beobachtung von Lichtabsorption zwischen Probe und standertisierten
gefärben Glassch3eiben. Lovibond--Skala °L (2-bis 70)
Subjektive Wahrnehmung und Umwelteinflüsse haben Auswirkung auf Messwert.
Kolorimeter

\parencite{KrausWeyermann2021a}
1951 ASBC Spektrophotometer. Die Intensität eines Lichstrahls vor und nach
dem passieren einer Flüssigkeit. indirekt Messung. Wird Farbe zugeschrieben.
Wellenlänge 430 nm, Quarzglasküvette 1cm. Gefilterte Würze oder Bierprobe.
1 srm = 12,7-FACHE lOGARITHMUS DER aBSORPTION für grobe Äquivalenz mit Lovibond
Skala.

\parencite{KrausWeyermann2021a}
SRM=L+0,04662*(L) hoch 2
L= 0,808*SRM-0,0083 für Helle
Kongressmaische, standartisierte Maischeprogramm
EBC 1996 EBC Skala 25-fache log. 1EBC = srm * 1,97,
3,6-25,3 zuverlässig. 
Dunkle Flüssigkeit absorbiert Viel. Keine zuverlässige Messung ab 80-100 EBC
80 EBC Wahrnehmungsschlelle. Dunkler als 80 wird nicht mehr vom Auge wahrgenommen.
Mit wasser verdünnen. 

\parencite{KrausWeyermann2021b}
Stark divergierende Farbwerte für gleiche Schüttungen.
Gleiche Farbwerte für unterschiedliche Schüttungen die 
unterschiedliche Schüttungen verursachen.
1L = 2,65 EBC
EBC=0,377L
bei mehreren Malzen proportionale Anteil der Schüttung.
Malt Color Unit MCU=Malzfarbe[°L]*Schüttungsanteil[lb]/Volumen Anstellwürze [gal]
lb gal 
1MCU = 1 SRM (bis 6 SRM)
cEBC=Malzfarbe[EBC]*Gewicht[kg]/Gesamtgewicht Maische * Stw/10 
Je dunkler, desto mehr faktoren sind zu berechnen
sobald bierfarbe 10 SRM überschreitet wird lineare MCU unbrauchbar

CEBC = G/10+BT*D+CC
CW= G/10+CC

Krüger hat Zufärbungskorrekturen
1-2 EBC pro Stunde kochen
Morey hat sich zum Standard entwickelt.
Mosher Kommerzielle Bire mit bekannten Schüttungen und Farbwerten. über 10 SRM nich sehr zuverlässig.
Daniels; versuche mit Hobbysuden

Morey: abnehmende Genauigkeit der Lineare für dunkel logarithmisch der Lichtabsoption
anwendbar bis 50 SRM

\section*{Malt Color Unit (MCU)}

\parencite{Colby2000}
The HCU equation is similar to the SGP and IBU calculations, except it is missing a utilization or efciency term.
Essentially, it is only a measure of how much color is put into the beer from the grains. How that color is
changed by the brewing process is ignored.
Ignoring the effect of the brewing process on beer color is unfortunate, since many brewing processes affect
wort and beer color. Some factors increase color. These include: high hop rates, longer boil times, ner grain
crushes and oxidation from wort abuse or aging. Fermentation, in contrast, reduces the color of the wort.
The equation could be improved if a factor was added to account for changes in color due to the brewing
process. In fact, two factors would be needed. Some color changes are modications of the grain color added to
the beer. Loss of color during fermentation, for example, depends on the amount of color initially in the wort.
Other color changes do not depend on the initial beer color. Some of the wort darkening during the boil is due
to the sugars caramelizing. In solution, the sugars are colorless. The color that sugar caramelization contributes
does not come from the roasting of the grain.
So, an equation that took both of these factors into consideration would need two new variables. The resulting
equation would be:

NHCU=[W(lbs) * CR(degrees L)*TF + AF]/V(gallon)
NHBU is New Homebrew Color Units. The two new factors are TF and AF. TF is the transforming factor, the
extent that the grain color is modidfied. AF is the additive factor, the amount of color that appears during the
brewing process or aging that is not dependent on roasted grain-derived color.

\section*{Modelle zur Korrelation zwischen MCU und Bierfarbe}

Farbvorhersage

\subsection*{Mosher (1994)}

\parencite{Morey}:
\begin{equation}
\SRM = 0,3 \cdot \MCU + 4,7
\label{eq:mcumosher}
\end{equation}

Homebrew Color Unit (HCU), digramm
\parencite[34]{Mosher1994}

Metric Malt Color Units = l*kg/l
\parencite[258]{Mosher2015}

\subsection*{Daniels (1995)}

In seinem Buch „\citetitle{Daniels1996}“ hat Daniels die Korrelation zwischen SRM und MCU in \autoref{table:mcudaniels} veröffentlicht. \textcite[10]{Holle2010} verwendet die gleiche Tabelle für Farbvorhersagen. Die darauf basierende lineare \autoref{eq:mcudaniels} ist darin nicht enthalten, wird von \textcite{Morey} aber ebenfalls Daniels zugeschrieben. Eine in die Tabellenwerte eingepasste alternative Exponentationfunktion \autoref{eq:mcudanielsdruey} stammt von \textcite{Druey1998}.

\begin{table}[H]
\centering
\begin{tabular}{rrr}
\toprule
\multicolumn{1}{c}{\textbf{SRM}} & \multicolumn{1}{c}{\textbf{EBC}} & \multicolumn{1}{c}{\textbf{MCU}} \\
\midrule
1–10 & 2–20 & 1–10 \\
11–20 & 22–39 & 8–12 \\
21–30 & 41–59 & 11–15 \\
31–40 & 61–79 & 14–17 \\
41–50 & 81–98 & 17–20 \\
50–85 & 98–167 & 20–30 \\
>85 & >167 & >30 \\
\bottomrule
\end{tabular}
\caption{Korrelation zwischen SRM und MCU \parencite[61]{Daniels1996}}
\label{table:mcudaniels}
\end{table}

\begin{equation}
\SRM = 0,2 \cdot \MCU + 8,4
\label{eq:mcudaniels}
\end{equation}

\begin{equation}
\SRM = 1,73 \cdot \MCU^{0,64} - 0,267
\label{eq:mcudanielsdruey}
\end{equation}

\subsection*{Noonan (1996)}

\begin{table}[H]
\centering
\begin{tabular}{rrr}
\toprule
\multicolumn{1}{c}{\textbf{SRM}} & \multicolumn{1}{c}{\textbf{EBC}} & \multicolumn{1}{c}{\textbf{MCU}} \\
\midrule
1–10 & 2–20 & 1–10 \\
10,5 & 21 & 10,8 \\
11   & 22 & 11,6 \\
11,5 & 23 & 12,4 \\
12   & 24 & 13,3 \\
12,5 & 25 & 14,1 \\
13   & 26 & 14,9 \\
13,5 & 27 & 17,7 \\
14   & 28 & 18,6 \\
14,5 & 29 & 20,5 \\
15   & 30 & 22,4 \\
15,5 & 31 & 24,3 \\
16   & 32 & 26,2 \\
16,5 & 33 & 28,1 \\
17   & 33 & 30 \\
17,5 & 34 & 32,9 \\
18   & 35 & 35,8 \\
18,5 & 36 & 38,8 \\
19   & 37 & 41,9 \\
19,5 & 38 & 45 \\
20   & 39 & 47,8 \\
\bottomrule
\end{tabular}
\caption{Korrelation zwischen SRM und MCU \parencite[206]{Noonan1996}}
\label{table:mcunoonan}
\end{table}

\parencite{Druey1998}:
Kyle Druey
\begin{equation}
\SRM = 15,03 \cdot \MCU^{0,27} - 15,53
\label{eq:mcunoonandruey}
\end{equation}

\subsection*{Morey (1998)}

Heimbrauer Daniel Morey war unzufrieden mit der Farbvorhersage, die er in seiner Brausoftware auf Basis der MCU ohne zusätzliche Korrelation integriert hatte. Diese berechnete seiner Meinung nach zu hohe SRM Werte. Im mittlerweile nicht mehr erhältlichem Brewing Techniques Magazin erfuhr er dann 1995 von der Mosher und Daniels Modell. Zunächst wollte er alle Korrelationen in ein Berechnungsmodell zusammenführen, entschied sich aber dann dazu, die ihm vorliegenden Daten so zu modifizieren, dass es möglich war eine seinen Annahmen entsprechende Kurve einzupassen. \parencite{Smith2010}

Das von Morey veröffentlichte Modell beruht auf folgenden Annahmen: Bis zu einem Betrag von 10 entspricht die MCU ungefähr dem zu erwartenden SRM Messwert. Ab 10 MCU gibt das Daniels Modell eine bessere Korrelation, über 37 MCU hingegen das Mosher Modell. Das menschliche Auge kann keinen Farbunterschiede über 40 SRM mehr feststellen. Die \autoref{eq:mcumorey} wurde daher nur bis 50 SRM ausgelegt.
\parencite{Morey}

\begin{equation}
\SRM = 1,49 \cdot \MCU^{0,69}
\label{eq:mcumorey}
\end{equation}

\subsection*{Krüger (2019)}



\subsection*{Hanghofer (2019)}

\parencite[78]{Hanghofer2019}



\subsection*{Weyermann (2021)}

\parencite{KrausWeyermann2021b}

\begin{table}[H]
\centering
\begin{tabular}{rr}
\toprule
\multicolumn{1}{c}{\textbf{Stammwürze}} & \multicolumn{1}{c}{\textbf{CC [EBC]}} \\
\midrule
0–7 & 0 \\
7,1–10 & 3 \\
10,1–15 & 5 \\
15,1–20 & 7 \\
>20 & 0 \\
\bottomrule
\end{tabular}
\caption{TODO \parencite{KrausWeyermann2021b}}
\label{table:ccweyermann}
\end{table}

\subsection*{Modellvergleich}

\parencite{KrausWeyermann2021b}
20 Biere in Weyermann Braumanufaktur. Gemessen und Labo-Messung mit Formel verglichen.
wobei 1 Bier (Export hell) als ausreißwer entfernt wurde.
keine 100 prozen überinstimmung mit labormessung.
Weyermann und Krüger beste Ergebnis.

\begin{table}[H]
\centering
\begin{tabular}{lr}
\toprule
\multicolumn{1}{c}{\textbf{Modell}} & \multicolumn{1}{c}{\textbf{Mittlere Abw. [\%]}} \\
\midrule
Mosher & -57,2 \\
Daniels & -14,4 \\
Morey & -51,8 \\
Krüger & 6,9 \\
Weyermann & -2,1 \\
\bottomrule
\end{tabular}
\caption{TODO \parencite{KrausWeyermann2021b}}
\label{table:modelcompare}
\end{table}


\section*{Zusammenfassung}


\parencite{Bruecklmeier2018}
\parencite{Holle2010}

\parencite{Smith2008}
\parencite{Bies2010}
\parencite{Tucker2017}

\parencite{KrausWeyermann2021a}
\parencite{KrausWeyermann2021b}
\parencite{KrausWeyermann2021c}

\parencite{Lange2016}
\parencite{Caro2019}
\parencite{Daniels2012}

Die relevanten Eckpunkte dieses Artikels sind:

\begin{itemize}
\item TODO
\end{itemize}

\printbibliography[title=Quellen]

\end{document}