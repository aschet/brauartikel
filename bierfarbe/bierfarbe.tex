% Copyright 2022 Thomas Ascher
% SPDX-License-Identifier: CC-BY-SA-4.0

\documentclass[a4paper,parskip=half]{scrartcl}

\usepackage[T1]{fontenc}
\usepackage{mathpazo}
\usepackage[naustrian]{babel}
\usepackage{csquotes}
\usepackage{booktabs}
\usepackage{graphicx}
\usepackage{chemformula}
\usepackage{icomma}
\usepackage{gensymb}
\usepackage{float}
\usepackage[section]{placeins}
\usepackage[style=apa,backend=biber]{biblatex}

\usepackage[hidelinks,pdfencoding=auto,
  pdfauthor={Thomas Ascher},
  pdfusetitle,
  pdfkeywords={Bier,Farbe,SRM,IBU,EBC,Morey}]{hyperref}
\usepackage{microtype}

\setkomafont{disposition}{\normalfont\bfseries}

\addto\extrasnaustrian{
\def\figureautorefname{Abb.}
\def\tableautorefname{Tab.}
\def\equationautorefname{Gl.}
}

\addto\captionsnaustrian{
\renewcommand{\figurename}{Abb.}
\renewcommand{\tablename}{Tab.}
}

\NewBibliographyString{gethesis}
\DefineBibliographyStrings{naustrian}{
  mathesis = {Masterarbeit},
  gethesis = {Diplomarbeit},
}

\newcommand{\MCU}{\mathit{MCU}}
\newcommand{\SRM}{\mathit{SRM}}
\newcommand{\umin}{\:[\textrm{min}]}
\newcommand{\uden}{\:[\text{g/cm³}]}
\newcommand{\uper}{\:[\text{\%}]}
\newcommand{\uli}{\:[\text{l}]}
\newcommand{\ume}{\:[\text{m}]}
\newcommand{\ucon}{\:[\text{mg/l}]}
\newcommand{\FKd}{F_{\mathit{d}}}

\title{Es wird bunt: Bierfarbe aus der Schüttung berechnen}
\author{Thomas Ascher <thomas.ascher@gmx.at>}
\date{\today, \href{http://creativecommons.org/licenses/by-sa/4.0/}{CC BY-SA 4.0}}

\addbibresource{bierfarbe.bib}

\begin{document}
\maketitle

\section*{Einleitung}

„\href{https://bieranalyse.de}{Bieranalyse Fuchs}“

%In \autoref{table:parameters} sind die vom \citeauthor{BJCP2015} und die von
%\citeauthor{Strottner1999} genannten brautechnischen Parameter erfasst.
%Die Angaben der Brewers Association Stil-Richtlinien sind annähernd
%deckungsgleich \parencites[24]{BA2021}. 

\begin{table}[H]
\centering
\begin{tabular}{rrr}
\toprule
\multicolumn{1}{c}{\textbf{Beschreibung}} & \multicolumn{1}{c}{\textbf{EBC}} & \multicolumn{1}{c}{\textbf{SRM}} \\
\midrule
Stroh & - & 2–3 \\
Gelb & - & 3–4 \\
Gold & - & 5–6 \\
Berstein & - & 6–9 \\
Dunkles Bernstein / Helles Kupfer & - & 10–14 \\
Kupfer & - & 14–17 \\
Dunkles Kupfer / Helles Braun & - & 17–18 \\
Braun & - & 19–22 \\
Dunkles Braun & - & 22–30 \\
Sehr dunkles Braun & - & 30–35 \\
Schwarz & - & 30+ \\
Blickdichtes Schwarz & - & 40+ \\
\bottomrule
\end{tabular}
\caption{TODO \parencite{BJCP2015}}
\label{table:bjcpcolor}
\end{table}

\parencite{BA2021}


\parencite[11]{Noonan1996}
Color may not be stated as SRM/ASBC/°L units.
Where EBC units are given, the formula ASBC = (°EBC
+ 1.2)/2.65 gives reasonable but not entirely accurate
transposition. The color for English malts is often given as
Color IOB, EBC method. IOB color is only approximately
80 percent of EBC color. Where IOB units are given, the
formula ASBC = [(IOB/.80) + 1.2]/2.65 can be used.
HCU color units are a summary measurement of
wort color that is equal to the sum of the SRM/°Lovibond
of all the malts used to make a wort. HCU units and SRM
wort color are comparable up to about 10 °SRM; after that,
corrections must be made. See the conversion chart in
table 15 to convert from HCU color to approximate °SRM.
The color ranges for most malt types vary widely,
depending on the country of origin and the maltster. From
the maltster’s point of view, it is of some advantage to
have a “unique” color range, since it requires any brewer
using the malt to make serious adjustments in the brew
house program in order to switch suppliers. To some
extent, this diversity also benefits brewers, but on the
whole, color variation in the same nominal malt types is a
problem for the brewer seeking predictable results.

\section*{Messung von Bier- und Malzfarbe}



\section*{Malt Color Unit (MCU)}

\parencite{Colby2000}
The HCU equation is similar to the SGP and IBU calculations, except it is missing a utilization or efciency term.
Essentially, it is only a measure of how much color is put into the beer from the grains. How that color is
changed by the brewing process is ignored.
Ignoring the effect of the brewing process on beer color is unfortunate, since many brewing processes affect
wort and beer color. Some factors increase color. These include: high hop rates, longer boil times, ner grain
crushes and oxidation from wort abuse or aging. Fermentation, in contrast, reduces the color of the wort.
The equation could be improved if a factor was added to account for changes in color due to the brewing
process. In fact, two factors would be needed. Some color changes are modications of the grain color added to
the beer. Loss of color during fermentation, for example, depends on the amount of color initially in the wort.
Other color changes do not depend on the initial beer color. Some of the wort darkening during the boil is due
to the sugars caramelizing. In solution, the sugars are colorless. The color that sugar caramelization contributes
does not come from the roasting of the grain.
So, an equation that took both of these factors into consideration would need two new variables. The resulting
equation would be:

NHCU=[W(lbs) * CR(degrees L)*TF + AF]/V(gallon)
NHBU is New Homebrew Color Units. The two new factors are TF and AF. TF is the transforming factor, the
extent that the grain color is modidfied. AF is the additive factor, the amount of color that appears during the
brewing process or aging that is not dependent on roasted grain-derived color.

\section*{Modelle zur Korrelation zwischen MCU und Bierfarbe}


\subsection*{Mosher (1994)}

\parencite{Morey}:
\begin{equation}
\SRM = 0,3 \cdot \MCU + 4,7
\label{eq:mcumosher}
\end{equation}

Homebrew Color Unit (HCU), digramm
\parencite[34]{Mosher1994}

Metric Malt Color Units = l*kg/l
\parencite[258]{Mosher2015}

\subsection*{Daniels (1996)}

\parencite{Daniels1996}

\begin{equation}
\SRM = 0,2 \cdot \MCU + 8,4
\label{eq:mcudaniels}
\end{equation}

\begin{table}[H]
\centering
\begin{tabular}{rr}
\toprule
\multicolumn{1}{c}{\textbf{SRM}} & \multicolumn{1}{c}{\textbf{MCU}} \\
\midrule
1–10 & 1–10 \\
11–20 & 8–12 \\
21–30 & 11–15 \\
31–40 & 14–17 \\
41–50 & 17–20 \\
50–85 & 20–30 \\
>85 & >30 \\
\bottomrule
\end{tabular}
\caption{Ungefähre Korrelation zwischen SRM und MCU \parencite[61]{Daniels1996}}
\label{table:mcudaniels}
\end{table}

\parencite{Morey}:
Kyle Druey
\begin{equation}
\SRM = 1,73 \cdot \MCU^{0,64} - 0,27
\label{eq:mcudanielsdruey}
\end{equation}

\parencite[10]{Holle2010}

\subsection*{Noonan (1996)}

\begin{table}[H]
\centering
\begin{tabular}{rr}
\toprule
\multicolumn{1}{c}{\textbf{SRM}} & \multicolumn{1}{c}{\textbf{MCU}} \\
\midrule
1–10 & 1–10 \\
10,5 & 10,8 \\
11   & 11,6 \\
11,5 & 12,4 \\
12   & 13,3 \\
12,5 & 14,1 \\
13   & 14,9 \\
13,5 & 17,7 \\
14   & 18,6 \\
14,5 & 20,5 \\
15   & 22,4 \\
15,5 & 24,3 \\
16   & 26,2 \\
16,5 & 28,1 \\
17   & 30 \\
17,5 & 32,9 \\
18   & 35,8 \\
18,5 & 38,8 \\
19   & 41,9 \\
19,5 & 45 \\
20   & 47,8 \\
\bottomrule
\end{tabular}
\caption{Ungefähre Korrelation zwischen SRM und MCU \parencite[206]{Noonan1996}}
\label{table:mcunoonan}
\end{table}

\parencite{Morey}:
Kyle Druey
\begin{equation}
\SRM = 15,03 \cdot \MCU^{0,27} - 15,53
\label{eq:mcunoonandruey}
\end{equation}

\parencite{Druey1998}

\subsection*{Morey}

\parencite{Morey}:
\begin{equation}
\SRM = 1,49 \cdot \MCU^{0,69}
\label{eq:mcumorey}
\end{equation}



\parencite{Morey}
\parencite{Smith2010}

\subsection*{Hanghofer}

\parencite[78]{Hanghofer2019}

\subsection*{Krüger (2019)}

\subsection*{Weyermann (2021)}

\section*{Zusammenfassung}


\parencite{Bruecklmeier2018}
\parencite{Holle2010}

\parencite{Smith2008}
\parencite{Bies2010}
\parencite{Tucker2017}

\parencite{KrausWeyermann2021a}
\parencite{KrausWeyermann2021b}
\parencite{KrausWeyermann2021c}

\parencite{Lange2016}
\parencite{Caro2019}
\parencite{Daniels2012}

Die relevanten Eckpunkte dieses Artikels sind:

\begin{itemize}
\item TODO
\end{itemize}

\printbibliography[title=Quellen]

\end{document}